
In questa sezione verrà riportato il bilancio delle ore dei vari componenti del gruppo in base ai ruoli sostenuti. Il bilancio sarà così analizzato:
\begin{itemize}
	\item {\bfseries in positivo}: se il numero delle ore effettuate è minore rispetto a quelle preventivate;
	\item {\bfseries in negativo}: se il numero delle ore effettuate è maggiore rispetto a quelle preventivate;
	\item {\bfseries in pari}: se il numero delle ore effettuate è uguale rispetto a quelle preventivate. \\
\end{itemize}

\subsection {Consuntivo di periodo: Analisi dei requisiti}
Verranno analizzate le ore di lavoro effettuate durante l'analisi dei requisiti, destinate principalmente allo studio personale; quindi non rendicontate.
\subsubsection{Prospetto orario}
	\begin{table} [h!]
	\begin{center}
		\rowcolors{2}{gray!25}{gray!6}
		\begin{tabular} { m{6 cm} c c c c c c c c }
			\rowcolor{lightgray}
			\textbf{Nome} & \textbf{Re} & \textbf{Am} & \textbf{An} & \textbf{Pg} &\textbf{Pr} & \textbf{Ve} & \textbf{Totale} \\ 
			Matteo Alba & 0 & 8(-5) &8(+4) & 0 & 0 & 10(+1) & 26  \\ 
			Giacomo Bulbarelli & 7(+4) & 6(-1) & 8(-2) & 0 & 0 & 5(-1) & 26 \\ 
			Alessandro Chimetto & 7 & 0(-4) & 11(+4) & 0 & 0 & 8 & 26 \\
			Alessandro Dindinelli & 0 & 4(-1) & 12(+1) & 0 & 0 & 10 & 26 \\
			Lucia Fenu & 0(-3) & 6(-4) & 10(+6) & 0 & 0 & 10(+1) & 26 \\
			Paolo Scanferlato & 5(+1) & 7(-4) & 6(+2) & 0 & 0 & 8(+1) & 26 \\
			Valton Tahiraj & 6(+3) & 5(-2) &7(+1) & 0 & 0 & 8(-2) & 26 \\
			\textbf{Ore Totali Ruolo} & 25(+5) & 36(-21) & 62(+16) & 0 & 0 & 59 & 182\\
		
		\end{tabular}
		\caption{Consuntivo - Analisi dei requisiti - ore per persona/ruolo}
	\end{center}
\end{table}

	\begin{figure} [h!]
	\centering
	\includegraphics[width=0.7\textwidth]{res/img/grafici/consuntivo-barre_ ore analisi requisiti.png}
	\caption{Consuntivo - Analisi dei requisiti - ore di suddivisione dei ruoli} 
\end{figure}


\newpage
\subsubsection{Prospetto economico}
Nella tabella sono riportate le ore totali per ciascun costo seguendo le indicazioni descritte all'inizio della sezione.
Allo stesso modo, il costo in euro sarà così analizzato:
\begin{itemize}
	\item {\bfseries in positivo}: la somma finale è minore rispetto a quella preventivata, il proponente ha risparmiato denaro;
	\item {\bfseries in negativo}: la somma finale è maggiore rispetto a quella preventivata, il proponente ha investito più denaro;
	\item {\bfseries in pari}: la somma finale è invariata rispetto a quanto preventivato. \\
\end{itemize}
\begin{table} [h!]
	\begin{center}
		\rowcolors{2}{gray!25}{gray!6}
		\begin{tabular} { m{3 cm} c c c  }
			\rowcolor{lightgray}
			\textbf{Ruolo} & \textbf{Ore} & \textbf{Costo in \euro} \\
			Responsabile & 25(+5) & 750,00 (+150,00) \\
			Amministratore & 36(-21) & 720,00 (-420,00)  \\
			Analista & 62(+16) & 1550,00 (+400,00) \\
			Progettista & 0 & 0 \\
			Programmatore & 0 & 0  \\
			Verificatore & 59 & 885,00  \\
			\textbf{Totale} & 182  & 3905,00 (+130,00) \\
			
		\end{tabular}
		\caption{Consuntivo - Analisi dei requisiti - costo per ruolo}
	\end{center}
\end{table}

\begin{figure} [h!]
	\centering
	\includegraphics[width=0.6\textwidth]{res/img/grafici/consuntivo- torta_ costo_per_ora- analisi dei requisiti.png}
	\caption{Consuntivo - Analisi dei requisiti - costo per ruolo} 
\end{figure}

\newpage 

\subsubsection{Conclusioni}
Il gruppo nel complesso, ha lavorato secondo le ore preventivate seppur con alcune modifiche nella rotazione dei ruoli.
Nello specifico:
\begin{itemize}
	\item {\bfseries responsabile}: le ore previste, nel complesso, sono state rispettate. Per parallelizzare meglio il lavoro, è stato incluso un membro all'interno del ruolo (rispetto al preventivo), alleggerendo anche il carico ad altri membri;
	\item {\bfseries amministratore}: il lavoro previsto per la stesura dei documenti ha richiesto più ore rispetto a quelle preventivate, distribuite su tutti i membri;
	\item {\bfseries analista}: la stesura dell'\dext{Analisi dei requisiti} ha richiesto meno tempo rispetto a quanto preventivato e dunque, le ore mancanti, sono state distribuite nei restanti ruoli che ne necessitavano di più;
	\item {\bfseries verificatore}: grazie all'attenzione riposta nella gestione dei documenti, è stato svolto il ruolo di verificatore nelle ore previste.\\
		
	A causa della variazione oraria di alcuni ruoli, il totale prevede una spesa inferiore di \euro 130,00 rispetto a quanto preventivato.
	Tuttavia, la fase di analisi dei requisiti, non è rendicontata.
	
\end{itemize}



	

	
\newpage	
	
\subsection{Consuntivo di periodo: Analisi di dettaglio}
Verranno qui di seguito analizzate le ore per ruolo di ogni membro del gruppo sulla base del preventivo in sezione 5.2.

\subsubsection{Prospetto orario}
	\begin{table} [h!]
	\rowcolors{2}{gray!25}{gray!6}
	\begin{center}
		\begin{tabular} { m{3.5cm} c c c c c c c }
			\rowcolor{lightgray}
			\textbf{Nome} & \textbf{Re} & \textbf{Am} & \textbf{An} & \textbf{Pg} & \textbf{Pr} & \textbf{Ve} & \textbf{Totale} \\
			Matteo Alba               & 2(+2) & 0(-2) & 0      & 0  & 0  & 4      & 6 \\
			Giacomo Bulbarelli        & 0     & 2     & 4      & 0  & 0  & 0      & 6 \\
			Alessandro Chimetto       & 0(-1) & 5(+2) & 1(-1)  & 0  & 0  & 0      & 6 \\
			Alessandro Dindinelli     & 3     & 0(-2) & 1(+1)  & 0  & 0  & 2(+1)  & 6 \\
			Lucia Fenu                & 1     & 0(-2) & 3(+2)  & 0  & 0  & 2      & 6 \\
			Paolo Scanferlato         & 0     & 0     & 2      & 0  & 0  & 4      & 6 \\
			Valton Tahiraj            & 0     & 0     & 3(+1)  & 0  & 0  & 3(-1)  & 6\\
			\textbf{Ore totali Ruolo} & 6(+1) & 7(-4) & 14(+3) & 0  & 0  & 15     & 42
		\end{tabular}
		\caption{Consuntivo - Analisi di dettaglio - ore per persona/ruolo}
	\end{center}
\end{table}

	\begin{figure} [h!]
	\centering
	\includegraphics[width=0.7\textwidth]{res/img/grafici/consuntivo-barre- ore analisi dettaglio.png}
	\caption{Consuntivo - Analisi di dettaglio - ore di suddivisione dei ruoli} 
\end{figure}

\newpage
\subsubsection{Prospetto economico}
Nella tabella sono riportate le ore totali per ciascun costo seguendo le indicazioni descritte all'inizio della sezione.

\begin{table} [h!]
	\begin{center}
		\rowcolors{2}{gray!25}{gray!6}
		\begin{tabular} { m{3 cm} c c c  }
			\rowcolor{lightgray}
			\textbf{Ruolo}  & \textbf{Ore} & \textbf{Costo in \euro} \\
			Responsabile    & 6(+1)      & 180,00 (+30,00) \\
			Amministratore  & 7(-4)      & 140,00 (-80,00)  \\
			Analista        & 14(+3)     & 350,00 (+75,00) \\
			Progettista     & 0          & 0 \\
			Programmatore   & 0          & 0  \\
			Verificatore    & 15         & 225,00  \\
			\textbf{Totale} & 42         & 895,00 (+25,00) \\
			
		\end{tabular}
		\caption{Consuntivo - Analisi di dettaglio - costo per ruolo}
	\end{center}
\end{table}
	\begin{figure} [h!]
	\centering
	\includegraphics[width=0.7\textwidth]{res/img/grafici/consuntivo-torta-analisi di dettaglio.png}
	\caption{Consuntivo - Analisi di dettaglio - costo per ruolo} 
\end{figure}

\subsubsection{Conclusioni }
In conclusione, si osserva un risparmio orario nei ruoli di Responsabile e Analista compensato dall'investimento orario parallelo nel ruolo di Amministratore.\\
Dal punto di vista economico si osserva un leggero risparmio dei costi previsti. \\
In generale, il preventivo era corretto in un ottica generale ma non erano previsti oneri amministrativi come quelli incontrati.


\newpage

\subsection{Consuntivo di periodo: Codifica \glock{Technlogy Baseline}}
Verranno qui di seguito analizzate le ore per ruolo di ogni membro del gruppo sulla base del preventivo in sezione 5.3.
\subsubsection{Prospetto orario}
\begin{table} [h!]
	\rowcolors{2}{gray!25}{gray!6}
	\begin{center}
		\begin{tabular} { m{3.5cm} c c c c c c c }
			\rowcolor{lightgray}
			\textbf{Nome} & \textbf{Re} & \textbf{Am} & \textbf{An} & \textbf{Pg} & \textbf{Pr} & \textbf{Ve} & \textbf{Totale} \\
			Matteo Alba               & 8(+3)      & 0(-3)    & 4      & 3  & 5  & 6      & 27 \\
			Giacomo Bulbarelli        & 3     & 0     & 5(+2)  & 7  & 5(-4)  & 7(+2)      & 27 \\
			Alessandro Chimetto       & 0     & 2    & 4(+2)       & 6(-1)  & 6(-1)  & 9     & 27 \\
			Alessandro Dindinelli     & 2     & 4    & 0       & 7  & 7  & 7  & 27 \\
			Lucia Fenu                & 0(-2)  & 2(-2)  & 3      & 7(+2)   & 7(+2)   & 9      & 27 \\
			Paolo Scanferlato         & 2      & 3      & 3      & 6       & 4       & 9      & 27 \\
			Valton Tahiraj            & 0      & 3      & 4(+2)  & 3(-2)   & 6(-2)   & 11(+2) & 27\\
			\textbf{Ore totali Ruolo} & 15(+1) & 14(-5) & 23(+6) & 39(-1)  & 40(-5)  & 58(+4) & 189\\
		\end{tabular}
		\caption{Consuntivo - Codifica \glock{Technlogy Baseline} - ore per persona/ruolo}
	\end{center}
\end{table}
	\begin{figure} [h!]
	\centering
	\includegraphics[width=0.7\textwidth]{res/img/grafici/consuntivo-barre-tb.png}
	\caption{Consuntivo - Codifica \glock{Technlogy Baseline} - costo per ruolo} 
\end{figure}
\newpage
\subsubsection{Prospetto economico}
Nella tabella sono riportate le ore totali per ciascun costo seguendo le indicazioni descritte all'inizio della sezione.

\begin{table} [h!]
	\begin{center}
		\rowcolors{2}{gray!25}{gray!6}
		\begin{tabular} { m{3 cm} c c c  }
			\rowcolor{lightgray}
			\textbf{Ruolo}  & \textbf{Ore} & \textbf{Costo in \euro} \\
			Responsabile    & 15(+1)      & 450,00 (+30,00) \\
			Amministratore  & 14(-5)      & 280,00 (-100,00)  \\
			Analista        & 23(+6)     & 575,00 (+150,00) \\
			Progettista     & 39(-1)          & 858,00 (-22,00) \\
			Programmatore   & 40(-5)        & 600,00 (-75,00) \\
			Verificatore    & 58(+4)        & 870,00 (+60,00) \\
			\textbf{Totale} & 189         & 3633,00 (+43,00) \\
			
		\end{tabular}
		\caption{Consuntivo - Codifica \glock{Technlogy Baseline} - costo per ruolo}
	\end{center}
\end{table}
\begin{figure} [h!]
	\centering
	\includegraphics[width=0.7\textwidth]{res/img/grafici/consuntivo-torta-tb.png}
	\caption{Consuntivo - Codifica \glock{Technlogy Baseline} - costo per ruolo} 
\end{figure}
\subsubsection{Conclusioni}
L'impegno orario preventivato corretto rispetto al totale monte ore.\\ Decisamente meno il prospetto orario di dettaglio sui singoli ruoli: il lavoro svolto si e' incentrato molto di piu' sulla codifica e sulla progettazione, mentre (complice un workflow funzionale e una necessita' meno stringente di verificare il codice prodotto, i ruoli di Analista e Verificatore sono risultati meno importanti del previsto).\\
Il risparmio anche in questo caso e' significativo nel sottolineare come il preventivo orario sia nel corretto rispetto al monte ore finale ma da rivedere rispetto all'impegno che il singolo ruolo ha nella fase specifica di cui si fa preventivo. \\ 

