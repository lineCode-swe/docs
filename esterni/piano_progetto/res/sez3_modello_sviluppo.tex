Il modello di ciclo di vita utilizzato è fortemente ispirato al modello
incrementale. Lo sviluppo avviene per incrementi dove ogni incremento rilascia parte delle funzionalità richieste.
\\\\
I requisiti vengono inizialmente classificati in base alla loro priorità e viene
creata un'architettura che permetta la successiva aggiunta di molteplici
incrementi. Ciò permette di sviluppare fin da subito un insieme di funzionalità
fondamentali e stabili. Segue il soddisfacimento di requisiti meno importanti e
meno chiari che, nel tempo, riescono a stabilizzarsi ed integrarsi con quanto già
realizzato.
\\\\
Il modello presenta i seguenti vantaggi:
\begin{itemize}
    \item ogni incremento può essere verificato prima dell'integrazione rendendo la verifica economica perché la parte già realizzata è stata precedentemente verificata;
    \item ogni incremento rende disponibili nuove funzionalità e chiarisce i requisiti per incrementi successivi;
    \item ad ogni incremento è possibile ricevere un feedback da committente o proponente;
    \item una volta realizzato ed integrato, ogni incremento riduce la probabilità che il progetto fallisca.
\end{itemize}

\noindent
Nonostante il modello incrementale non preveda che gli incrementi già realizzati
possano subire successivi interventi correttivi o di raffinamento (come invece
accadrebbe in un modello iterativo) e vista la mancanza di sufficiente esperienza
in progetti software di tali dimensioni, il gruppo si riserva il diritto di
effettuare iterazioni potenzialmente distruttive sul prodotto già realizzato.
Questa pratica, che comunque viene considerata non desiderabile, verrà messa in
atto solo dopo un eventuale feedback negativo da parte del proponente durante il
confronto che segue la realizzazione di un iterazione.
