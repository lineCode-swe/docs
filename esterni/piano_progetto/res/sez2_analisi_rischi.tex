	Al fine di garantire il successo dell'attività di progetto, risulta necessario definire un piano per la gestione dei rischi. \\
	Nello specifico, tale piano è composto da quattro attività:
	\begin{enumerate}
		\item \textbf{Identificazione dei rischi}: attività che si pone l'obiettivo di identificare e definire i rischi che possono insorgere nel corso dell'attività di progetto. In particolare, essa si propone di dare un identificativo e una descrizione ai rischi che emergono durante l'attività.
		\item \textbf{Analisi dei rischi}: attività di analisi che consiste nello studio di ciascuno dei rischi identificati, al fine di evidenziare:
		\begin{itemize}
			\item probabilità di occorrenza;
			\item fattore di rischio;
			\item conseguenze che il manifestarsi del rischio potrebbe avere.
		\end{itemize}
		\item \textbf{Pianificazione di controllo e mitigazione}: attività che si propone di:
			\begin{itemize}
				\item stabilire le modalità attraverso cui prevenire il verificarsi di un rischio;
				\item stabilire le modalità attraverso cui reagire al verificarsi di un rischio.
			\end{itemize}
			I prodotti di questa attività sono dunque:
			\begin{itemize}
				\item metodi di controllo dei rischi;
				\item piano di contingenza.
			\end{itemize}
		\item \textbf{Monitoraggio dei rischi}: attività che si propone di monitorare i rischi al fine di prevenirli, e di permettere una reazione tempestiva nel caso in cui questi si verifichino. Il prodotto di tale attività sono delle metodologie di rilevamento di rischi.
	\end{enumerate}
	
	I rischi rilevati nel contesto specifico dell'attività progettuale proposta sono riassumibili nelle seguenti categorie:
	\begin{enumerate}
		\item rischi legati alle dinamiche di gruppo;
		\item rischi legati all'organizzazione del lavoro;
		\item rischi legati ai requisiti;
		\item rischi legati ai mezzi tecnologici.
	\end{enumerate}

\subsection{Rischi legati alle dinamiche di gruppo}
	\subsubsection{Inesperienza del gruppo}	
	\begin{enumerate}
		\item \textbf{Descrizione e conseguenze}: nessun membro del gruppo ha mai affrontato prima un progetto come questo dal punto di vista dell'impegno richiesto, del tempo richiesto, delle conoscenze da acquisire e del numero di persone con cui è necessario coordinarsi. Ciò può portare a conseguenze non desiderabili, quali ritardi e difficoltà nel sincronizzare il lavoro;
		\item \textbf{Probabilità di occorrenza}: alta;
		\item \textbf{Fattore di rischio}: alto;
		\item \textbf{Rilevamento}: ciascun membro che dovesse incontrare difficoltà provvederà a comunicarlo immediatamente al Responsabile. Se un qualsiasi compito tarda ad essere ultimato, è onere del Responsabile informarsi sullo stato dei lavori. 
		\item \textbf{Controllo}: il Responsabile aiuterà i membri che incontrano difficoltà attraverso delle attività di auto-miglioramento.
		\item \textbf{Piano di contingenza}: il membro in difficoltà consulterà il Responsabile, il quale provvederà a riportare la problematica al gruppo chiedendo aiuto ad un altro componente che andrà a fornire supporto nell'analisi delle criticità incontrate e nella risoluzione del problema affrontato, accelerando il completamento del lavoro.
	\end{enumerate}
	
	\subsubsection{Indisponibilità}
	\begin{enumerate}
		\item \textbf{Descrizione e conseguenze}: tutti i membri del gruppo sono studenti a tempo pieno. Impegni Universitari e personali potrebbero portare all'indisponibilità momentanea di alcuni membri, portando a ritardi e a complicazioni del lavoro da svolgere; 
		\item \textbf{Probabilità di occorrenza}: media;
		\item \textbf{Fattore di rischio}: medio;
		\item \textbf{Rilevamento}: qualora un membro risultasse indisponibile, provvederà a comunicarlo tempestivamente al Responsabile;
		\item \textbf{Controllo}: il Responsabile provvederà a informarsi della disponibilità dei singoli membri e a coordinare il gruppo di conseguenza;
		\item \textbf{Piano di contingenza}: nel caso in cui un membro risultasse indisponibile, il Responsabile provvederà a comunicarlo al gruppo e a coordinare la nuove distribuzione del lavoro sugli altri membri disponibili.
	\end{enumerate}
	
	\subsubsection{Contrasti interni}
	\begin{enumerate}
		\item \textbf{Descrizione e conseguenze}: i membri del gruppo non si conoscevano prima di intraprendere questo progetto. Il numero elevato di componenti del gruppo unito alla scarsa conoscenza che i membri hanno reciprocamente sia a livello personale che lavorativo potrebbero portare all'insorgere contrasti interni al gruppo con conseguente deterioramento del clima lavorativo;
		\item \textbf{Probabilità di occorrenza}: bassa;
		\item \textbf{Fattore di rischio}: molto alto;
		\item \textbf{Rilevamento}: il Responsabile monitorerà costantemente le interazioni tra i membri e li inviterà a informarlo nel caso in cui vi siano delle difficoltà;
		\item \textbf{Controllo}: il Responsabile andrà a moderare gli incontri tra i membri e avrà l'onere di assicurare un ambiente di lavoro sano per tutti. 
		\item \textbf{Piano di contingenza}: nel caso in cui vi siano dei conflitti interni tra i membri, il Responsabile si proporrà di moderare un incontro tra i membri in conflitto (escludendo eventualmente i membri non interessati) e fungerà da mediatore nel corso del suddetto incontro.
	\end{enumerate}
	
\subsection{Rischi legati all'organizzazione del lavoro}
	\subsubsection{Pianificazione delle attività}
	\begin{enumerate}
		\item \textbf{Descrizione e conseguenze}: ciascuna attività ha un costo in termini di tempo. Un errata pianificazione delle stesse potrebbe portare a rallentamenti e compromettere lo sviluppo; 
		\item \textbf{Probabilità di occorrenza}: media;
		\item \textbf{Fattore di rischio}: alto;
		\item \textbf{Rilevamento}: nel caso in cui un membro non sia in grado di svolgere gli incarichi assegnatigli nei tempi previsti, provvederà a comunicarlo tempestivamente al Responsabile;		
		\item \textbf{Controllo}: ad ogni incontro ciascun membro notificherà il Responsabile e gli altri membri circa lo stato dei propri lavori e si provvederà a fare un confronto tra i tempi di esecuzione del lavoro previsti e quelli effettivi; 
		\item \textbf{Piano di contingenza}: il Responsabile provvederà a coordinare il gruppo per redistribuire il lavoro in eccesso e moderare la nuova pianificazione dei lavori.
	\end{enumerate}
	
\subsection{Rischi legati ai requisiti}
	\subsubsection{Analisi dei rischi incompleta o incorretta}
	\begin{enumerate}
		\item \textbf{Descrizione e conseguenze}: data l'inesperienza del gruppo è possibile che il prodotto dell'analisi dei requisiti sia incompleto o incorretto. Tale scenario porterebbe a notevoli rallentamenti dei lavori;
		\item \textbf{Probabilità di occorrenza}: media;
		\item \textbf{Fattore di rischio}: medio;
		\item \textbf{Rilevamento}: in caso di dubbi o incertezze in sede di analisi, ciascuno dei membri è tenuto a comunicarlo al Responsabile e agli altri membri del gruppo;		
		\item \textbf{Controllo}: l'analisi dei requisiti verrà periodicamente rivista e controllata in fase di redazione al fine di migliorarla e raffinarla continuativamente.
		\item \textbf{Piano di contingenza}: nel caso in cui vengano evidenziate imprecisioni o inesattezze da parte del proponente in sede di revisione il gruppo provvederà a correggere gli errori nel minor tempo possibile.
	\end{enumerate}
	
	\subsubsection{Modifica dei requisiti}
	\begin{enumerate}
		\item \textbf{Descrizione e conseguenze}: è possibile che il proponente SanMarco Informatica decida di aggiungere dei requisiti in corso d'opera, portando alla necessità di modificare l'analisi dei requisiti precedentemente svolta; 
		\item \textbf{Probabilità di occorrenza}: bassa;
		\item \textbf{Fattore di rischio}: medio;
		\item \textbf{Rilevamento}: il gruppo si propone di contattare in maniera cadenzata il proponente durante lo sviluppo per mantenere sincronizzate le due parti;
		\item \textbf{Controllo}: il gruppo si propone di tracciare quanto viene deciso e discusso con il proponente dopo ogni incontro;
		\item \textbf{Piano di contingenza}: nel caso in cui sia necessario modificare dei requisiti, il gruppo provvederà a svolgere le attività di analisi e correzione della documentazione nel minor tempo possibile.
	\end{enumerate}	
	
\subsection{Rischi legati ai mezzi tecnologici}
	\subsubsection{Tecnologie da usare}		
	\begin{enumerate}
		\item \textbf{Descrizione e conseguenze}: le tecnologie da usare sono varie e spesso non conosciute da tutti i membri del gruppo. I tempi di apprendimento di tali tecnologie potrebbero portare a ritardi e difficoltà nello svolgimento del progetto;
		\item \textbf{Probabilità di occorrenza}: alta;
		\item \textbf{Fattore di rischio}: alto;
		\item \textbf{Rilevamento}: i membri che hanno delle difficoltà a interfacciarsi con una specifica tecnologia provvederanno a comunicarlo tempestivamente al responsabile del gruppo;
		\item \textbf{Controllo}: il Responsabile provvederà a coordinare i membri rispetto alle conoscenze necessarie per poter utilizzare una determinata tecnologia;  
		\item \textbf{Piano di contingenza}: al rilevamento di una difficoltà nell'interfacciarsi con una specifica tecnologia, il Responsabile provvederà a indirizzare i componenti verso le risorse utili e, qualora fosse necessario, coordinare una riunione per compensare e correggere le criticità che i membri presentano.
	\end{enumerate}	
	
	\subsubsection{Problemi Hardware}
	\begin{enumerate}
		\item \textbf{Descrizione e conseguenze}: ciascuno dei componenti lavora su macchine private. Un guasto a una di queste macchine potrebbe portare a rallentamenti del lavoro o perdita di dati;
		\item \textbf{Probabilità di occorrenza}: media;
		\item \textbf{Fattore di rischio}: basso;
		\item \textbf{Rilevamento}: ciascun membro che presenta dei problemi con la propria macchina provvede a comunicarlo al Responsabile;		
		\item \textbf{Controllo}: ciascun membro cerca, per quanto possibile, di avere cura della propria attrezzatura.
		\item \textbf{Piano di contingenza}: in caso di problematiche specifiche i membri sono tenuti a comunicarlo tempestivamente per permettere di organizzare il lavoro di conseguenza. Provvederanno poi a sostituire le componenti corrotte. 
	\end{enumerate}
	
	\subsubsection{Problemi Software}
	\begin{enumerate}
		\item \textbf{Descrizione e conseguenze}: il gruppo si appoggia a piattaforme online per la condivisione dei dati e per la comunicazione tra i membri. Data anche l'attuale situazione sanitaria che obbliga a lavorare per via telematica, un malfunzionamento di tali piattaforme potrebbe compromettere la comunicazione e il lavoro del gruppo;
		\item \textbf{Probabilità di occorrenza}: bassa;
		\item \textbf{Fattore di rischio}: alto;
		\item \textbf{Rilevamento}: i membri del gruppo monitoreranno le piattaforme utilizzate;
		\item \textbf{Controllo}: non attuabile, non si ha controllo sulle piattaforme esterne; 
		\item \textbf{Piano di contingenza}: non pianificabile a priori. Si applica un azione reattiva e correttiva nel caso dovessero verificarsi malfunzionamenti.
	\end{enumerate}		
