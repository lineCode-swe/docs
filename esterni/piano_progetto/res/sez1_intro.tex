\subsection{Scopo del documento}
	Lo scopo del documento è quello di pianificare il modo con cui verranno svolte le attività dai membri del gruppo lineCode. Verranno fornite delle descrizioni delle attività scelte, analizzandone i rischi ed assegnandole ai vari membri del gruppo.
	In particolare il documento coprirà i seguenti punti:
	\begin{itemize}
		\item analisi dei rischi relativi al progetto;
		\item organizzazione dei tempi e delle attività;
		\item stima preventiva delle risorse da impiegare;
		\item calcolo del consuntivo di utilizzo delle risorse;
		\item gestione dei rischi presentatisi.
	\end{itemize}

\subsection{Scopo del Prodotto}
	Il \glock{capitolato} C5 ha come obbiettivo la realizzazione di un applicativo \glock{Real-Time} in grado di guidare delle unità dotate di mobilità autonoma in ambienti specifici, partendo dal presupposto che queste si muovano in ambienti in cui sono presenti altre unità (autonome o meno).

\subsection{Glossario e documenti esterni}
	In supporto alla documentazione viene fornito un glossario per chiarire, con una definizione, eventuali termini specifici contenuti in questo documento.
	Saranno adottati quindi questi due simboli a pedice:
	\begin{itemize}
		\item \textit{D} se indicano un documento specifico;
		\item \textit{G} se incluse nel \dext{glossario}.
	\end{itemize}

\subsection{Riferimenti}
	\subsubsection{Riferimenti Normativi}
	\begin{itemize}
		\item \textbf{Norme di Progetto}: \dext{Norme di Progetto v1.0.0};
		\item \textbf{{\glock{Capitolato} C5 - PORTACS}}: \url{https://www.math.unipd.it/~tullio/IS-1/2020/Progetto/C5.pdf};
	\end{itemize}
	\subsubsection{Rifermenti informativi}
	\begin{itemize}
		\item \textbf{ISO/IEC 12207:1995}: \url{https://www.math.unipd.it/~tullio/IS-1/2009/Approfondimenti/ISO_12207-1995.pdf}
		\item \textbf{Analisi dei Requisiti}: \dext{Piano di Qualifica v1.0.0};
		\item \textbf{Piano di Qualifica}: \dext{Piano di Qualifica v1.0.0};
		\item \textbf{Studio di Fattibilità}: \dext{Studio di Fattibilità v1.0.0}; 
	\end{itemize}

\subsection{Scadenze}
	Il gruppo lineCode, a seguito di una prima pianificazione, ha deciso di rispettare le seguenti scadenze per lo svolgimento del progetto:
	\begin{table} [h!]
		\begin{center}
			\rowcolors{2}{gray!25}{gray!6}
			\begin{tabular} { m{5.5cm} c }
				\rowcolor{lightgray}
				\textbf{Nome Scadenza} & \textbf{Data di Scadenza} \\ 
				Revisione dei Requisiti [RR] & 18-01-2021 \\ 
				Revisione di Progettazione [RP] & 08-03-2021 \\ 
				Revisione di Qualifica [RQ] & 09-04-2021 \\ 
				Revisione di Accettazione [RA] & 10-05-2021 \\
			\end{tabular}
			\caption{Scadenze pianificate}
		\end{center}
	\end{table}