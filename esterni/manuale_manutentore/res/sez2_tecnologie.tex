\subsection{Tecnologie utilizzate}

	\subsubsection{Angular}
	\glock{Angular} è un framework open source per lo sviluppo di applicazioni web. Implementa l'interfaccia grafica del programma.
	
	\subsubsection{Docker}
	\glock{Docker} è un progetto open-source che automatizza la distribuzione di applicazioni all’interno di contenitori software, fornendo un’astrazione aggiuntiva grazie alla virtualizzazione a livello di sistema operativo di Linux. Le componenti del software sono contenuti in vari container di questo tipo.
	
	\subsubsection{GitHub}
	\glock{GitHub} è un servizio web di hosting per lo sviluppo di progetti software che usa il sistema di controllo di versione Git. Tutto il progetto sfrutta questo servizio ed i suoi strumenti.
	
	\subsubsection{Java}
	\glock{Java} è un linguaggio di programmazione ad alto livello, orientato agli oggetti. È stato scelto per l'enorme quantità di librerie presenti. Tutta la parte del server è sviluppata in Java.
	
	\subsubsection{Javascript}
	\glock{Javascript} è un linguaggio di scripting orientato agli oggetti e agli eventi. È stato utilizzato per implementare l'unità mobile.
	
	\subsubsection{JSON}
	\glock{JSON} è un formato di file adatto all'interscambio di dati tra applicazioni client/server ed è stato utilizzato nelle comunicazioni da e per il server.
	
	\subsubsection{LaTeX}
	\glock{LaTeX} è un linguaggio di markup per la preparazione di testi basato sul programma di composizione tipografica TEX. Quasi tutti i documenti sono stati prodotti utilizzando questo linguaggio.
	
	\subsubsection{Redis}
	\glock{Redis} è un datastore in memoria rapido, open source e di tipo chiave-valore utilizzato per la persistenza dei dati del server.

	\subsubsection{Typescript}
	\glock{Typescript} è un linguaggio di programmazione open-source sviluppato da Microsoft che estende il linguaggio \glock{Javascript} aggiungendo alcuni costrutti sintattici. È presente nel codice delle unità e dell'interfaccia.
