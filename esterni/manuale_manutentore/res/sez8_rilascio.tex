La repository \glock{Git} PORTACS contiene le repo delle tre componenti fondamentali (Server, UI e unità) come submodule e due script Bash, chiamati \textit{prepare} e \textit{reset}, per la preparazione delle immagini \glock{Docker}.

\subsection{Preparazione}
Lo script \textit{prepare}, una volta lanciato, crea gli artefatti nei submodule e sulla base di questi inizializza le relative immagini \glock{Docker}.
\\A questo punto il sistema di esecuzione dello script contiene le immagini \glock{Docker} delle tre componenti che possono essere rilasciate su una repository Docker remota e successivamente caricate in produzione.

\subsection{Esecuzione}
L'immagine del server espone la porta TCP:8080 per la comunicazione tramite \glock{WebSocket} con le unità e le UI, mentre l'immagine della UI espone la porta TCP:8081 per la comunicazione \glock{HTTP} con il browser web al fine di visualizzare l'interfaccia grafica.

\subsection{Reset}
Lo script \textit{reset} cancella gli artefatti nei submodule e le immagini \glock{Docker} create.