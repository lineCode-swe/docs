\subsection{Scopo del documento}
	Il documento vuole essere una guida a chiunque volesse estendere le funzionalità del progetto PORTACS. Nelle varie sezioni vengono spiegate nel dettaglio l'architettura e le tecnologie utilizzate.

\subsection{Scopo del prodotto}
	Il \glock{capitolato} C5 ha come obbiettivo la realizzazione di un applicativo \glock{Real-Time} in grado di guidare delle unità dotate di mobilità autonoma in ambienti specifici, partendo dal presupposto che queste si muovano in ambienti in cui sono presenti altre unità (autonome o meno).

\subsection{Glossario e documenti esterni}
	In supporto alla documentazione viene fornito un glossario per chiarire, con una definizione, eventuali termini specifici contenuti in questo documento.
	Saranno adottati quindi questi due simboli a pedice:
	\begin{itemize}
		\item \textit{D}: indica un documento specifico;
		\item \textit{G}: indica un termine incluso nel \dext{Glossario v3.0.0}.
	\end{itemize}

\subsection{Riferimenti}
	\subsubsection{Riferimenti normativi}
	\begin{itemize}
		\item \textbf{Norme di Progetto}: \dext{Norme di Progetto v2.0.0};
		\item \textbf{{\glock{Capitolato} C5 - PORTACS}}: \url{https://www.math.unipd.it/~tullio/IS-1/2020/Progetto/C5.pdf};
	\end{itemize}
	\subsubsection{Rifermenti informativi}
	\begin{itemize}
		\item \textbf{ISO/IEC 12207:1995}: \url{https://www.math.unipd.it/~tullio/IS-1/2009/Approfondimenti/ISO_12207-1995.pdf}
		\item \textbf{Analisi dei Requisiti}: \dext{Analisi dei Requisiti v2.0.0};
		\item \textbf{Piano di Qualifica}: \dext{Piano di Qualifica v2.0.0};
		\item \textbf{Studio di Fattibilità}: \dext{Studio di Fattibilità v2.0.0}.
	\end{itemize}
