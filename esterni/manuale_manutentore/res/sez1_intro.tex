\subsection{Scopo del documento}
	Il documento vuole essere una guida a chiunque volesse estendere le funzionalità del progetto \textit{PORTACS}. Nelle varie sezioni vengono spiegate nel dettaglio l'architettura e le tecnologie utilizzate.

\subsection{Scopo del prodotto}
	Il \glock{capitolato} C5 ha come obbiettivo la realizzazione di un applicativo \glock{Real-Time} in grado di guidare delle unità dotate di mobilità autonoma in ambienti specifici, partendo dal presupposto che queste si muovano in ambienti in cui sono presenti altre unità (autonome o meno).

\subsection{Glossario e documenti esterni}
	In supporto alla documentazione viene fornito un glossario per chiarire, con una definizione, eventuali termini specifici contenuti in questo documento.
	Saranno adottati quindi questi due simboli a pedice:
	\begin{itemize}
		\item \textit{D}: indica un documento specifico;
		\item \textit{G}: indica un termine incluso nel Glossario.
	\end{itemize}

\subsection{Riferimenti}
	\subsubsection{Riferimenti normativi}
	\begin{itemize}
		\item \textbf{{\glock{Capitolato} C5 - PORTACS}}: \url{https://www.math.unipd.it/~tullio/IS-1/2020/Progetto/C5.pdf};
	\end{itemize}
	\subsubsection{Riferimenti informativi}
	\begin{itemize}
		\item \textbf{Docker}: \url{https://www.math.unipd.it/~tullio/IS-1/2020/Progetto/ST3.pdf};
		\item \textbf{MVVM}: \url{https://www.math.unipd.it/~rcardin/sweb/2020/L02.pdf};
		\item \textbf{Layered architecture}: \url{https://www.math.unipd.it/~rcardin/sweb/2021/L03.pdf};
		\item \textbf{Hexagonal architecture}: \url{https://github.com/rcardin/hexagonal};
		\item \textbf{Docker - requisiti minimi Windows}: \url{https://docs.docker.com/docker-for-windows/install/}
		\item \textbf{Docker - requisiti minimi macOS}: \url{https://docs.docker.com/docker-for-mac/install/}
	\end{itemize}
