	\paragraph*{Angular}
	Framework open source per lo sviluppo di applicazioni web con licenza MIT, evoluzione di AngularJS.
	
	\paragraph*{AsciiDoc}
	Formato per documenti di testo che usa convenzioni di testo semplice come marcatori.
	
	\paragraph*{BIOS}
	Il Basic Input-Output System è il primo programma che viene eseguito dopo l'accensione, coinvolto pertanto nella fase di avvio (boot) del sistema di elaborazione.
	
	\paragraph*{Chai}
	Libreria di testing per Node.js e browser.
	
	\paragraph*{Checkstyle}
	Strumento di sviluppo per automatizzare il processo di controllo del codice Java. Aiuta a scrivere codice conforme a delle code conventions ben definite.
	
	\paragraph*{Continuous Integration}
	Pratica che si applica in contesti in cui lo sviluppo del software avviene attraverso un sistema di controllo versione. Consiste nell'allineamento frequente dagli ambienti di lavoro degli sviluppatori verso l'ambiente condiviso.
	
	\paragraph*{CSS}
	Cascaded Style Sheet, è un linguaggio che permette di definire tutte le proprietà di stile e formattazione di una pagina web in maniera modulare, tenendo questa parte della progettazione di un sito separata da quella relativa al contenuto.
	
	\paragraph*{Docker}
	Progetto open-source che automatizza la distribuzione di applicazioni all'interno di contenitori software, fornendo un'astrazione aggiuntiva grazie alla virtualizzazione a livello di sistema operativo di Linux.
	
	\paragraph*{ESLint}
	Strumento di analisi del codice statico per identificare i modelli problematici trovati nel codice JavaScript.
	
	\paragraph*{Framework}
	Architettura logica di supporto (spesso un'implementazione logica di un particolare design pattern) su cui un software può essere progettato e realizzato, spesso facilitandone lo sviluppo da parte del programmatore.
	
	\paragraph*{Git}
	Software di controllo versione distribuito utilizzabile da interfaccia a riga di comando.
	
	\paragraph*{GitHub}
	Servizio web di hosting per lo sviluppo di progetti software che usa il sistema di controllo di versione Git.
	
	\paragraph*{HTML}
	Linguaggio di markup per la strutturazione delle pagine web.
	
	\paragraph*{HTTP}
	Hypertext Transfer Protocol, è un protocollo a livello applicativo usato come principale sistema per la trasmissione d'informazioni sul web ovvero in un'architettura tipica client-server.
	
	\paragraph*{IDE}
	Integrated Development Environment, è un ambiente di sviluppo ovvero un software che, in fase di programmazione, supporta i programmatori nello sviluppo e debugging del codice sorgente di un programma.
	
	\paragraph*{IntelliJ IDEA}
	Ambiente di sviluppo integrato per il linguaggio di programmazione Java.
	
	\paragraph*{Jackson}
	Libreria ad alte prestazioni per Java che processa istanze trasformandole in JSON e viceversa.
	
	\paragraph*{Jasmine}
	Framework di test open source per JavaScript.
	
	\paragraph*{Java}
	Linguaggio di programmazione ad alto livello, orientato agli oggetti e a tipizzazione statica, che si appoggia sull'omonima piattaforma software di esecuzione, specificamente progettato per essere il più possibile indipendente dalla piattaforma hardware di esecuzione tramite l'utilizzo di macchina virtuale.
	
	\paragraph*{Javascript}
	Linguaggio di scripting orientato agli oggetti e agli eventi, utilizzato nella programmazione Web sia lato client che server.
	
	\paragraph*{JSON}
	Acronimo di JavaScript Object Notation, è un formato di file adatto all'interscambio di dati fra applicazioni client/server.
	
	\paragraph*{JUnit}
	Framework di unit testing per il linguaggio di programmazione Java.
	
	\paragraph*{LaTeX}
	Linguaggio di markup per la preparazione di testi, basato sul programma di composizione tipografica TEX.
	
	\paragraph*{MariaDB}
	MariaDB è un DBMS nato da un fork di MySQL creato dal programmatore originale di tale programma.
	
	\paragraph*{Maven}
	Strumento completo per la gestione di progetti software Java, in termini di compilazione del codice, distribuzione, documentazione e collaborazione del team di sviluppo.
	
	\paragraph*{Mocha}
	Framework di test JavaScript per i programmi Node.js, con supporto per browser, test asincroni, rapporti sulla copertura dei test e utilizzo di qualsiasi libreria di asserzioni.
	
	\paragraph*{MongoDB}
	DBMS non relazionale, orientato ai documenti.
	
	\paragraph*{Node.js}
	Runtime di JavaScript Open source multipiattaforma orientato agli eventi per l'esecuzione di codice JavaScript.
	
	\paragraph*{NPM}
	Node Package Manager, è un gestore di pacchetti per il linguaggio di programmazione JavaScript. È il gestore di pacchetti predefinito per l'ambiente di runtime JavaScript Node.js. Consiste in un client da linea di comando, chiamato anch'esso npm e un database online di pacchetti pubblici e privati, chiamato npm registry.
	
	\paragraph*{POI}
	Point Of Interest, è un punto specifico che qualcuno potrebbe trovare utile o interessante.
	
	\paragraph*{Redis}
	Database chiave-valore open-source residente in memoria con persistenza facoltativa.
	
	\paragraph*{SonarCloud}
	Servizio Cloud per la misurazione della qualità di un prodotto software tramite analisi statica del codice.

	\paragraph*{Typescript}
	Linguaggio di programmazione open-source sviluppato da Microsoft che estende il linguaggio Javascript aggiungendo alcuni costrutti sintattici.
	
	\paragraph*{WebSocket}
	È una tecnologia web che fornisce canali di comunicazione full-duplex attraverso una singola connessione TCP.