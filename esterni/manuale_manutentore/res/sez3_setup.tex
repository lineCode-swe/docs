Il progetto PORTACS prevede l'installazione del software \glock{DOCKER} per la sua esecuzione e/o manutenzione ed è quindi necessario che il sistema di esecuzione ne soddisfi i requisiti minimi:
\begin{itemize}
	\item processore a 64 bit con Second Level Address Translation (SLAT);
	\item RAM di sistema da 4 GB;
	\item il supporto per la virtualizzazione hardware deve essere abilitato nelle impostazioni del \glock{BIOS} (solo per Windows e macOS).
\end{itemize}
Per effettuare la manutenzione del software sono necessari i Package Manager \glock{Maven}, \glock{NPM} e qualsiasi \glock{IDE} in grado di supportarli.
Il software è stato testato in tutti e tre gli ambienti più utilizzati:
\begin{itemize}
	\item Linux Ubuntu 18.04 LTS o successiva; 
	\item macOS 10.14 o successiva;
	\item Windows 10 a 64bit build 19041 o successiva.
\end{itemize}  
