
I requisiti strutturati nel seguente modo:

\begin{itemize}

	\item \textbf{codice identificativo:} i codici identificativi sono univoci e conformi alla seguente codifica:
	\begin{center}
		\textbf{R[Priorità]-[Categoria]-[Codice]}
	\end{center}
	dove:
		\begin{itemize}
			\item \textbf{R:} requisito;
			\item \textbf{Priorità:}
			\begin{itemize}
				\item \textbf{M:} mandatory/obbligatorio, quindi necessario a garantire le funzioni base del prodotto;
				\item \textbf{D:} desirable/desiderabile, cioè non strettamente necessario, ma che porta alla completezza del prodotto;
				\item \textbf{O:} optional/opzionale, quindi che non pregiudica la funzionalità del prodotto finale.
			\end{itemize}
			\item \textbf{Categoria:}
			\begin{itemize}
				\item \textbf{F:} functional/funzionale;
				\item \textbf{P:} performance/prestazionale;
				\item \textbf{Q:} qualitative/qualitativo;
				\item \textbf{C:} constraint/vincolo.
			\end{itemize}
			\item \textbf{Codice:} numero progressivo per riconoscere univocamente il requisito. \\
		\end{itemize}
	\noindent In tabella:
	\item \textbf{Requisito:} il codice identificativo del  requisito;
	\item \textbf{Priorità:} priorità del requisito. Informazione ridondante che però semplifica la lettura;

	\item \textbf{Descrizione:} breve ma completa descrizione del requisito, il meno ambigua possibile;

	\item \textbf{Fonti:} ogni requisito è derivato da una o più delle seguenti opzioni:
		\begin{itemize}
			\item \glock{capitolato}: requisito individuato in seguito all'analisi del \glock{capitolato};
			\item \textit{interno:} requisito individuato dagli analisti e che si è ritenuto opportuno da aggiungere;
			\item \textit{caso d'uso:} requisito derivato da uno o più casi d'uso. Viene riportato il codice univoco del caso d'uso;
			\item \textit{verbale:} requisiti individuati durante incontri con il proponente.
		\end{itemize}
\end{itemize}

\vspace{0.5cm}

\subsection{Requisiti funzionali}

	\newcommand*{\thead}[1]{\multicolumn{1}{c}{\bfseries #1}}
	\rowcolors{2}{gray!6}{gray!25}
	\setlength{\tabcolsep}{10pt}
	\begin{longtable}[h!] { c c m{8cm} c}
		\caption{Tabella dei requisiti funzionali} \\
		\rowcolor{lightgray}
		\thead{Requisito} & \thead{Priorità} & \thead{Descrizione} & \thead{Fonti} \\ \endhead%

		RMF1 & Obbligatorio & Deve essere disponibile per l'utente generico una mappatura dell'ambiente in cui le unità operano & Interno \& UC1 \\

		RMF1.1 & Obbligatorio & Deve essere disponibile per l'utente generico una legenda esplicativa della simbologia utilizzata all'interno della mappatura & Interno \& UC1 \\

		RMF1.2 & Obbligatorio & La mappatura deve essere aggiornata periodicamente, garantendo che i dati siano coerenti con l'ambiente descritto & Interno \& UC1 \\

		RMF1.3 & Obbligatorio & La mappatura deve fornire informazioni sul movimento di unità facenti parte del sistema & Interno \& UC1 \\

		RDF1.4 & Desiderabile & La mappatura deve fornire informazioni sul movimento di pedoni presenti nell'ambiente & Interno \& UC1 \\

		RMF1.5 & Obbligatorio & La sezione dedicata alla mappatura deve essere sempre disponibile all'utente & Interno \& UC1 \\

		RMF2 & Obbligatorio & Un utente può effettuare il login & VE\_2020-12-29\_2.5 \& UC2 \\

		RMF2.1 & Obbligatorio & Per effettuare login l'utente deve inserire il suo nome utente & VE\_2020-12-29\_2.5 \& UC2.1 \\

		RMF2.2 & Obbligatorio & Per effettuare login l'utente deve inserire la sua password & VE\_2020-12-29\_2.5 \& UC2.2 \\

		RMF3 & Obbligatorio & Viene visualizzato un errore se vengono fornite credenziali errate & VE\_2020-12-29\_2.5 \& UC3 \\

		RMF4 & Obbligatorio & Un utente autenticato può effettuare logout & VE\_2020-12-29\_2.5 \& UC4 \\

		RMF5 & Obbligatorio & Deve essere disponibile una guida utente che contiene le istruzioni operative di utilizzo per un utente autenticato & Interno \& UC5 \\

		RMF5.1 & Obbligatorio & La guida utente deve contenere tutte le istruzioni relative alle operazioni elementari che un utente può svolgere all'interno dell'applicativo & Interno \& UC5 \\

		RMF5.2 & Obbligatorio & La guida utente deve essere accessibile a tutti e soli gli utenti registrati & Interno \& UC5 \\

		RMF6 & Obbligatorio & L'utente autenticato può visualizzare la lista di unità attive & Capitolato \& UC6 \\

		RMF6.1 & Obbligatorio & Ogni elemento della lista di unità mette a disposizione un ambiente grafico dedicato per visualizzazione delle proprietà dell'unità & Interno \& UC6 \\

		ROF6.1.1 & Opzionale & Dentro all'ambiente grafico dedicato, l'utente autenticato può visualizzare il l'identificativo dell'unità & Interno \& UC6.1 \\

		RMF6.1.2 & Obbligatorio & Dentro all'ambiente grafico dedicato, l'utente autenticato può visualizzare le coordinate della posizione attuale all'interno della mappa & Interno \& UC6.1 \\

		RMF6.1.3 & Obbligatorio &  Dentro all'ambiente grafico dedicato, l'utente autenticato può visualizzare lo stato dell'unità & Interno \& UC6.1 \\

		RMF6.1.4 & Obbligatorio &  Dentro all'ambiente grafico dedicato, l'utente autenticato può visualizzare la velocità attuale dell'unità & Interno \& UC6.1 \\

		RMF6.1.5 & Obbligatorio &  Dentro all'ambiente grafico dedicato, l'utente autenticato può visualizzare la direzione del prossimo passo suggerita dal sistema & Interno \& UC6.1 \\

		RMF6.1.6 & Obbligatorio &  Dentro all'ambiente grafico dedicato, l'utente autenticato può visualizzare la coda di ordini assegnati all'unità & Interno \& UC6.2 \\

		RMF7 & Obbligatorio & L'utente autenticato può gestire le unità attive & Capitolato \& UC7 \\

		RMF7.1 & Obbligatorio & Per ogni unità attiva il sistema mette a disposizione un ambiente grafico dedicato per la gestione di essa & Interno \& UC7 \\

		RMF7.1.1 & Obbligatorio & L'utente autenticato può impartire il comando \underline{Start} all'unità & Capitolato \& UC7.1 \\

		RMF7.1.2 & Obbligatorio & L'utente autenticato può impartire il comando \underline{Stop} all'unità & Capitolato \& UC7.2 \\

		RMF7.1.3 & Obbligatorio & L'utente autenticato può impartire il comando \underline{Go Back} all'unità & Interno \& UC7.3 \\

		RMF7.1.4 & Obbligatorio & L'utente autenticato può impartire il comando \underline{Shutdown} all'unità & Interno \& UC7.4 \\

		RMF7.1.5 & Obbligatorio & L'utente autenticato può accodare un nuovo ordine alla lista degli ordini di un'unità & Capitolato \& UC7.5 \\

		RMF7.1.6 & Obbligatorio & L'utente autenticato può eliminare un ordine dalla lista degli ordini di un'unità qualsiasi sia la sua posizione & Capitolato \& UC7.6 \\

		RMF8 & Obbligatorio & L'amministratore deve poter visualizzare la lista degli utenti esistenti & VE02.4 \& UC8 \\

		RMF8.1 & Obbligatorio & L'amministratore deve poter visualizzare il nome utente di ogni utente & Interno \& UC8 \\

		RMF8.2 & Obbligatorio & L'amministratore deve poter visualizzare la password di ogni utente & Interno \& UC8 \\

		RMF9 & Obbligatorio & L'amministratore deve poter creare nuovi utenti & VE\_2020-12-29\_2.4 \& UC9 \\

		RMF9.1 & Obbligatorio & L'input delle credenziali utente richiede un nome utente & VE\_2020-12-29\_2.4 \& UC9.1 \\

		RMF9.2 & Obbligatorio & L'input delle credenziali utente richiede una password & VE\_2020-12-29\_2.4 \& UC9.2 \\

		RMF9.3 & Obbligatorio & L'input delle credenziali utente richiede lo status & VE\_2020-12-29\_2.4 \& UC9.3 \\

		RMF10 & Obbligatorio & L'amministratore deve poter eliminare utenti esistenti & VE\_2020-12-29\_2.4 \& UC10 \\

		RMF11 & Obbligatorio & Viene visualizzato un errore se sono state fornite credenziali già assegnate ad un utente esistente & VE\_2020-12-29\_2.4 \& UC11 \\

		RMF12 & Obbligatorio & L'amministratore deve poter visualizzare la mappa formattata oppurtanamente & Capitolato \& UC12 \\

		RMF13 & Obbligatorio & L'amministratore deve poter modificare la mappa formattandola opportunamente & Capitolato \& UC13 \\

		RMF14 & Obbligatorio & Viene visualizzato un errore se si tenta di formattare la mappa erratamente & Capitolato \& UC14 \\

		RMF15 & Obbligatorio & L'amministratore deve poter visualizzare la lista unità gestite dal sistema & VE\_2020-12-29\_2.4 \& UC15 \\

		RMF15.1 & Obbligatorio & L'amministratore deve poter visualizzare l'ID di ogni unità & Interno \& UC15 \\

		RMF16 & Obbligatorio & L'amministratore deve poter inserire nuove unità tramite l'input del rispettivo ID & VE\_2020-12-29\_2.4 \& UC16 \\

		RMF17& Obbligatorio & L'amministratore deve poter eliminare unità esistenti & VE\_2020-12-29\_2.4 \& UC17 \\

		RMF18 & Obbligatorio & Viene visualizzato un errore se viene fornito un ID già assegnato ad un'unità esistente & VE\_2020-12-29\_2.4 \& UC18 \\

		RMF19 & Obbligatorio & L'unità, dotata di sensoristica, deve comunicare al sistema la posizione relativa degli ostacoli che riescono a rilevare & VE\_2020-12-29\_2.6 \\

		RMF20 & Obbligatorio & All'unità viene assegnato dal sistema un percorso definito per poter raggiungere il prossimo \glock{POI} & Capitolato \\

		RMF21 & Obbligatorio & L'unità dispone di una coda degli ordini con i prossimi \glock{POI} da raggiungere & Capitolato \\

		RMF22 & Obbligatorio & L'unità riceve modifiche alla coda degli ordini solo quando si trova alla base di ricarica & Interno \& UC7.5 \& UC7.6 \\

		RMF23 & Obbligatorio & L'unità con coda degli ordini vuota ritorna alla base & Interno \& UC7.3 \\

		RMF24 & Obbligatorio & L'unità dispone di una velocità massima & Capitolato \\

		RMF25 & Obbligatorio & L'unità può assumere lo stato \underline{Going to X} & Interno \& UC7.1 \\

		RMF26 & Obbligatorio & L'unità può assumere lo stato \underline{Stop} & Interno \& UC7.2 \\

		RMF27 & Obbligatorio & L'unità può assumere lo stato \underline{Base} & Interno \& UC7.3 \\

		RMF28 & Obbligatorio & L'unità può assumere lo stato \underline{Error Y} & Interno \& UC7.4 \\

		RMF29 & Obbligatorio & Il sistema dispone di un algoritmo per il calcolo del percorso che l'unità deve effettuare per raggiungere il prossimo \glock{POI} & Capitolato \\

		RMF29.1 & Obbligatorio & L'algoritmo di calcolo del percorso deve evitare la collisione delle unità con ostacoli & Capitolato \\

		RMF29.2 & Obbligatorio & L'algoritmo di calcolo del percorso deve evitare la collisione delle unità con altre unità & Capitolato \\

		RMF29.3 & Opzionale & L'algoritmo di calcolo del percorso deve restituire il percorso ottimo & Capitolato \\

		RDF30 & Desiderabile & Nella mappa sono presenti dei pedoni & Capitolato \\

		RDF30.1 & Desiderabile & Il pedone comunica in \glock{real-time} al sistema la propria posizione & VE02.8 \\

		RDF30 .2 & Desiderabile & Il pedone rappresenta un ostacolo mobile per l'unità & Capitolato \\

	\end{longtable}

\newpage

\subsection{Requisiti di vincolo}

\rowcolors{2}{gray!6}{gray!25}
\setlength{\tabcolsep}{10pt}
\begin{longtable}[h!] { c c m{8.5cm} c}
	\caption{Tabella dei requisiti di vincolo} \\
	\rowcolor{lightgray}
	\thead{Requisito} & \thead{Priorità} & \thead{Descrizione} & \thead{Fonti} \\ \endhead%

	RMC1 & Obbligatorio & Ogni entità sviluppata facente parte del sistema dovrà essere contenuta in un container \glock{Docker} del quale andrà fornito il \glock{Dockerfile} & Capitolato \\

	RMC1.1 & Obbligatorio & Deve essere fornito un \glock{Dockerfile} con il motore di calcolo & Capitolato \\

	RMC1.2 & Obbligatorio & Deve essere fornito un \glock{Dockerfile} con il visualizzatore \glock{real-time} & Capitolato \\

	RMC1.3 & Obbligatorio & Deve essere fornito un \glock{Dockerfile} per la singola unità & Capitolato \\

	RDC1.4 & Desiderabile & Deve essere fornito un \glock{Dockerfile} per il singolo pedone & Capitolato \\

\end{longtable}

\vspace{1cm}

\subsection{Requisiti di qualità}

\rowcolors{2}{gray!6}{gray!25}
\setlength{\tabcolsep}{10pt}
\begin{longtable}[h!] { c c m{8.5cm} c}
	\caption{Tabella dei requisiti di qualità} \\
	\rowcolor{lightgray}
	\thead{Requisito} & \thead{Priorità} & \thead{Descrizione} & \thead{Fonti} \\ \endhead%

	RMQ1 & Obbligatorio & Il prodotto va rilasciato con la licenza \glock{open-source} più aperta possibile in base alle librerie utilizzate & VE\_2020-12-14\_1.4 \\

	RMQ2 & Obbligatorio & Il prodotto deve essere conforme con quanto dichiarato nel documento \dext{Piano di Qualifica v3.0.0} e successivi aggiornamenti del medesimo & Interno \\

	RMQ3 & Obbligatorio & Devono essere realizzati test di unità e di integrazione per verificare le singole componenti del prodotto & Interno \\

	RMQ4 & Obbligatorio & Al rilascio del prodotto dovrà essere disponibile il manuale utente & Interno \\

	RMQ5 & Obbligatorio & Al rilascio del prodotto dovrà essere disponibile il manuale manutentore & Interno \\

\end{longtable}

\newpage

\subsection{Tracciamento}

\subsubsection{Fonte - Requisiti}

\rowcolors{2}{gray!6}{gray!25}
\setlength{\tabcolsep}{10pt}
\begin{longtable}[h!] { >{\centering}m{5cm} >{\centering}m{5cm} }
	\caption{Tabella di tracciamento fonte-requisiti} \\
	\rowcolor{lightgray}
	\thead{Fonte} & \thead{Requisiti} \\ \endhead%


	 Interno & RMF1 \\
	 RMF1.1 \\
	 RMF1.2 \\
	 RMF1.3 \\
	 RDF1.4 \\
	 RMF1.5 \\
	 RMF5 \\
	 RMF5.1 \\
	 RMF5.2 \\
	 RMF6.1 \\
	 RMF6.2 \\
	 ROF6.1.1 \\
	 RMF6.1.2 \\
	 RMF6.1.3 \\
	 RMF6.1.4 \\
	 RMF6.1.5 \\
	 RMF6.1.6 \\
	 RMF7.1 \\
	 RMF7.1.3 \\
	 RMF7.1.4 \\
	 RMF8.1 \\
	 RMF8.2 \\
	 RMF15.1 \\
	 RMF22 \\
	 RMF23 \\
	 RMF25 \\
	 RMF26 \\
	 RMF27 \\
	 RMF28 \\
	 RMQ2 \\
	 RMQ3 \\
	 RMQ4 \\
	 RMQ5
	 \tabularnewline
	 Capitolato & RMF5 \\
	 RMF6 \\
	 RMF7
	 RMF7.1.1\\
	 RMF7.1.2 \\
	 RMF7.1.5 \\
	 RMF7.1.6 \\
	 RMF12 \\
	 RMF13 \\
	 RMF14 \\
	 RMF19 \\
	 RMF20 \\
	 RMF21 \\
	 RMF24 \\
	 RMF29 \\
	 RMF29.1 \\
	 RMF29.2 \\
	 RMF29.3 \\
	 RDF30\\
	 RDF30.2 \\
	 RMC1 \\
	 RMC1.1 \\
	 RMC1.2 \\
	 RMC1.3 \\
	 RDC1.4
	 \tabularnewline
	 UC1 & RMF1 \\
	 RMF1.1 \\
	 RMF1.2 \\
	 RMF1.3 \\
	 RDF1.4 \\
	 RMF1.5
	 \tabularnewline
	 UC2 & RMF2
	 \tabularnewline
	  UC2.1 & RMF2.1
	 \tabularnewline
	  UC2.2 & RMF2.2
	 \tabularnewline
	  UC3 & RMF3
	 \tabularnewline
	  UC4 & RMF4
	 \tabularnewline
	 UC5 & RMF5 \\
	 RMF5.1 \\
	 RMF5.2
	 \tabularnewline
	 UC6 & RMF6 \\
	 RMF6.1
	 \tabularnewline
	 UC6.1 & ROF6.1.1 \\
	 RMF6.1.2 \\
	 RMF6.1.3 \\
	 RMF6.1.4 \\
	 ROF6.1.5
	 \tabularnewline
	 UC6.2 & RMF6.1.6
	 \tabularnewline
	 UC7 & RMF7 \\
	 RMF7.1
	 \tabularnewline
	 UC7.1 & RMF7.1.1 \\
	 RMF25
	 \tabularnewline
	 UC7.2 & RMF7.1.2 \\
	 RMF26
	 \tabularnewline
	 UC7.3 & RMF7.1.3 \\
	 RMF23 \\
	 RMF27
	 \tabularnewline
	 UC7.4 & RMF7.1.4 \\
	 RMF28
	 \tabularnewline
	 UC7.5 & RMF7.1.5 \\
	 RMF22
	 \tabularnewline
	 UC7.6 & RMF7.1.6 \\
	 RMF23
	 \tabularnewline
	 UC8 & RMF8 \\
	 RMF8.1 \\
	 RMF8.2
	 \tabularnewline
	 UC9 & RMF9 \\
	 \tabularnewline
	 UC9.1 & RMF9.1 \\
	 \tabularnewline
	 UC9.2 & RMF9.2
	 \tabularnewline
	 UC9.3 & RMF9.3
	 \tabularnewline
	 UC10 & RMF10
	 \tabularnewline
	 UC11 & RMF11
	 \tabularnewline
	 UC12 & RMF12
	 \tabularnewline
	 UC13 & RMF13
	 \tabularnewline
	 UC14 & RMF14
	 \tabularnewline
	 UC15 & RMF15 \\
	 RMF15.1
	 \tabularnewline
	 UC16 & RMF16
	 \tabularnewline
	 UC17 & RMF17
	 \tabularnewline
	 UC18 & RMF18
	 \tabularnewline
	 VE\_2020-12-14\_1.2 & RMP1 \\
	 RDP1.1 \\
	 ROP1.2
	 \tabularnewline
	 VE\_2020-12-14\_1.4 & RMQ1
	 \tabularnewline
	 VE\_2020-12-29\_2.4 & RMF7\\
	 RMF8 \\
	 RMF9 \\
	 RMF10 \\
	 RMF11.1 \\
	 RMF11.2 \\
	 RMF11.3 \\
	 RMF11.4 \\
	 RMF11.5 \\
	 RMF15 \\
	 RMF16 \\
	 RMF17 
	 \tabularnewline
	 VE\_2020-12-29\_2.5 & RMF2 \\
	 RMF2.1 \\
	 RMF2.2 \\
	 RMF2.3 \\
	 RMF2.4 \\
	 RMF3
	 \tabularnewline
	 VE\_2020-12-29\_2.6 & RMF19
	 \tabularnewline
	 VE\_2020-12-29\_2.8 & RDF31.1
	 \tabularnewline

\end{longtable}

\newpage

\subsubsection{Requisito - Fonti}

\rowcolors{2}{gray!6}{gray!25}
\setlength{\tabcolsep}{10pt}
\begin{longtable}[h!] { >{\centering}m{5cm} >{\centering}m{5cm} }
	\caption{Tabella di tracciamento requisito-fonti} \\
	\rowcolor{lightgray}
	\thead{Requisito} & \thead{Fonti} \\ \endhead%

	RMF1    &  Interno\\UC1
	\tabularnewline
	RMF1.1  & Interno \\UC1
	\tabularnewline
	RMF1.2  & Interno\\UC1
	\tabularnewline
	RMF1.3   & Interno\\UC1
	\tabularnewline
	RDF1.4 & Interno\\UC1
	\tabularnewline
	RMF1.5 & Intern\\UC1
	\tabularnewline
	RMF2 & VE\_2020-12-29\_2.5\\UC2
	\tabularnewline
	RMF2.1 & VE\_2020-12-29\_2.5\\UC2.1
	\tabularnewline
	RMF2.2 & VE\_2020-12-29\_2.5\\UC2.2
	\tabularnewline
	RMF3 & VE\_2020-12-29\_2.5\\UC3
	\tabularnewline
	RMF4 & VE\_2020-12-29\_2.5\\UC4
	\tabularnewline
	RMF5 & Interno\\UC5
	\tabularnewline
	RMF5.1 & Interno\\UC5
	\tabularnewline
	RMF5.2 & Interno\\UC5
	\tabularnewline
	RMF6 & Capitolato\\UC6
	\tabularnewline
	RMF6.1 & Interno\\UC6
	\tabularnewline
	ROF6.1.1 & Interno\\UC6.1
	\tabularnewline
	ROF6.1.2 &  Interno\\UC6.1
	\tabularnewline
	RMF6.1.3 & Interno\\UC6.1
	\tabularnewline
	RMF6.1.4 & Interno\\UC6.1
	\tabularnewline
	RMF6.1.5 & Interno\\UC6.1
	\tabularnewline
	RMF6.1.6 & Interno\\UC6.2
	\tabularnewline
	RMF7 &  Capitolato\\UC7
	\tabularnewline
	RMF7.1 & Interno\\UC7
	\tabularnewline
	RMF7.1.1 & Capitolato\\UC7.1
	\tabularnewline
	RMF7.1.2 & Capitolato\\U7.2
	\tabularnewline
	RMF7.1.3 & Interno\\UC7.3
	\tabularnewline
	RMF7.1.4 & Interno\\UC7.4
	\tabularnewline
	RMF7.1.5 & Capitolato\\UC7.5
	\tabularnewline
	RMF7.1.6 & Capitolato\\UC7.6
	\tabularnewline
	RMF8 & VE02.4\\UC8
	\tabularnewline
	RMF8.1 & Interno\\UC8
	\tabularnewline
	RMF8.2 & Interno\\UC8
	\tabularnewline
	RMF9 & VE\_2020-12-29\_2.4\\UC9
	\tabularnewline
	RMF9.1 & VE\_2020-12-29\_2.4\\UC9.1
	\tabularnewline
	RMF9.2 & VE\_2020-12-29\_2.4\\UC9.2
	\tabularnewline
	RMF9.3 & VE\_2020-12-29\_2.4\\UC9.3
	\tabularnewline
	RMF10 & VE\_2020-12-29\_2.4\\UC10
	\tabularnewline
	RMF11 & VE\_2020-12-29\_2.4\\UC11
	\tabularnewline
	RMF12 & Capitolato\\UC12
	\tabularnewline
	RMF13 & Capitolato\\UC13
	\tabularnewline
	RMF14 & Capitolato\\UC14
	\tabularnewline
	RMF15  & VE\_2020-12-29\_2.4\\UC15
	\tabularnewline
	RMF15.1 &Interno\\UC15
	\tabularnewline
	RMF16 & VE\_2020-12-29\_2.4\\UC16
	\tabularnewline
	RMF17 & VE\_2020-12-29\_2.4\\UC17
	\tabularnewline
	RMF18 & VE\_2020-12-29\_2.4\\UC18
	\tabularnewline
	RMF19 & VE\_2020-12-29\_2.6
	\tabularnewline
	RMF20 & Capitolato
	\tabularnewline
	RMF21 & Capitolato
	\tabularnewline
	RMF22 & Interno \\ UC6.5 \\ UC6.6
	\tabularnewline
	RMF23 & Interno\\ UC6.3
	\tabularnewline
	RMF24 & Capitolato
	\tabularnewline
	RMF25 & Interno\\UC6.1
	\tabularnewline
	RMF26 & Interno\\UC6.2
	\tabularnewline
	RMF27 & Interno\\UC6.3
	\tabularnewline
	RMF28 & Interno\\UC6.4
	\tabularnewline
	RMF29 & Capitolato
	\tabularnewline
	RMF29.1 & Capitolato
	\tabularnewline
	RMF29.2 & Capitolato
	\tabularnewline
	RMF29.3 & Capitolato
	\tabularnewline
	RDF30 & Capitolato
	\tabularnewline
	RDF30.1 & VE02.8
	\tabularnewline
	RDF31.1 & VE\_2020-12-29\_2.8
	\tabularnewline
	RDF31.2 & Capitolato
	\tabularnewline
	RMQ1 & VE\_2020-12-14\_1.4
	\tabularnewline
	RMQ2 & Interno
	\tabularnewline
	RMQ3 & Interno
	\tabularnewline
	RMQ4 & Interno
	\tabularnewline
	RMQ5 & Interno
	\tabularnewline
	RMC1 & Capitolato
	\tabularnewline
	RMC1.1 & Capitolato
	\tabularnewline
	RMC1.2 & Capitolato
	\tabularnewline
	RMC1.3 & Capitolato
	\tabularnewline
	RDC1.4 & Capitolato
	\tabularnewline

\end{longtable}

