
I requisiti strutturati nel seguente modo:

\begin{itemize}
	
	\item \textbf{codice identificativo:} i codici identificativi sono univoci e conformi alla seguente codifica:
	\begin{center}
		\textbf{R[Priorità]-[Categoria]-[Codice]}
	\end{center}
	dove: 
		\begin{itemize}
			\item \textbf{R:} requisito;
			\item \textbf{Priorità:}
			\begin{itemize}
				\item \textbf{M:} mandatory/obbligatorio, quindi necessario a garantire le funzioni base del prodotto;
				\item \textbf{D:} desirable/desiderabile, cioè non strettamente necessario, ma che porta alla completezza del prodotto;
				\item \textbf{O:} optional/opzionale, quindi che non pregiudica la funzionalità del prodotto finale.
			\end{itemize}
			\item \textbf{Categoria:}
			\begin{itemize}
				\item \textbf{F:} functional/funzionale;
				\item \textbf{P:} performance/prestazionale;
				\item \textbf{Q:} qualitative/qualitativo;
				\item \textbf{C:} constraint/vincolo.
			\end{itemize}
			\item \textbf{Codice:} numero progressivo per riconoscere univocamente il requisito.
		\end{itemize}
	
	\item \textbf{Priorità:} priorità del requisito. Informazione ridondante che però semplifica la lettura.
	
	\item \textbf{Descrizione:} breve ma completa descrizione del requisito, il meno ambigua possibile. 
	
	\item \textbf{Fonti:} ogni requisito è derivato da una o più delle seguenti opzioni:
		\begin{itemize}
			\item \glock{capitolato}: requisito individuato in seguito all'analisi del \glock{capitolato};
			\item \textit{interno:} requisito individuato dagli analisti e che si è ritenuto opportuno da aggiungere;
			\item \textit{caso d'uso:} requisito derivato da uno o più casi d'uso. Viene riportato il codice univoco del caso d'uso.
			\item \textit{verbale:} requisiti individuati durante incontri con il proponente.			
		\end{itemize}	
\end{itemize}

 \newpage
 
\subsection{Requisiti funzionali}

	\newcommand*{\thead}[1]{\multicolumn{1}{c}{\bfseries #1}}	
	\rowcolors{2}{gray!6}{gray!25}
	\setlength{\tabcolsep}{10pt}
	\begin{longtable}[h!] { c c m{8.5cm} c}
		\caption{Tabella dei requisiti funzionali} \\
		\rowcolor{lightgray}
		\thead{Requisito} & \thead{Priorità} & \thead{Descrizione} & \thead{Fonti} \\ \endhead%
		
		RMFX   & Obbligatorio & Blabla blablablabl ablablabla bl abla bla blablablablabla blabla blablabla & Interno UCX \\

	\end{longtable}


