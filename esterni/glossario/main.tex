\documentclass[]{article}

\usepackage[italian]{babel}
\usepackage[margin=20mm, footskip = 20pt]{geometry}
\usepackage{array}
\usepackage{tabularx}
\usepackage{graphicx}
\usepackage{subfiles}
\usepackage{hyperref}
\usepackage{nameref}
\usepackage{titlesec}
\usepackage{longtable}
\usepackage[table]{xcolor}
\usepackage{titling}
\usepackage{lastpage}
\usepackage{ifthen}
\usepackage{calc}
\usepackage{soulutf8}
\usepackage{contour}
\usepackage{float}
\usepackage{fancyhdr}
\usepackage{multirow}
\usepackage{pgfgantt}
\usepackage{lscape}

\newcommand{\hr}{\par\vspace{-.1\ht\strutbox}\noindent\hrulefill\par}

\graphicspath{ {./}
	{./commons/res}
}

%--------------------------------------------------
% Comandi per inserire contenuto del documento
%--------------------------------------------------
\makeatletter

\newcommand\appendToGraphicsPath[1]{%
	\g@addto@macro\Ginput@path{{#1}}%
}

\newcommand{\setTitle}[1]{%
	\newcommand{\@phTitle}{#1}%
}
\newcommand{\phTitle}{\@phTitle}

\newcommand{\setDate}[1]{%
	\newcommand{\@phDate}{#1}%
}
\newcommand{\phDate}{\@phDate}

\newcommand{\setUso}[1]{%
	\newcommand{\@uso}{#1}%
}
\newcommand{\uso}{\@uso}

\newcommand{\setVersione}[1]{%
	\newcommand{\@versione}{#1}%
}
\newcommand{\versione}{\@versione}

\newcommand{\disabilitaVersione}{%
	\renewcommand{\setVersione}[1]{}%
	\renewcommand{\versione}{DISABILITATA}
}

\newcommand{\setResponsabile}[1]{%
	\newcommand{\@responsabile}{#1}%
}
\newcommand{\responsabile}{\@responsabile}

\newcommand{\setRedattori}[1]{%
	\newcommand{\@redattori}{#1}%
}
\newcommand{\redattori}{\@redattori}

\newcommand{\setVerificatori}[1]{%
	\newcommand{\@verificatori}{#1}%
}
\newcommand{\verificatori}{\@verificatori}

\newcommand{\setModifiche}[1]{%
	\newcommand{\@modifiche}{#1}%
}
\newcommand{\modifiche}{\@modifiche}

\makeatother 

%--------------------------------------------------
% Comandi per i documenti esterni e il glossario
%--------------------------------------------------

\newcommand{\dext}[1]{\textsc{#1\textsubscript{\textit{D}}}}

\newcommand{\glock}[1]{\textsc{#1\textsubscript{\textit{G}}}}

%--------------------------------------------------
% Comandi per impostare sottotitoli di quarto e quinto livello
%--------------------------------------------------

\setcounter{secnumdepth}{4}
\setcounter{tocdepth}{4}

\titleformat{\paragraph}
{\normalfont\normalsize\bfseries}{\theparagraph}{1em}{}
\titlespacing*{\paragraph}{0pt}{2.25ex plus 1ex minus .2ex}{1.5ex plus .2ex}

\titleformat{\subparagraph}
{\normalfont\normalsize\bfseries}{\thesubparagraph}{1em}{}
\titlespacing*{\subparagraph}{0pt}{1.75ex plus 1ex minus .2ex}{.75ex plus .1ex}

\appendToGraphicsPath{../../commons/res/}

%------------------------------
%
% COMANDI DI CONFIGURAZIONE
%
%------------------------------

\setTitle{Glossario}

\setVersione{0.1.0}

\setDate{06-01-2021}

\setResponsabile{Paolo Scanferlato}

\setRedattori{Paolo Scanferlato}

\setVerificatori{Valton Tahiraj}

\setUso{Esterno}

\setModifiche{
	0.1.0 & Valton Tahiraj    & Verificatore   & 06-01-2021 & Verifica termini studio di fattibilità \\
	0.1.0 & Paolo Scanferlato & Amministratore & 06-01-2021 & Aggiunti termini studio di fattibilità\\
	0.0.1 & Valton Tahiraj    & Verificatore   & 04-01-2021 & Verifica prima stesura\\
	0.0.1 & Paolo Scanferlato & Amministratore & 04-01-2021 & Prima stesura}

\begin{document}

	% Direttive per la creazione del titolo tramite comando maketitle
\title{\huge \textsc{\phTitle{}} \\
	\vspace{11pt} \large \textsc{\phDate{}}}

\author{} % Non toccare
\date{} % Non toccare

%--------------------
% Frontespizio
%--------------------

% Logo del gruppo
\begin{figure}[t!]
	\centering
	\includegraphics[width=20em]{lclong}
\end{figure}

% Titolo / Nome
\maketitle
\thispagestyle{empty}

% Dati specifici sul doc in forma tabulare
\begin{table}[ht]
	\begin{center}
		\label{tab:Dati sul documento}
		\begin{tabular}{r|l}
			\multicolumn{2}{c}{ \textsc{Dati sul documento} } \\
			\hline
			\textbf{Versione} & \versione{} \\
			\textbf{Uso} & \uso{}  \\
			\textbf{Redattori} & \redattori{} \\
			\textbf{Verificatori} & \verificatori{} \\
			\textbf{Responsabile} & \responsabile{} \\
			\textbf{Destinatari} & lineCode \\
								& prof.\ Vardanega Tullio \\		
								& prof.\ Cardin Riccardo \\
			\ifthenelse{\equal{\uso}{Esterno}}{
								& Sanmarco Informatica
			}{} \\
		\end{tabular}
	\end{center}
\end{table}

\newpage

\renewcommand{\arraystretch}{2} % allarga le righe con dello spazio sotto e sopra
\begin{longtable}[H]{>{\centering\bfseries}m{2cm} >{\centering}m{3.5cm} >{\centering}m{2.5cm} >{\centering}m{3cm} >{\centering\arraybackslash}m{5cm}}
	\rowcolor{lightgray}
	{\textbf{Versione}} & {\textbf{Nominativo}} & {\textbf{Ruolo}} & {\textbf{Data}} & {\textbf{Descrizione}}  \\
	\endfirsthead%
	\rowcolor{lightgray}
	{\textbf{Versione}} & {\textbf{Nominativo}}  & {\textbf{Ruolo}} & {\textbf{Data}} & {\textbf{Descrizione}}  \\
	\endhead%
	\modifiche{}%
\end{longtable}
	\newpage

	%--------------------------------
	%
	% IL CONTENUTO INIZIA DA QUI
	%
	%--------------------------------

	\section*{A}
	
	\paragraph*{Amazon CloudWatch}
	Servizio di monitoraggio e osservabilità delle risorse e applicazioni AWS su AWS e sui server locali. Fornisce dati e analisi concrete per monitorare le applicazioni, rispondere ai cambiamenti di prestazioni a livello di sistema, ottimizzare l'utilizzo delle risorse e ottenere una visualizzazione unificata dello stato di integrità operativa.
	
	\paragraph*{Amazon Cognito}
	Servizio di Amazon Web Service che permette di aggiungere strumenti di registrazione, accesso e controllo degli accessi alle applicazioni Web e per dispositivi mobili.
	
	\paragraph*{Amazon Gamelift}
	È una soluzione di hosting di server di giochi che distribuisce, gestisce e dimensiona i server cloud per giochi multigiocatore.
	
	\paragraph*{Amazon Web Service}
	Vedi AWS.
	
	\paragraph*{Angular}
	Framework open source per lo sviluppo di applicazioni web con licenza MIT, evoluzione di AngularJS.
	
	\paragraph*{API}
	Con Application Programming Interface si intende un insieme di procedure e funzioni offerte ai programmatori per facilitare lo sviluppo. Le API espongono blocchi di codice delle librerie di cui fanno parte, per permetterne il riuso.
	
	\paragraph*{API Rest}
	Sono un set di API che rispettano il modello architetturale REST.
	
	\paragraph*{Android}
	Sistema operativo per dispositivi mobili sviluppato da Google Inc., basato su kernel Linux in cui la quasi totalità delle utilità sono costituite da software Java.
		
	\paragraph*{AWS}
	Amazon Web Services, è una collezione di servizi di cloud computing che compongono la piattaforma on-demand di servizi, offerta da Amazon.	Comprende servizi di calcolo, di rete e di storage di basi di dati oltre a molti altri servizi.
	
	\paragraph*{AWS Appsync}
	Servizio completamente gestito che facilita lo sviluppo di API GraphQL gestendo le attività impegnative derivanti dalla connessione sicura a origini dati come AWS Lambda.
	
	\paragraph*{AWS Lambda}
	Servizio di elaborazione che consente di eseguire il codice senza gestire i server o effettuarne il provisioning.
	
	\newpage
	
	\section*{B}
	
	\paragraph*{Back-end}
	Indica generalmente l'interfaccia amministrativa di un applicativo o di un sito web.
	
	\paragraph*{Big Data}
	Una raccolta di dati molto estesa in termini di	volume, velocità e varietà da richiedere tecnologie e metodi analitici specifici per l'estrazione di valore o conoscenza.
	
	\paragraph*{Blockchain}
	È una struttura dati condivisa e immutabile. È definita come un registro digitale le cui voci sono raggruppate in blocchi, concatenati in ordine cronologico, e la cui integrità è garantita dall'uso della crittografia.
	
	\paragraph*{Browser}
	Applicazione per l'acquisizione, la presentazione e la navigazione di risorse sul web.
	
	\newpage
	
	\section*{C}
	
	\paragraph*{Capitolato}
	Documento tecnico redatto dal cliente in cui vengono specificati i vincoli contrattuali	(prezzo e scadenze) per lo sviluppo di un determinato prodotto software. Viene presentato in un bando d'appalto per trovare qualcuno che possa svolgere il lavoro richiesto.
	
	\paragraph*{Cloud}
	Indica un paradigma di erogazione di servizi offerti on-demand da un fornitore ad un cliente finale attraverso la rete Internet.
	
	\paragraph*{CSS}
	Cascaded Style Sheet, è un linguaggio che permette di definire tutte le proprietà di stile e formattazione di una pagina web in maniera modulare, tenendo questa parte della progettazione di un sito separata da quella relativa al contenuto.
	
	\paragraph*{CSV}
	Comma-Separated Values, è un formato di file basato su file di testo utilizzato per l'importazione ed esportazione (ad esempio da fogli elettronici o database) di una tabella di dati.

	\newpage

	\section*{D}
	
	\paragraph*{D3.js}
	È una libreria JavaScript per creare visualizzazioni dinamiche ed interattive partendo da dati organizzati, visibili attraverso un comune browser.
	
	\paragraph*{Deploy} **** DA VERIFICARE **** Deployment
	
	\paragraph*{DevOps}
	Metodo di sviluppo del software che punta alla comunicazione, collaborazione e integrazione tra sviluppatori e addetti alle operations della Information Technology.
	
	\paragraph*{Docker}
	Progetto open-source che automatizza la distribuzione di applicazioni all'interno di contenitori software, fornendo un'astrazione aggiuntiva grazie alla virtualizzazione a livello di sistema operativo di Linux.
	
	\newpage
	
	\section*{E}
	
	\paragraph*{E-commerce}
	È una pratica commerciale che mette in contatto commercianti e acquirenti tramite Internet. Le transazioni di beni e/o servizi vengono effettuate da un negozio online, da un'applicazione mobile e da altri canali di vendita come social network, marketplace, piattaforme di affiliazione, siti di compravendita, ecc.
	
	\paragraph*{Ethereum}
	Ethereum è una piattaforma decentralizzata per la creazione e pubblicazione peer-to-peer di contratti intelligenti (smart contracts).
	
	\paragraph*{Exploratory Data Analysis}
	Spesso abbreviato in EDA, è una tecnica usata nel campo della Data Science per approfondire la conoscenza del dataset su cui si intende lavorare, operazione cruciale per svolgere su di esso qualsiasi tipo di attività.
	
	
	\newpage
	
	\section*{F}
	
	\paragraph*{Force Field} **** DA VERIFICARE ****
	
	\paragraph*{Framework}
	Architettura logica di supporto (spesso un'implementazione logica di un particolare design pattern) su cui un software può essere progettato e realizzato, spesso facilitandone lo sviluppo da parte del programmatore.
	
	\paragraph*{Front-end}
	Indica generalmente l'interfaccia utente di un applicativo o sito web.
	
	\newpage

	\section*{G}
	
	\paragraph*{GGobi}
	È uno strumento software statistico gratuito per la visualizzazione interattiva dei dati.
	
	\paragraph*{Git}
	Software di controllo versione distribuito utilizzabile da interfaccia a riga di comando.
	
	\paragraph*{GitHub}
	Servizio web di hosting per lo sviluppo di progetti software che usa il sistema di controllo di versione Git.
	
	\paragraph*{GraphQL}
	Linguaggio di interrogazione lato server per interfacce di programmazione delle applicazioni, in grado di fornire ai client unicamente i dati di cui hanno bisogno.
	
	\paragraph*{gRPC}
	Sistema di chiamata di procedura remota open source sviluppato inizialmente da Google.
	
	\newpage
	
	\section*{H}
	
	\paragraph*{Heat Map}
	Rappresentazione grafica dei dati dove i singoli valori contenuti in una matrice sono rappresentati da colori.
	
	\paragraph*{HTML}
	Linguaggio di markup per la strutturazione delle pagine web.
	
	\newpage
	
	\section*{I}
	
	\paragraph*{IA}
	L'Intelligenza Artificiale è una disciplina appartenente all'informatica che studia i fondamenti teorici, le metodologie e le tecniche che consentono la progettazione di sistemi hardware e sistemi di programmi software capaci di fornire all'elaboratore elettronico prestazioni che, a un osservatore comune, sembrerebbero essere di pertinenza esclusiva dell'intelligenza umana.
	
	\paragraph*{iOS}
	Sistema operativo mobile sviluppato da Apple per iPhone, iPod touch e iPad.
		
	\newpage
	
	\section*{J}
	
	\paragraph*{Java}
	Linguaggio di programmazione ad alto livello, orientato agli oggetti e a tipizzazione statica, che si appoggia sull'omonima piattaforma software di esecuzione, specificamente progettato per essere il più possibile indipendente dalla piattaforma hardware di esecuzione tramite l'utilizzo di macchina virtuale.
	
	\paragraph*{Javascript}
	Linguaggio di scripting orientato agli oggetti e agli eventi, utilizzato nella programmazione Web sia lato client che server.
	
	\newpage
	
	\section*{K}
	
	\paragraph*{Kotlin}
	Linguaggio di programmazione general purpose, multi-paradigma, open source sviluppato dall'azienda di software JetBrains.
	
	\paragraph*{Kubernetes}
	Sistema open source di orchestrazione e gestione di container.
	
	\newpage 
	
	\section*{L}
	
	\paragraph*{Leaflet}
	Libreria JavaScript per sviluppare mappe geografiche interattive.
	
	\paragraph*{Learning Vector Quantization}
	Algoritmo di classificazione supervisionato basato su prototipo.
	
	\paragraph*{Linux}
	Famiglia di sistemi operativi open source di tipo Unix-like, pubblicati in varie distribuzioni, aventi la caratteristica comune di utilizzare come nucleo il kernel Linux.
	
	\newpage
	
	\section*{M}
	
	\paragraph*{MacOs}
	Sistema operativo desktop distribuito da Apple Inc. per	computer Macintosh che fornisce l'interfaccia utente caratteristica dei sistemi	MacOS ad un sistema operativo di derivazione Unix.
	
	\paragraph*{MAPI}
	Architettura di messaggistica basata sulle API per Microsoft Windows.
	
	\paragraph*{MQTT}
	Protocollo ISO standard di messaggistica leggero di tipo publish-subscribe posizionato in cima a TCP/IP. 
	
	\paragraph*{Multiplayer}
	Modalità di utilizzo in cui più persone partecipano al gioco nello stesso tempo, per mezzo di un solo apparecchio (computer, console, dispositivo mobile, ecc.) con più periferiche oppure usando diversi apparecchi in connessione.
	
	\newpage
	
	\section*{N}
	
	\paragraph*{Node.js}
	Runtime di JavaScript Open source multipiattaforma orientato agli eventi per l'esecuzione di codice JavaScript.
	
	\paragraph*{Next.js}
	Framework web di sviluppo front-end React open source che abilita funzionalità come il rendering lato server e la generazione di siti web statici per applicazioni web basate su React.
	
	\paragraph*{NoSQL} 
	Movimento che promuove sistemi software dove la persistenza dei dati è in generale caratterizzata dal fatto di non utilizzare il modello relazionale, di solito usato dalle basi di dati tradizionali.
	
	\newpage
	
	\section*{O}
	
	\paragraph*{Openshift}
	Piattaforma container per le imprese basata su Kubernetes, che offre operazioni automatizzate in tutto lo stack per gestire deployment di cloud ibridi e multicloud.
	
	\paragraph*{Open Source}
	Software non protetto da copyright e liberamente modificabile dall'utente.
	
	\paragraph*{Orange Canvas}
	*** DA VERIFICARE ***
	
	\newpage
	
	\section*{P}
	
	\paragraph*{POI}
	Point Of Interest, è un punto specifico che qualcuno potrebbe trovare utile o interessante.
	
	\paragraph*{Python}
	Linguaggio di programmazione di più "alto livello" rispetto alla maggior parte degli altri linguaggi, orientato a oggetti, adatto, tra gli altri usi, a sviluppare applicazioni distribuite, scripting, computazione numerica e system testing.
	
	\paragraph*{Proiezione Lineare Multi Asse}
	*** DA VERIFICARE ***
	
	\paragraph*{Power-up}
	*** DA VERIFICARE ***
	
	\paragraph*{Publisher}
	*** DA VERIFICARE ***
	
	\newpage
	
	\section*{Q}
	
	\paragraph*{Qt Framework}
	Libreria multipiattaforma per lo sviluppo di programmi con interfaccia grafica tramite l'uso di widget.
	
	\paragraph*{Query}
	Indica l'interrogazione da parte di un utente su un database, strutturato tipicamente secondo il modello relazionale, per compiere determinate operazioni sui dati.
	
	\newpage
	
	\section*{R}
	
	\paragraph*{Rancher}
	*** DA VERIFICARE ***
	
	\paragraph*{Real-Time}
	*** DA VERIFICARE ***
	
	\paragraph*{Repository}
	Ambiente di un sistema informativo, in cui vengono gestiti i metadati, attraverso tabelle relazionali.
	
	\paragraph*{RFID}
	Radio-Frequency IDentification, è la tecnologia di identificazione automatica basata sulla propagazione nell'aria di onde elettro-magnetiche, consentendo la rilevazione univoca, automatica (hand free), massiva e a distanza di oggetti, animali e persone sia statici che in movimento.
		
	\newpage	
	
	\section*{S}
	
	\paragraph*{Scatter plot Matrix}
	*** DA VERIFICARE ***
	
	\paragraph*{Self Organizing Map}
	Tipo di organizzazione di processi di informazione in rete analoghi alle reti neurali artificiali.
	
	\paragraph*{SEO}
	Search Engine Optimization, definisce tutte le attività di ottimizzazione di un sito web volte a migliorarne il posizionamento nei risultati organici dei motori di ricerca.
	
	\paragraph*{Singleplayer}
	Nel mondo dei videogiochi indica la modalità di gioco in cui una sola persona prende parte al gioco per tutta la durata della partita.
	
	\paragraph*{Smart Contract}
	Gli Smart Contract sono protocolli informatici che facilitano, verificano, o fanno rispettare, la negoziazione o l'esecuzione di un contratto, permettendo talvolta la parziale o la totale esclusione di una clausola contrattuale. Di solito, hanno anche un'interfaccia utente e spesso simulano la logica delle clausole contrattuali.
	
	\paragraph*{Solidity}
	Linguaggio di programmazione orientato agli oggetti per scrivere contratti intelligenti. Viene utilizzato per implementare contratti intelligenti su varie piattaforme Blockchain, in particolare Ethereum.
	
	\paragraph*{SQL}
	Linguaggio standardizzato per database basati sul modello relazionale, progettato per creare e modificare schemi di database; inserire, modificare e gestire dati memorizzati; interrogare i dati memorizzati; creare e gestire strumenti di controllo e accesso ai dati.
	
	\paragraph*{Subscriber}
	*** DA VERIFICARE ***
	
	\paragraph*{Swift}
	Linguaggio di programmazione orientato agli oggetti per sistemi macOS, iOS, watchOS, tvOS e Linux. È concepito per coesistere con il linguaggio Objective-C, tipico degli sviluppi per i sistemi operativi Apple, semplificando la scrittura del codice.
	
	\newpage
	
	\section*{T}
	
	\paragraph*{Tomcat}
	Apache Tomcat è un server web open source sviluppato dalla Apache Software Foundation.
	
	\paragraph*{t-SNE}
	Algoritmo di riduzione della dimensionalità ampiamente utilizzato come strumento di apprendimento automatico in molti ambiti di ricerca.
	
	\paragraph*{Typescript}
	Linguaggio di programmazione open source sviluppato da Microsoft.
	
	\newpage
	
	\section*{U}
	
	\paragraph*{User-interface}
	Interfaccia uomo-macchina, ovvero ciò che si frappone tra una macchina e un utente, consentendone l'interazione reciproca.
	
	\paragraph*{UMAP}
	Piattaforma che fornisce una interfaccia grafica di semplice utilizzo per personalizzare mappe e geolocalizzare dataset appositamente creati o recuperati negli archivi Open Data.
	
	\newpage
	
	\section*{V}
	
	\paragraph*{Vyper}
	*** DA VERIFICARE ***
	
	\newpage
	
	\section*{W}
	
	\paragraph*{Windows}
	Famiglia di ambienti operativi e sistemi operativi prodotta da Microsoft Corporation dal 1985, orientato a personal computer, workstation, server e smartphone; prende il nome dall'interfaccia di programmazione di un'applicazione a finestre, detta File Explorer.
	
	\newpage
	
	\section*{X}

	\newpage
	
	\section*{Y}

	\newpage
	
	\section*{Z}
	
	\paragraph*{Zextras Drive}
	*** DA VERIFICARE ***

\end{document}