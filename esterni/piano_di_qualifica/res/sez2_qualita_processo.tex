

\subsubsection{Introduzione}
Durante lo svolgimento del progetto sono stati identificati e perseguiti criteri di qualità al fine di realizzare un prodotto efficace. Tali criteri permettono un miglioramento continuo tramite il monitoraggio e la stima degli stessi.\\
All'interno del documento norme di progetto sezione 7.1.9, sono stati identificati i criteri necessari che verranno qui di seguito analizzati.
\subsubsection{Indici di qualità}

\begin{table} [h!]
	\rowcolors{2}{gray!25}{gray!6}
	\begin{center}
		\begin{tabular} {m{2 cm} m{7 cm} c c }
			\rowcolor{lightgray}
			\textbf{ID} & \textbf{Nome}& \textbf{Valore Accettabile} & \textbf{Valore Ottimale}\\
			MP01 & Schedule Variance (SV)   & $\geq$ -5\%    & 0\% \\
			MP02 & Budgeted Cost of Work Performed (BCWP) & //           & // \\
			MP03 & Budgeted Cost of Work Scheduled (BCWS) & //           & // \\
			MP04 & Cost Variance (CV)   & $\geq$ -5\% & 0\% \\
			MP05 & Actual Cost of Work Performed (ACWP) & //           & // \\
			MP06 & Unbudgeted Risks (UR)   & $\leq$ 5 rischi & 0 rischi\\
			MP07 & Defects Removal Efficiency (DRE)  & $\geq$  80\% & 100\%\\

		\end{tabular}
	\end{center}
\end{table}





