Questa sezione si focalizza sul miglioramento della produttività dei processi coinvolti nella
realizzazione del prodotto descritto nel \glock{capitolato} scelto. Essendo il primo progetto realistico affrontato dai membri del gruppo, problemi di natura organizzativa interna, di adempimento efficace
dei ruoli assegnati e di giusto utilizzo degli strumenti scelti sono dietro l’angolo. Per far fronte a queste
possibili problematiche e cercare di migliorare in maniera costante la produttività del gruppo, verranno elencati i problemi più grandi rilevati e le relative contromisure, man mano che saranno identificati
nel corso della realizzazione del prodotto.

\subsection{Valutazioni sull'organizzazione}
\begin{table} [h!]
	\rowcolors{2}{gray!25}{gray!6}
	\begin{center}
		\begin{tabular} { m{8cm} m{8cm}  }
			\rowcolor{lightgray}
			\textbf{Problema rilevato} & \textbf{Contromisura}\\
			Nei primi mesi di lavoro è stato difficile trovarsi data la attuale situazione di emergenza globale. & E' stato creato un canale \glock{discord} per garantire una comunicazione testuale e vocale costante tra alcuni o tutti i membri del gruppo.
			
		\end{tabular}
	\end{center}
	\caption{Valutazioni organizzazione}
\end{table}

\subsubsection{Valutazione sui ruoli}	
\rowcolors{2}{gray!6}{gray!25}
\setlength{\tabcolsep}{10pt}
\begin{longtable}[h!] { c  m{7 cm} m{6cm}}
	\caption{Valutazione ruoli} \\
	\rowcolor{lightgray}
	\thead{Test}  & \thead{Descrizione} & \thead{Esito} \\ \endhead%
			Responsabile & Durante la fase relativa alla stesura dei documenti per proporsi ufficialmente come fornitori per il capitolato, i compiti sono stati assegnati dal responsabile su base volontaria, portando così a situazioni in cui alcuni membri si sono ritrovati sovraccarichi di lavoro
			mentre altri erano tenuti a svolgere attività più sbrigative. Questo fatto ha portato alcuni colleghi a chiedere una ridistribuzione dei compiti.& Dopo ogni assegnazione di compiti da svolgere, il responsabile di turno si deve impegnare a ricontrollare se sono stati spartiti equamente tra i membri, cosicché non si subiscano rallentamenti per via di ridistribuzioni degli oneri di progetto. \\			
			Analista & 	Essendo il compito dell'analista quello di guidare il gruppo nel migliorare processi, prodotti, servizi e software attraverso l'analisi dei dati è stato difficile inizialmente coordinarsi efficientemente con gli altri membri del gruppo. & Gli analisti hanno deciso di svolgere quest'attività scambiandosi continuamente i propri risultati e chiedendo chiarimenti agli altri quando necessario.\\
			Verificatori &	Data l’inesperienza dei membri nell'attività di stesura dei documenti, è indispensabile che i verificatori controllino ogni sezione scrupolosamente.	Per fare ciò però,  devono avere una buona conoscenza di tutte le tematiche trattate nella documentazione. & Per risolvere questi problemi, ai verificatori
			si è deciso di non assegnare altri compiti durante lo svolgimento del processo di verifica, in
			modo che avessero il tempo materiale per attuarlo nel modo più efficace possibile.\\
			Amministratore & Per redarre alcune sezioni di certi documenti, i membri che hanno ricoperto questo ruolo hanno trovato delle difficoltà relative alla mancata presenza di materiale autorevole e chiaro che le descrivesse. & Si è deciso di prendere spunto dalla documentazione prodotta dai colleghi degli anni passati, confrontarla con quanto appreso durante il corso di ingegneria del software e procedere alla loro stesura cercando di trattare ogni sezione in modo approfondito e chiaro. \\
	\end{longtable}


\subsection{Valutazioni sugli strumenti}
\begin{table} [h!]
	\rowcolors{2}{gray!25}{gray!6}
	\begin{center}
		\begin{tabular} { m{2cm} m{7cm} m{6,5cm} }
			\rowcolor{lightgray}
			\textbf{Strumento} & \textbf{Problema rilevato} & \textbf{Contromisura}\\
			\glock{Version Control System} & Nel \glock{workflow} del particolare \glock{VCS} utilizzato, si sono fissate diverse regole su come condividere il materiale prodotto nel server. Queste regole non sono state attuate da alcuni membri, vista l’inesperienza, e ciò ha portato a rallentamenti nella fase di condivisione del materiale scelto per poter chiudere le \glock{milestone} relative e procedere ad aprire quelle successive. & E’ stato condiviso un documento testuale che spiega nel dettaglio come ottemperare alle regole scelte per l’utilizzo dello strumento, in modo da facilitare i membri più in difficoltà e non rendere il \glock{VCS} un ostacolo all'avanzamento del gruppo verso la realizzazione della documentazione necessaria. \\
			\LaTeX\ &	Questo strumento di scrittura è stato una novità per quasi tutti i membri del gruppo. All'inizio, in molti avevano errori di compilazione nei propri file, per cui non riuscivano a produrre opportuni file pdf. & Dopo il primo mese e mezzo dalla formazione del gruppo, i membri più esperti di \LaTeX\ hanno creato un template da utilizzare per la produzione della documentazione, riducendo ai minimi termini il numero di comandi da imparare per utilizzare questo software efficacemente.
		\end{tabular}
	\end{center}
\caption{Valutazione strumenti}
\end{table}
