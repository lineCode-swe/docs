\subsection{Verifica della documentazione}
Per la verifica della documentazione si utilizza la tecnica del \glock{walkthrough} revisionando i documenti per intero e la tecnica di \glock{inspection} secondo i punti descritti nelle \dext{Norme di Progetto v1.0.0}.

\subsection{Calcolo leggibilità e controllo ortografia dei documenti}
Per verificare quanto sono leggibili i documenti redatti si utilizza \glock{l'indice di gulpease} ed è stato effettuato un controllo di ortografia, di seguito il grafico contenente i risultati ottenuti durante il periodo di analisi dei requisiti:

\begin{table} [h!]
	\rowcolors{2}{gray!25}{gray!6}
	\begin{center}
		\begin{tabular} { c c c c}
			\rowcolor{lightgray}
			\textbf{Documento}&\textbf{Errori ortografici}&\textbf{Indice di Gulpease}&\textbf{Esito}\\
			\dext{Piano di progetto v1.0.0}	&0    						&xxx				&Superato\\
			\dext{Norme di progetto v1.0.0} &0							&xxx					&Superato\\
			\dext{Studio di fattibilità v1.0.0}	&0						&xxx					&Superato\\
			\dext{Glossario v1.0.0}			&0							&xxx					&Superato\\
			\dext{Piano di qualifica v1.0.0}	&0						&xxx					&Superato\\
			\dext{Media verbali v1.0.0}			&0						&xxx					&Superato\\
			\dext{Analisi dei requisiti v1.0.0}	&0						&xxx					&Superato\\
		\end{tabular}
	\end{center}
\caption{Tabella delle medie degli errori di ortografia e dell'\glock{indice di Gulpease} durante il periodo di analisi dei requisiti}
\end{table}

