\subsection{Scopo del documento}
Il documento ha il fine di esporre i metodi di verifica e validazione adottate dal gruppo lineCode per garantire la qualità di prodotto e di processo. Per far sì che questo accada, viene utilizzato un sistema di verifica persistente che permette l'identificazione di potenziali errori in modo da poterli risolvere in maniera efficiente ed efficace.

\subsection{Scopo del prodotto}
Il \glock{capitolato} C5 ha come obbiettivo la realizzazione di un applicativo \glock{Real-Time} in grado di guidare delle unità dotate di mobilità autonoma e semiautonoma in ambienti specifici, partendo dal presupposto che queste si muovano in ambienti in cui sono presenti altre unità (autonome o meno).

\subsection{Glossario e Documenti esterni}
Per evitare possibili ambiguità relative alle terminologie (che andranno indicate in MAIUSCOLETTO) utilizzate	nei vari documenti, verranno utilizzate due simboli:
\begin{itemize}
	\item \textit{D}: indica un documento specifico;
	\item \textit{G}: indica un termine presente nel documento \dext{glossario}.
\end{itemize}

\subsection{Riferimenti}

\subsubsection{Normativi}
\begin{itemize}
	\item \textbf{Norme di Progetto}: \dext{Norme di Progetto v1.0.0};
	\item \textbf{\glock{Capitolato} C5 - PORTACS}:					\url{https://www.math.unipd.it/~tullio/IS-1/2020/Progetto/C5.pdf}.
\end{itemize}


\subsubsection{Informativi}
\begin{itemize}
	\item \textbf {ISO/IEC 9126}:
	\url {https://en.wikipedia.org/wiki/ISO/IEC_9126}
	\item \textbf{ISO/IEC 12207:1995}:
	\url{https://www.math.unipd.it/~tullio/IS-1/2009/Approfondimenti/ISO_12207-1995.pdf}
	\item \textbf {Indici di Gulpease}:
	\url{https://it.wikipedia.org/wiki/Indice_Gulpease}
	\item \textbf {Slide del corso di ingegneria del software, qualità del software}:
	\url{https://www.math.unipd.it/~tullio/IS-1/2019/Dispense/L12.pdf}
	\item \textbf {Slide del corso di ingegneria del software, qualità del processo}:
	\url{https://www.math.unipd.it/~tullio/IS-1/2019/Dispense/L13.pdf}
\end{itemize}