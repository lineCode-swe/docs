La classificazione dei test è riportata alla sezione Verifica delle \dext{Norme di progetto v2.0.0}. \\

\subsection{Tipologie di test}
I test saranno di quattro tipologie:
\begin{itemize}
	\item \textbf{Test di Sistema [TS] };
	\item \textbf{Test di Accettazione [TA]};
	\item \textbf{Test di Integrazione [TI]};
	\item \textbf{Test di Unità [TU]}.\\
\end{itemize}

\subsubsection{Test di sistema}

	\newcommand*{\thead}[1]{\multicolumn{1}{c}{\bfseries #1}}
	\rowcolors{2}{gray!6}{gray!25}
	\setlength{\tabcolsep}{10pt}
	\begin{longtable}[h!] { c  m{12cm} c}
		\caption{Test di Sistema} \\
		\rowcolor{lightgray}
		\thead{Test}  & \thead{Descrizione} & \thead{Esito} \\ \endhead%

		TSMF1   & La mappa deve essere disponibile per l'utente generico in sola lettura	& NI \\

		TSMF1.1 & L'utente generico deve poter visualizzare la legenda dei simboli utilizzati all'interno della mappa & NI \\

		TSFM1.2 & La mappa deve aggiornarsi periodicamente, mostrando le informazioni sul movimento delle unità che fanno parte del sistema & NI\\

		TSDM1.3 & La mappa deve mostrare le informazioni sui pedoni che comunicano con il sistema & NI \\

		TSFM1.4 & La mappa deve essere visualizzabile per ogni tipo di utente & NI\\


		TSFM2   & L'utente registrato deve poter effettuare il login. All'utente viene chiesto di:
				\begin{itemize}
					\item accedere all'area login;
					\item inserire il nome utente;
					\item inserire la password.
				\end{itemize}
								& NI \\

		TSFM2.1 & L'utente deve visualizzare un messaggio di errore nel caso in cui:
					\begin{itemize}
						\item viene fornito un nome utente non esistente;
						\item viene fornita una password errata.
					\end{itemize}
									& NI \\
		TSFM3   & L'utente autenticato deve poter effettuare il logout dal sistema. L'utente può verificare che la disconnessione sia avvenuta con successo & NI \\

		TSFM4   & La guida utente deve essere visualizzabile dai soli utenti registrati & NI \\

		TSFM4.1 & La guida utente deve contenere tutte le istruzioni che un utente può svolgere all'interno del sistema & NI \\

		TSFM5   & L'utente autenticato deve poter visualizzare la lista delle unità attive con il proprio ID & NI \\

		TSFM5.1 & Ogni elemento della lista di unità fornisce un ambiente grafico per la gestione delle proprietà dell'unità selezionata & NI \\

		TSFO5.2 & L'utente autenticato, dentro l'ambiente grafico, deve visualizzare l'ID dell'unità selezionata & NI \\

		TSFM5.3 & L'utente autenticato deve poter visualizzare le seguenti informazioni dell'unità:
					\begin{itemize}
						\item le coordinate della posizione attuale all'interno della mappa;
						\item lo stato in cui si trova;
						\item la velocità attuale;
						\item la direzione del prossimo passo suggerita dal sistema;
						\item la coda degli ordini assegnati.
					\end{itemize}
									&NI\\

		TSFM5.4  & L'utente autenticato deve poter impartire i seguenti comandi all'unità:
					\begin{itemize}
						\item \underline{Start};
						\item \underline{Stop};
						\item \underline{Go Back};
						\item \underline{Shutdown}.
					\end{itemize}
									& NI \\

		TSFM5.5  & L'utente autenticato deve poter aggiungere un nuovo ordine alla coda degli ordini assegnati all'unità & NI\\

		TSFM5.6  & L'utente autenticato deve poter eliminare un ordine alla coda degli ordini assegnati all'unità & NI\\

		TSFM6 & L'amministratore deve poter visualizzare la lista degli utenti registrati, con le seguenti informazioni:
					\begin{itemize}
						\item nome utente;
						\item password.
					\end{itemize}
									& NI \\
		TSFM6.1 & L'amministratore deve poter creare un nuovo utente. Il sistema richiede in input le seguenti credenziali:
					\begin{itemize}
						\item nome utente;
						\item password;
						\item status.
					\end{itemize}
										& NI \\

		TSFM6.2  & L'amministratore deve poter modificare un utente registrati. Il sistema richiede in input le seguenti credenziali:
					\begin{itemize}
						\item nome utente;
						\item password.
					\end{itemize}
											& NI \\
		TSFM6.3 & L'amministratore deve poter eliminare un utenti registrato & NI \\

		TSFM6.4 & L'amministratore deve visualizzare un messaggio di errore nel caso in cui:
						\begin{itemize}
						\item il nome utente inserito sia già esistente;
						\item la password non è formattata correttamente.
					\end{itemize}
										& NI \\
		TSFM7   & L'amministratore per poter modificare la mappa, deve poter inserire un file da validare & NI\\

		TSFM7.1 & L'amministratore deve visualizzare un messaggio di errore nel caso in cui il file non sia formattato correttamente & NI \\

		TSFM8   & L'amministratore deve poter visualizzare la lista delle unità registrate, con il proprio ID & NI\\

		TSFM8.1 & L'amministratore deve poter aggiungere una nuova unità alla liste delle unità registrate. Il sistema richiede in input l'ID & NI\\

		TSFM8.2 & 	L'amministratore deve poter modificare la lista delle unità registrate. Il sistema richiede in input il rispettivo ID & NI\\

		TSFM8.3 & L'amministratore deve poter eliminare una unità dalla lista & NI\\

		TSFM8.4  & L'amministratore deve visualizzare un messaggio d'errore nel caso venga fornito un ID già esistente & NI \\

		TSFM9   & L'unità, dotata di sensoristica, deve comunicare al sistema la posizione relativa degli ostacoli che riescono a rilevare & NI \\

		TSFM10 & Il sistema deve ricevere le coordinate dall'unità sulla sua posizione attuale & NI \\

		TSFM11 & L'unità deve avere un percorso fornito dal sistema per raggiungere il prossimo \glock{POI} & NI \\

		TSFM12 & L'unità deve avere una lista di \glock{POI} da raggiungere & NI \\

		TSFM13  & L'utente autenticato può modificare la lista deglli ordini solo quando l'unità si trova in una base di ricarica & NI \\

		TSFM14  & L'unità, con coda degli ordini vuota, deve ritornare nella base di ricarica & NI \\

		TSFM15  & L'unità non può superare un certo limite di velocità & NI \\

		TSFM16 & L'unità può assumere uno dei seguenti stati:
				\begin{itemize}
					\item \underline{Going to X};
					\item \underline{Stop};
					\item \underline{Base};
					\item \underline{Error Y}.
				\end{itemize}
										& NI \\

		TSFM17  & Il sistema dispone di un algoritmo che deve:
				\begin{itemize}
					\item calcolare il percorso che l'unità deve percorrere per raggiungere il \glock{POI};
					\item evitare la collisione con altre unità;
					\item evitare la collisione con gli ostacoli.
				\end{itemize}
										& NI \\

		TSFO18  & Il percorso restituito dall'algoritmo, è un percorso ottimo & NI\\

		TSFD19 & Il sistema riceve dal pedone le informazioni sulla sua posizione & NI\\

		TSFD20  & Il sistema riconosce il pedone come un ostacolo & NI \\
		\hline
		
		TSMC1   & Ogni entità sviluppata facente parte del sistema dovrà essere contenuta in un container \glock{Docker} & NI \\
		
		TSMC2   & Deve essere fornito un \glock{Dockerfile} contenente:
						\begin{itemize}
							\item il motore di calcolo;
							\item il visualizzatore \glock{Real-Time}.
						\end{itemize}
											& NI \\
											
		TSMC2.1 & Deve essere fornito un \glock{Dockerfile} per la singola unità & NI \\
		
		TSD2.2 &  Deve essere fornito un \glock{Dockerfile} per il singolo pedone & NI \\
		\hline

		TSMQ1 & Il prodotto va rilasciato con la licenza \glock{open-source} più aperta possibile in base alle librerie utilizzate & NI \\

		TSMQ2 & Il prodotto deve essere conforme con quanto dichiarato nel documento \dext{ Piano di Qualifica v2.0.0} & NI \\

		TSMQ3  & Devono essere realizzati test di unità e di integrazione per vericare le singole componenti del prodotto & NI \\

		\hline

		TSMP1  &  Il tempo di aggiornamento della mappatura rispetto alla posizione delle unità nell'ambiente deve essere $\geq$ 5 secondi & NI \\

		TSDP1.1  & Il tempo di aggiornamento della mappatura rispetto alla posizione delle unità nell'ambiente deve essere $\geq$ 2.5 secondi & NI \\

		TSOP1.2  &  Il tempo di aggiornamento della mappatura rispetto alla posizione delle unità nell'ambiente deve essere $\geq$ 1 secondi & NI \\

		TSMP2	& Il servizio non deve mai essere interrotto secondo una logica di \glock{zero downtime}	& NI \\\\
\end{longtable}

\subsubsection{Test di accettazione}
I test di Accettazione hanno lo scopo di dimostrare che il prodotto sviluppato soddisfi i requisiti presenti nel capitolato e concordati con il proponente e, con quest'ultimo, eseguito durante il collaudo finale del prodotto.

\subsubsection{Test di integrazione}
Le specifiche di questi test verranno scritte successivamente rispettando il \glock{Modello a V}.
\subsubsection{Test di Unità}
Le specifiche di questi test verranno scritte successivamente rispettando il \glock{Modello a V}.
