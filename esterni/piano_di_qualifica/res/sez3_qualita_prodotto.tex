Per garantire e valutare la qualità del prodotto, il gruppo ha deciso di fare riferimento allo standard ISO/IEC 9126, il quale definisce i parametri per produrre un prodotto di buona qualità ed il raggiungimento di tale caratteristica.

\subsection{Qualità dei documenti}
Oltre alle qualità presenti nello standard sopra citato, il gruppo ha deciso di utilizzare altri parametri per quantificare la qualità della documentazione fornita con il prodotto software; le quali, riportate di seguito.

    \subsubsection{Comprensione}
    Tutti i documenti devono essere leggibili e comprensibili.\\
    Queste qualità derivano dalla correttezza lessicografica, grammaticale e semantica.

    \subsubsection{Obiettivi}
    \begin{itemize}
        \item \textbf{leggibilità:} per garantire la leggibilità dei documenti si è deciso di utilizzare l'\glock{indice di gulpease} come indicatore per questa caratteristica;
        \item \textbf{correttezza:} i documenti presentati non devono contenere errori ortografici di alcun genere.
    \end{itemize}

    \subsubsection{Metriche}
    \begin{table} [h!]
    	\rowcolors{2}{gray!25}{gray!6}
    	\begin{center}
    		\begin{tabular} {m{2 cm} m{7 cm} c c }
    			\rowcolor{lightgray}
    			\textbf{ID} & \textbf{Nome}           & \textbf{Valore Accettabile} & \textbf{Valore Ottimale}\\
    			MD01        & Indice di Gulpease      & $\geq$ 40                   & $\geq$ 60\\
    			MD02        & Correttezza ortografica & 0                           & 0
    		\end{tabular}
    	\caption{Metriche qualità dei documenti}
    	\end{center}
    \end{table}

\subsection{Qualità del software}
Le seguenti metriche sono state scelte al fine di garantire affidabilità, sicurezza, manutenibilità e semplicità del prodotto software.

    \subsubsection{Metriche}
    \begin{table} [h!]
        \rowcolors{2}{gray!25}{gray!6}
        \begin{center}
            \begin{tabular} {m{2 cm} m{7 cm} c c }
                \rowcolor{lightgray}
                \textbf{ID} & \textbf{Nome}           & \textbf{Valore Accettabile} & \textbf{Valore Ottimale}\\
                MS01        & Numero Bug Rilevati     & 0                           & 0\\
                MS02        & Vulnerabilità           & 0                           & 0\\
                MS03        & Code Smells             & 25                          & 5\\
                MS04        & Debito Tecnico          & h 24                        & h 0,5\\
                MS05        & Code Coverage           & 60\%                        & 80\%\\
                MS06        & Complessità Ciclomatica & 30                          & 10\\
                MS07        & Requisiti Sbbligatori Soddisfatti & 100\%             & 100\%\\
                MS08        & Requisiti Desiderabili Soddisfatti & 0\%              & 70\%\\
                MS09        & Requisiti Opzionali Soddisfatti & 0\%                 & 20\%\\
            \end{tabular}
            \caption{Metriche qualità del prodotto software}
        \end{center}
    \end{table}