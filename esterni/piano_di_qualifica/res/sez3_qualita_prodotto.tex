Per garantire e valutare la qualità del prodotto, il gruppo ha deciso di fare riferimento allo standard ISO/IEC 9126, il quale definisce i parametri per produrre un prodotto di buona qualità ed il raggiungimento di tale caratteristica. \\
Oltre alle qualità presenti nello standard sopra citato, il gruppo ha deciso di utilizzare altri parametri per quantificare la qualità della documentazione fornita con il prodotto software; le quali, riportate di seguito.

\subsection{Qualità dei documenti}

\subsubsection{Comprensione}
Tutti i documenti devono essere leggibili e comprensibili.\\
Queste qualitá derivano dalla correttezza lessicografica, grammaticale e semantica.

\subsubsection{Obiettivi}
\begin{itemize}
    \item \textbf{leggibilità:}per garantire la leggibilità dei documenti si è deciso di utilizzare \glock{l'indice di gulpease} come indicatore per questa caratteristica;
    \item \textbf{correttezza:} i documenti presentati non devono contenere errori ortografici di alcun genere.
\end{itemize}

\subsubsection{Metriche}

\begin{table} [h!]
	\rowcolors{2}{gray!25}{gray!6}
	\begin{center}
		\begin{tabular} {m{2 cm} m{7 cm} c c }
			\rowcolor{lightgray}
			\textbf{ID} & \textbf{Nome}& \textbf{Valore Accettabile} & \textbf{Valore Ottimale}\\
			MD01 & Indice di Gulpease  		 & $\geq$ 40\%    			& $\geq$ 60\% \\
			MD02 & Correttezza ortografica 				&0						&0
		\end{tabular}
	\end{center}
\end{table}