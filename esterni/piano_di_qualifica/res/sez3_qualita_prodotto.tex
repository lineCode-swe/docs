Per garantire e valutare la qualità del prodotto il gruppo ha deciso di fare riferimento allo standard
ISO/IEC 9126, il quale definisce i parametri per produrre un prodotto di buona qualità, questi parametri
quantificano il grado di raggiungimento di tale caratteristica. Oltre alle qualità presenti nello
standard sopra citato il gruppo ha deciso di utilizare altri parametri per quantificare la qualità della
documentazione fornita con il prodotto software. Di seguito sono riportate le qualità che il gruppo ha
ritenuto appropriate per quanto riguarda lo stato attuale del progetto.

\subsection{Qualità dei documenti}

\subsubsection{Comprensione}
Tutti i documenti devono essere leggibili e comprensibili, queste qualitá derivano dalla correttezza
lessicografica, grammaticale, e semantica.

\subsubsection{Obiettivi}
\begin{itemize}
    \item \textbf{leggibilità:}per garantire la leggibilità dei documenti si è deciso di utilizzare \glock{l'indice di gulpease} come indicatore per questa caratteristica;
    \item \textbf{correttezza:} i documenti presentati non devono contenere errori ortografici di alcun genere.
\end{itemize}

\subsubsection{Metriche}
La comprensione dei documenti viene valutata dai seguenti criteri:
\begin{itemize}
\item \textbf{MD01} indice di gulpease;
\item \textbf{MD02} correttezza ortografica.

\end{itemize}