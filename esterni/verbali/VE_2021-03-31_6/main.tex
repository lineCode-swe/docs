\documentclass[]{article}

\usepackage[italian]{babel}
\usepackage[margin=20mm, footskip = 20pt]{geometry}
\usepackage{array}
\usepackage{tabularx}
\usepackage{graphicx}
\usepackage{subfiles}
\usepackage{hyperref}
\usepackage{nameref}
\usepackage{titlesec}
\usepackage{longtable}
\usepackage[table]{xcolor}
\usepackage{titling}
\usepackage{lastpage}
\usepackage{ifthen}
\usepackage{calc}
\usepackage{soulutf8}
\usepackage{contour}
\usepackage{float}
\usepackage{fancyhdr}
\usepackage{multirow}
\usepackage{pgfgantt}
\usepackage{lscape}

\newcommand{\hr}{\par\vspace{-.1\ht\strutbox}\noindent\hrulefill\par}

\graphicspath{ {./}
	{./commons/res}
}

%--------------------------------------------------
% Comandi per inserire contenuto del documento
%--------------------------------------------------
\makeatletter

\newcommand\appendToGraphicsPath[1]{%
	\g@addto@macro\Ginput@path{{#1}}%
}

\newcommand{\setTitle}[1]{%
	\newcommand{\@phTitle}{#1}%
}
\newcommand{\phTitle}{\@phTitle}

\newcommand{\setDate}[1]{%
	\newcommand{\@phDate}{#1}%
}
\newcommand{\phDate}{\@phDate}

\newcommand{\setUso}[1]{%
	\newcommand{\@uso}{#1}%
}
\newcommand{\uso}{\@uso}

\newcommand{\setVersione}[1]{%
	\newcommand{\@versione}{#1}%
}
\newcommand{\versione}{\@versione}

\newcommand{\disabilitaVersione}{%
	\renewcommand{\setVersione}[1]{}%
	\renewcommand{\versione}{DISABILITATA}
}

\newcommand{\setResponsabile}[1]{%
	\newcommand{\@responsabile}{#1}%
}
\newcommand{\responsabile}{\@responsabile}

\newcommand{\setRedattori}[1]{%
	\newcommand{\@redattori}{#1}%
}
\newcommand{\redattori}{\@redattori}

\newcommand{\setVerificatori}[1]{%
	\newcommand{\@verificatori}{#1}%
}
\newcommand{\verificatori}{\@verificatori}

\newcommand{\setModifiche}[1]{%
	\newcommand{\@modifiche}{#1}%
}
\newcommand{\modifiche}{\@modifiche}

\makeatother 

%--------------------------------------------------
% Comandi per i documenti esterni e il glossario
%--------------------------------------------------

\newcommand{\dext}[1]{\textsc{#1\textsubscript{\textit{D}}}}

\newcommand{\glock}[1]{\textsc{#1\textsubscript{\textit{G}}}}

%--------------------------------------------------
% Comandi per impostare sottotitoli di quarto e quinto livello
%--------------------------------------------------

\setcounter{secnumdepth}{4}
\setcounter{tocdepth}{4}

\titleformat{\paragraph}
{\normalfont\normalsize\bfseries}{\theparagraph}{1em}{}
\titlespacing*{\paragraph}{0pt}{2.25ex plus 1ex minus .2ex}{1.5ex plus .2ex}

\titleformat{\subparagraph}
{\normalfont\normalsize\bfseries}{\thesubparagraph}{1em}{}
\titlespacing*{\subparagraph}{0pt}{1.75ex plus 1ex minus .2ex}{.75ex plus .1ex}

\appendToGraphicsPath{../../../commons/res/}

%------------------------------
%
% COMANDI DI CONFIGURAZIONE
%
%------------------------------

\setTitle{Verbale esterno \#6}

\setVersione{0.1.0}

\setDate{05-04-2021}

\setResponsabile{Alessandro Chimetto}

\setRedattori{Valton Tahiraj}

\setVerificatori{Lucia Fenu}

\setUso{Esterno}

\setModifiche{

	0.1.0 & Lucia Fenu & Verificatore & 05-04-2021 & Verifica della prima stesura\\
	0.0.0 & Valton Tahiraj & Amministratore & 04-04-2021 & Stesura iniziale}

\begin{document}

	% Direttive per la creazione del titolo tramite comando maketitle
\title{\huge \textsc{\phTitle{}} \\
	\vspace{11pt} \large \textsc{\phDate{}}}

\author{} % Non toccare
\date{} % Non toccare

%--------------------
% Frontespizio
%--------------------

% Logo del gruppo
\begin{figure}[t!]
	\centering
	\includegraphics[width=20em]{lclong}
\end{figure}

% Titolo / Nome
\maketitle
\thispagestyle{empty}

% Dati specifici sul doc in forma tabulare
\begin{table}[ht]
	\begin{center}
		\label{tab:Dati sul documento}
		\begin{tabular}{r|l}
			\multicolumn{2}{c}{ \textsc{Dati sul documento} } \\
			\hline
			\textbf{Versione} & \versione{} \\
			\textbf{Uso} & \uso{}  \\
			\textbf{Redattori} & \redattori{} \\
			\textbf{Verificatori} & \verificatori{} \\
			\textbf{Responsabile} & \responsabile{} \\
			\textbf{Destinatari} & lineCode \\
								& prof.\ Vardanega Tullio \\		
								& prof.\ Cardin Riccardo \\
			\ifthenelse{\equal{\uso}{Esterno}}{
								& Sanmarco Informatica
			}{} \\
		\end{tabular}
	\end{center}
\end{table}

\newpage

\renewcommand{\arraystretch}{2} % allarga le righe con dello spazio sotto e sopra
\begin{longtable}[H]{>{\centering\bfseries}m{2cm} >{\centering}m{3.5cm} >{\centering}m{2.5cm} >{\centering}m{3cm} >{\centering\arraybackslash}m{5cm}}
	\rowcolor{lightgray}
	{\textbf{Versione}} & {\textbf{Nominativo}} & {\textbf{Ruolo}} & {\textbf{Data}} & {\textbf{Descrizione}}  \\
	\endfirsthead%
	\rowcolor{lightgray}
	{\textbf{Versione}} & {\textbf{Nominativo}}  & {\textbf{Ruolo}} & {\textbf{Data}} & {\textbf{Descrizione}}  \\
	\endhead%
	\modifiche{}%
\end{longtable}

	\newpage

	%--------------------------------
	%
	% IL CONTENUTO INIZIA DA QUI
	%
	%--------------------------------

	\section{Introduzione}
	\subsection{Luogo e data dell'incontro}
	\begin{itemize}
		\item \textbf{Modalità}: Telematica;
		\item \textbf{Software utilizzato}: Google Meet;
		\item \textbf{Data}: 31 Marzo 2021;
		\item \textbf{Ora di inizio}: 10:00;
		\item \textbf{Ora di fine}: 11:00.
	\end{itemize}

	\subsection{Presenze}
	\begin{itemize}
		\item \textbf{Presenti}:
		\begin{itemize}
			\item Alessandro Chimetto
			\item Alessandro Dindinelli
			\item Giacomo Bulbarelli
			\item Matteo Alba
			\item Paolo Scanferlato
			\item Valton Tahiraj

		\end{itemize}
		\item \textbf{Assenti}:
		\begin{itemize}
			\item Lucia Fenu

		\end{itemize}
		\item \textbf{Partecipanti esterni}:
		\begin{itemize}
			\item Alex Beggiato (referente, Sanmarco Informatica)
		\end{itemize}
	\end{itemize}

	\subsection{Ordine del giorno}
	\begin{enumerate}
		\item  Discussione sulla architettura e sulla realizzazione di un sistema \glock{zero downtime};
		\item  discussione sulla necessità di rilevare i pedoni nel sistema;
		\item  discussione sugli aggiornamenti in \glock{real-time};
		\item  varie ed eventuali.
	\end{enumerate}
	\newpage
	\section{Svolgimento}


	\subsection{Discussione sulla architettura}
	Abbiamo presentato una prima bozza dell'architettura del sistema per analizzare i relativi dubbi riscontrati. \\
	Ci si è poi focalizzato sullo \glock{zero downtime} e su quanto questo debba essere efficiente sulla base del contesto da noi selezionato. Essendo un requisito non obbligatorio, il proponente, ha consigliato una implementazione futura e che, il suo soddisfacimento per quanto piccolo esse sia; verrà comunque considerato come un plus.

	\subsection{Discussione rilevazione pedoni}
	I pedoni essendo, anch'essi un requisito non obbligatorio, la sua implementazione si può valutare successivamente. Per questo motivo, si deve partire dal fatto che il sistema a priori conosca la posizione di tutto ciò che succede all'interno della mappa e dunque, anche di tutti i pedoni.
	Il rilevamento dei pedoni da parte delle unità deve avvenire grazie al sistema, che comunica insieme all'uso dei sensori dell'unità, solo i pedoni a loro più prossimi.
	\\
	\\
	Un esempio utile che ci è stato fornito, è l'esempio di una fila:\\
	la fila è composta da pedoni relativamente fermi, equidistanti 1 metro (per via della pandemia). Il sistema sa che è presenta una fila in cui vi è la possibilità di far passare in mezzo ai pedoni le unità.

	\subsection{Discussione aggiornamenti real-time}
 	Si è deciso di utilizzare due istanze del motore di calcolo essendo in un contesto non critico.
 	l'aggiunta di una terza istanza può essere inserito come requisito migliorabile, qualora il sistema evolvesse e dunque ritenersi necessario.
	\newpage

	\section{Tabella delle decisioni}

	\begin{table} [h!]
		\rowcolors{2}{gray!25}{gray!6}
		\begin{center}
			\begin{tabular} { m{2cm} m{14cm} }
				\rowcolor{lightgray}
				\textbf{ID} & \textbf{Decisione}\\
				VE\_2021-03-31\_6.1 & Il sistema centrale deve conoscere la posizione di pedoni e ostacoli.\\
				VE\_2021-03-31\_6.2 & Vi devono essere due istanze del motore di calcolo per favorire l'aggiornamento.\\

			\end{tabular}
		\end{center}
	\end{table}
\end{document}

