L'interfaccia grafica è realizzata tramite il framework \glock{Angular}, quindi il design architetturale utilizzato è il \textit{Model View ViewModel} (\textit{MVVM}) che è intrinseco nel framework stesso. \\
La comunicazione con il server avviene tramite \glock{WebSocket} ed è dunque asincrona. 
Inoltre, per garantire l'estensibilità del codice, il servizio \textit{WebSocketService} implementa l'interfaccia \textit{ServerService} che mette a disposizione i metodi per interfacciarsi con il server. In questo modo un cambio di tecnologia, ad esempio passando a richieste \glock{HTTP}, non comporterebbe uno stravolgimento dell'architettura. \\
Per lo stesso motivo sono previste delle interfacce per rappresentare i vari tipi di messaggi che vengono ricevuti dal server, che vengono successivamente implementate per specificarne le caratteristiche. \\
Si è pianificato di avere un componente per ogni funzionalità principale che l'interfaccia mette a disposizione all'utente. L'aggiornamento dei dati tra gli \glock{Angular Components} ed il \glock{WebSocket} avviene tramite l'uso dei \glock{Subject}, indirizzando i dati stessi ai componenti che prevedono appositi campi dati. \\
\newline
Di seguito illustriamo due diagrammi delle classi: il primo per descrivere il legame tra \textit{Model} e \textit{Viewmodel}, con la struttura dei vari \glock{Angular Components}, mentre la parte di \textit{View} comprende i templates degli \glock{Angular Components}, e quindi la struttura del codice \glock{HTML}.
Il secondo invece rappresenta i messaggi che vengono inviati e ricevuti dal server.

\newpage

\begin{landscape}
	\begin{figure}[h!]
		\includegraphics[width=25.5cm]{img/ui_modelview.png}
		\caption{Architettura dell'interfaccia - Model-View Model}
	\end{figure}
\end{landscape}
\newpage

\begin{landscape}
	\begin{figure}[h!]
		\includegraphics[width=25.5cm]{img/ui_messaggi.png}
		\caption{Architettura dell'interfaccia - Messaggi}
	\end{figure}
\end{landscape}