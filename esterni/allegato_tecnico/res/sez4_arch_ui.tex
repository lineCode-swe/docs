L'interfaccia grafica è realizzata tramite il framework \glock{Angular}, quindi il design architetturale utilizzato è il \textit{Model View ViewModel} (\textit{MVVM}) che è intrinseco nel framework stesso. \\
La comunicazione con il server avviene tramite \glock{WebSocket} ed è quindi di natura asincrona. Viene sfruttato il design pattern \textit{Observer} per permettere un'aggiornamento asincrono dei componenti dell'interfaccia. Per garantire l'estensibilità del codice, il servizio \textit{WebSocketService} implementa l'interfaccia \textit{ServerService} che mette a disposizione i metodi per interfacciarsi con il server. In questo modo un cambio di tecnologia, ad esempio passando a richieste \glock{Http} non comporterebbe uno stravolgimento dell'architettura. \\
Per garantire che ogni componente riceva solo i dati di suo interesse vengono utilizzati delle classi definite da utente per trasmettere le informazioni tra i componenti. \\

referente msg particolari interfaccia


\newline
Di seguito l'architettura dell'interfaccia viene rappresentata tramite due diagrammi delle classi: il primo per descrivere il \textit{Model}, e quindi le modalità con cui i messaggi vengono inviati e ricevuti dal server, ed il secondo per il \textit{Viewmodel} che rappresenta la struttura dei vari \glock{Angular Components}. La parte di \textit{View} comprenderà i templates degli \glock{Angular Components}, e quindi la struttura del codice \glock{HTML}. \\
Si è pianificato di avere un componente per ogni funzionalità(O ALTRA PAROLA?) principale che l'interfaccia permette tipologia di dati che richiede.





%\begin{figure}[H]
%	\centering
%	\includegraphics[width=9cm]{images/arch_ui_model.png}
%	\caption{Errori ortografici per revisione}
%\end{figure}

%\begin{figure}[H]
%	\centering
%	\includegraphics[width=9cm]{images/arch_ui_viewmodel.png}
%	\caption{Errori ortografici per revisione}
%\end{figure}