\subsection{Descrizione generale}
Durante la fase di progettazione, il gruppo lineCode ha deciso di modellare l'architettura generale in tre moduli distinti:
\begin{itemize}
	\item \textbf{server}: motore di calcolo che coordina le unità e genera le informazioni per la UI. Viene modellato seguendo la \textit{layered architecture}.
	\item \textbf{interfaccia grafica}: mostra la mappa e permette di gestire le unità e gli utenti. Viene modellato seguendo la \textit{MVVM}.
	\item \textbf{unità}: si occupa di simulare il comportamento di un'unità all'interno dell'ambiente. Viene modellato seguendo la \textit{hexagonal architecture}. 
\end{itemize}
La UI e l'unità comunicano con il server tramite l'uso di \glock{WebSocket} e lo scambio di messaggi \glock{JSON}.
