Per poter assicurare una conduzione e uno sviluppo del progetto che soddisfino le scadenze, il gruppo ha deciso di ripartire il lasso di tempo che va dalla sua formazione fino alla revisione di accettazione nelle seguenti cinque attività:

\begin{itemize}
	\item analisi dei requisiti;
	\item consolidamento dei requisiti;
	\item progettazione della \glock{technology baseline};
	\item progettazione di dettaglio e codifica;
	\item validazione e collaudo.
\end{itemize}
Ogni attività verrà poi frammentata in altri sotto periodi, in ognuno dei quali verrà associata una \glock{milestone},
riferita alla data di fine periodo, per il completamento delle singole attività previste al suo interno.\\
In un intervallo di tempo vengono quindi inserite più attività, le quali, possono essere svolte sia sequenzialmente, sia con un certo grado di parallelismo in base alle dipendenze che sussistono tra di loro.

\subsection{Analisi dei requisiti}
L’attività di analisi dei requisiti ha inizio il giorno 31-10-2020, successivamente alla formazione dei gruppi, suddivisa in cinque periodi, con termine fissato per il giorno 11-01-2020, giorno di consegna dei documenti in ingresso alla revisione dei requisiti.

\subsubsection{Ruoli attivi}
Durante questa attività è necessaria la presenza dei seguenti ruoli:
\begin{itemize}
	\item Responsabile;
	\item amministratore;
	\item analista;
	\item verificatore.
\end{itemize}

\subsubsection{Periodi}
L'attività di analisi dei requisiti è stata suddivisa nei seguenti cinque periodi:

\paragraph{Primo periodo dal 31-10-2020 al 05-12-2020}
\begin{itemize}
	\item \textbf{Analisi dei capitolati}: studio individuale dei \glock{capitolati} e discussione interna al gruppo dei pregi e svantaggi individuati da ogni componente, in modo da indirizzare l’interesse del gruppo su certi \glock{capitolati} piuttosto che su altri;
	\item \textbf{ricerca}: individuazione e studio degli strumenti e delle tecnologie di supporto da utilizzare per la gestione del progetto;
	\item \textbf{studio di fattibilità}: impostato sulla base dell'analisi dei \glock{capitolati} fatta in precedenza;
	\item \textbf{pianificazione attività}: decisione dell'organizzazione interna al gruppo riguardo i ruoli da assegnare ed i compiti da svolgere.
\end{itemize}

\paragraph{Secondo periodo dal 05-12-2020 al 17-12-2020}
\begin{itemize}
	\item \textbf{Scelta del capitolato}: decisione definitiva riguardo il \glock{capitolato} scelto;
	\item \textbf{normazione}: scelta delle regole da adottare durante lo sviluppo del progetto riguardanti i processi primari e processi organizzativi;
	\item \textbf{studio di fattibilità}: fine dello \dext{Studio di fattibilità v1.0.0}, basato sul \glock{capitolato} scelto.
\end{itemize}

\paragraph{Terzo periodo dal 18-12-2020 al 29-12-2020}
\begin{itemize}
	\item \textbf{Analisi dei casi d'uso}: analisi del prodotto e dei casi d’uso;
	\item \textbf{norme di progetto}: stesura delle \dext{Norme di progetto v1.0.0};
	\item \textbf{piano di progetto}: stesura del \dext{Piano di progetto v1.0.0};
	\item \textbf{analisi dei rischi}: individuazione dei rischi che possono presentarsi nello svolgimento del progetto.
\end{itemize}

\paragraph{Quarto periodo dal 30-12-2020 al 07-01-2021}
\begin{itemize}
	\item \textbf{Analisi dei requisiti}: stesura dell’\dext{Analisi dei requisiti v1.0.0};
	\item \textbf{piano di qualifica}: stesura del \dext{Piano di qualifica v1.0.0};
	\item \textbf{glossario}: stesura del \dext{Glossario v1.0.0}.
\end{itemize}

\paragraph{Quinto periodo dal 08-01-2021 al 11-01-2021}
\begin{itemize}
	\item \textbf{Revisione}: ultimo controllo di tutti i documenti scritti;
	\item \textbf{presentazione RR}: presentazione della revisione dei requisiti.
\end{itemize}


\newpage

\begin{landscape}
	\begin{figure}[h!]
		\includegraphics[width=24cm]{images/1_Analisi_dei_requisiti.png}
		\caption{Pianificazione - Analisi dei requisiti}
	\end{figure}
\end{landscape}

\newpage

\subsection{Consolidamento dei Requisiti}
L'attività di consolidamento dei requisiti ha inizio il giorno 12-01-2021, successivamente alla consegna del materiale in ingresso alla revisione dei requisiti, suddivisa in due periodi, con termine fissato
per il giorno 17-01-2021 che precede la revisione dei requisiti del 18-01-2021.

\subsubsection{Ruoli attivi}
Durante questa attività è necessaria la presenza dei seguenti ruoli:
\begin{itemize}
	\item Responsabile;
	\item amministratore;
	\item analista.
\end{itemize}

\subsubsection{Periodi}
L'attività di consolidamento dei requisiti è stata suddivisa nei seguenti due periodi:

\paragraph{Primo periodo dal 12-01-2021 al 15-01-2021}
\begin{itemize}

	\item \textbf{Preparazione presentazione}: redazione della presentazione da portare in sede di revisione e studio individuale.

\end{itemize}

\paragraph{Secondo periodo dal 16-01-2021 al 17-01-2021}
\begin{itemize}

	\item \textbf{Analisi dei requisiti}: revisione ed eventuale aggiornamento dei requisiti.

\end{itemize}

\newpage

\begin{landscape}
	\begin{figure}[h!]
		\includegraphics[width=24cm]{images/2_Consolidamento_dei_requisiti.png}
		\caption{Pianificazione - Consolidamento dei requisiti}
	\end{figure}
\end{landscape}

\newpage
\subsection{Codifica della \glock{Technology Baseline}}
L'attività di codifica della \glock{technology baseline} ha inizio il giorno 01-02-2021, successivamente alla revisione dei requisiti, suddivisa in quattro periodi, con termine fissato per il giorno 14-03-2021 che precede la revisione di progettazione del 15-03-2021.

\subsubsection{Ruoli attivi}
Durante questa attività è necessaria la presenza dei seguenti ruoli:
\begin{itemize}
	\item Responsabile;
	\item amministratore;
	\item analista;
	\item progettista;
	\item programmatore;
	\item verificatore.
\end{itemize}

\subsubsection{Periodi}
L'attività di progettazione della \glock{technology baseline} è stata suddivisa nei seguenti periodi:

\paragraph{Primo periodo dal 01-02-2021 al 10-02-2021}
\begin{itemize}
	\item \textbf{Normazione}: revisione ed eventuale aggiornamento delle norme;
	\item \textbf{aggiornamento della pianificazione};
	\item \textbf{aggiornamento della qualità};
	\item \textbf{analisi dei requisiti}: revisione ed eventuale aggiornamento dei casi d’uso e dei requisiti, in base alle indicazioni ricevute;
	\item \textbf{ricerca}: studio autonomo degli strumenti e le tecnologie da utilizzare per lo sviluppo del
	progetto;
	\item \textbf{verifica}: controllo della qualità di tutti i prodotti sviluppati durante il periodo attuale.
\end{itemize}

\paragraph{Secondo periodo dal 19-02-2021 al 06-03-2021}
\begin{itemize}
	\item \textbf{Normazione}: aggiornamento delle norme;
	\item \textbf{progettazione}: progettazione del \glock{proof of concept} che deve essere implementato;
	\item \textbf{codifica}: implementazione del \glock{proof of concept} progettato;
	\item \textbf{verifica}: controllo della qualità di tutti i prodotti sviluppati durante il periodo attuale.
\end{itemize}

\paragraph{Terzo periodo dal 09-03-2021 al 14-03-2021}
\begin{itemize}
		\item \textbf{Stesura della lettera di presentazione}: scrittura della lettera di presentazione con la quale ci
	si candida alla revisione di progettazione;
		\item \textbf{aggiornamento documentazione};
	\item \textbf{verifica}:controllo della qualità di tutti i prodotti sviluppati durante il periodo attuale.
\end{itemize}

\paragraph{Quarto periodo dal 15-03-2021 al 20-03-2021}
\begin{itemize}
	\item \textbf{Preparazione presentazione}: redazione della presentazione da portare in sede di revisione e
	studio individuale.
\end{itemize}

\newpage
\begin{landscape}
	\begin{figure}[h!]
		\includegraphics[width=24cm]{images/tb.jpg}
		\caption{Pianificazione - Codifica \glock{Technology baseline}}
	\end{figure}
\end{landscape}
\newpage

\subsection{Progettazione}
La progettazione concentra sulla realizzazione dell'intera architettura. Ha inizio il giorno 22-03-2021, fino al 28-04-2021, in vista della \glock{Product Baseline}.
Il gruppo verrà suddiviso in tre sottogruppi:
\begin{itemize}
	\item Gruppo UI;
	\item gruppo Server;
	\item gruppo Unità.
\end{itemize} 
Ogni gruppo si autogestirà per portare a termine le funzionalità previste in ogni periodo.
L'unico ruolo che non verrà ricoperto durante l'intera attività, è il ruolo del programmatore.

\subsubsection{Periodi}

\begin{table} [h!]
	\begin{center}
		\rowcolors{2}{gray!25}{gray!6}
		\begin{tabular} { m{4cm}  m{11cm}  }	
			\multicolumn{2}{c}{	\textbf{Progettazione: Primo Periodo dal 22-03-2021 al 10-04-2021}} \\
			\rowcolor{lightgray}
			\textbf{Assegnatari} & \textbf{Funzionalità} \\
			Gruppo UI & \begin{itemize}
				\item Progettazione Model;
				\item Progettazione Component;
				\item Definizione Message To Server.
			\end{itemize}\\
			Gruppo Server & \begin{itemize}
				\item Progettazione Layers;
				\item Definizione Message From/To Unit;
				\item Definizione Message From/To UI.
			\end{itemize}\\
			Gruppo Unità & \begin{itemize}
				\item Progettazione UnitLogic;
				\item Progettazione Sensor;
				\item Definizione Message To Server.
			\end{itemize}\\
		\end{tabular}
		\caption{Progettazione: Primo Periodo}
	\end{center}
\end{table}


\begin{table} [h!]
	\begin{center}
		\rowcolors{2}{gray!25}{gray!6}
		\begin{tabular} { m{4cm}  m{11cm}  }	
			\multicolumn{2}{c}{	\textbf{Progettazione: Secondo Periodo dal 11-04-2021 al 12-04-2021}} \\
			\rowcolor{lightgray}
			\textbf{Assegnatari} & \textbf{Funzionalità} \\
			Ogni Gruppo & Stesura Allegato Tecnico.\\
		\end{tabular}
		\caption{Progettazione: Terzo Periodo}
	\end{center}
\end{table}


\begin{table} [h!]
	\begin{center}
		\rowcolors{2}{gray!25}{gray!6}
		\begin{tabular} { m{4cm}  m{11cm}  }	
			\multicolumn{2}{c}{	\textbf{Progettazione: Terzo Periodo dal 15-04-2021 al 25-04-2021}} \\
			\rowcolor{lightgray}
			\textbf{Assegnatari} & \textbf{Funzionalità} \\
			Gruppo UI & 
			Correzione e revisione sulla base delle correzioni nel Server.\\
			Gruppo Server & \begin{itemize}
				\item Eliminazione Singleton;
				\item Ricerca soluzioni e applicazione delle stesse.
			\end{itemize}\\
			Gruppo Unità & Cambio modello di progettazione.\\	
		\end{tabular}
		\caption{Progettazione: Terzo Periodo}
	\end{center}
\end{table}


\begin{table} [h!]
	\begin{center}
		\rowcolors{2}{gray!25}{gray!6}
		\begin{tabular} { m{4cm}  m{11cm}  }	
			\multicolumn{2}{c}{	\textbf{Progettazione: Quarto Periodo dal 26-04-2021 al 28-04-2021}} \\
			\rowcolor{lightgray}
			\textbf{Assegnatari} & \textbf{Funzionalità} \\
			Ogni Gruppo & \begin{itemize}
				\item Correzione Allegato Tecnico;
				\item Creazione slides.
			\end{itemize}\\		
		\end{tabular}
		\caption{Progettazione: Quarto Periodo}
	\end{center}
\end{table}

\newpage
\begin{landscape}
	\begin{figure}[h!]
		\includegraphics[width=24cm]{images/5_periodi}
		\caption{Pianificazione - Codifica \glock{Technology baseline}}
	\end{figure}
\end{landscape}

\clearpage
 \subsection{Incrementi}
 Di seguito la pianificazione si baserà sugli incrementi ognuno dei quali ha una data di inizio e fine.\\
  L'obiettivo di ogni incremento è creare nuove funzionalità e, la loro realizzazione, è affine alla progettazione realizzata e validata.
  Il gruppo verrà suddiviso nei tre sottogruppi precedente descritti.
  
  \begin{table} [h!]
  	\begin{center}
  		\rowcolors{2}{gray!25}{gray!6}
  		\begin{tabular} { m{4cm}  m{11cm}  }	
  			\multicolumn{2}{c}{	\textbf{Incremento I: dal 10-05-2021 al 16-05-2021}} \\
  			\rowcolor{lightgray}
  			\textbf{Assegnatari} & \textbf{Funzionalità} \\
  			Gruppo UI & Codifica Model\\
  			Gruppo Server & Codifica Business layer\\
  			Gruppo Unità & Codifica dello strato di persistenza\\	
  		\end{tabular}
  		\caption{Incremento I}
  	\end{center}
  \end{table}

 \begin{table} [h!]
 	\begin{center}
 		\rowcolors{2}{gray!25}{gray!6}
 		\begin{tabular} { m{4cm}  m{11cm}  }	
 			\multicolumn{2}{c}{	\textbf{Incremento II: dal 17-05-2021 al 23-05-2021}} \\
 			\rowcolor{lightgray}
 			\textbf{Assegnatari} & \textbf{Funzionalità} \\
 			Gruppo UI & \begin{itemize}
 				\item Codifica \glock{WebSocket};
 				\item Creazione interfaccia login/logout;
 				\item Realizzazione dei permessi per ogni utente.
 			\end{itemize}\\		
 			Gruppo Server & \begin{itemize}
 				\item Codifica Persistance layer;
 				\item Creazione Database.
 			\end{itemize}\\		
 			Gruppo Unità & \begin{itemize}
 				\item Codifica Business logic;
 				\item Codifica UnitEngine.
 			\end{itemize}\\		
 		\end{tabular}
 		\caption{Incremento II}
 	\end{center}
 \end{table}

 \begin{table} [h!]
	\begin{center}
		\rowcolors{2}{gray!25}{gray!6}
		\begin{tabular} { m{4cm}  m{11cm}  }	
			\multicolumn{2}{c}{	\textbf{Incremento III: dal 24-05-2021 al 30-05-2021}} \\
			\rowcolor{lightgray}
			\textbf{Assegnatari} & \textbf{Funzionalità} \\
			Gruppo UI & \begin{itemize}
				\item Creazione interfaccia delle tabelle per l'inserimento di unità e utenti;
				\item Codifica MapComponent.
			\end{itemize}\\		
			Gruppo Server & Codifica API\\
			Gruppo Unità & Codifica Presentation layer per il server centrale
		
		\end{tabular}
		\caption{Incremento III}
	\end{center}
\end{table}


\begin{table} [h!]
	\begin{center}
		\rowcolors{2}{gray!25}{gray!6}
		\begin{tabular} { m{4cm}  m{11cm}  }	
			\multicolumn{2}{c}{	\textbf{Incremento IV: dal 31-05-2021 al 06-06-2021}} \\
			\rowcolor{lightgray}
			\textbf{Assegnatari} & \textbf{Funzionalità} \\
			Gruppo UI & Implementazione Messagge To/From Server\\	
			Gruppo Server & \begin{itemize}
				\item Implementazione Messagge To/From UI;
				\item Implementazione Messagge To/From Unit.
			\end{itemize}\\		
			Gruppo Unità & Test sulle componenti\\
			
		\end{tabular}
		\caption{Incremento IV}
	\end{center}
\end{table}

\begin{table} [h!]
	\begin{center}
		\rowcolors{2}{gray!25}{gray!6}
		\begin{tabular} { m{4cm}  m{11cm}  }	
			\multicolumn{2}{c}{	\textbf{Incremento V: dal 21-06-2021 al 28-06-2021}} \\
			\rowcolor{lightgray}
			\textbf{Assegnatari} & \textbf{Funzionalità} \\
				Ogni Gruppo & \begin{itemize}
				\item Prima stesura Manuale Utente;
				\item Prima stesura Manuale Manutentore;
				\item Aggiornamento della documentazione.
			\end{itemize}\\		
			
		\end{tabular}
		\caption{Incremento V}
	\end{center}
\end{table}

\newpage
\begin{landscape}
	\begin{figure}[h!]
		\includegraphics[width=24cm]{images/6_incrementi}
		\caption{Pianificazione - Codifica \glock{Technology baseline}}
	\end{figure}
\end{landscape}

\clearpage
\subsection{Validazione e collaudo}
L'attività di validazione e collaudo ha inizio il giorno 19-07-2021, successivamente alla revisione di qualifica, formato da un unico periodo, con termine fissato per il giorno 30-07-2021.

\subsubsection{Ruoli attivi}
Durante questa attività è necessaria la presenza dei seguenti ruoli:
\begin{itemize}
	\item Responsabile;
	\item amministratore;
	\item progettista;
	\item programmatore;
	\item verificatore.
\end{itemize}
\subsubsection{Periodi}
L'attività di validazione e collaudo è stata suddivisa nei seguenti periodi:
\paragraph{Primo periodo dal 19-07-2021 al 30-07-2021}
\begin{itemize}
	\item \textbf{Normazione}: revisione ed eventuale aggiornamento delle norme;
	\item \textbf{aggiornamento} della pianificazione;
	\item \textbf{aggiornamento della qualità};
	\item \textbf{completamento progettazione};
	\item \textbf{verifica}: controllo della qualità di tutti i prodotti sviluppati durante il periodo attuale;
	\item \textbf{preparazione presentazione}: redazione della presentazione da portare in sede di revisione e
	studio individuale.
\end{itemize}