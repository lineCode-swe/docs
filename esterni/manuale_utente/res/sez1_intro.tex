\subsection{Scopo del documento}
	Il documento vuole essere una guida per illustrare tutte le funzionalità del progetto \glock{PORTACS}. In questo modo l'utente finale potrà effettuare un corretto uso del \glock{software}.

\subsection{Scopo del prodotto}
	Il \glock{capitolato} C5 ha come obbiettivo la realizzazione di un applicativo \glock{Real-Time} in grado di guidare delle unità dotate di mobilità autonoma in ambienti specifici, partendo dal presupposto che queste si muovano in ambienti in cui sono presenti altre unità (autonome o meno).
	

\subsection{Glossario e documenti esterni}
	In supporto alla documentazione viene fornito un glossario per chiarire, con una definizione, eventuali termini specifici contenuti in questo documento.
	Saranno adottati quindi questi due simboli a pedice:
	\begin{itemize}
		\item \textit{D}: indica un documento specifico;
		\item \textit{G}: indica un termine incluso nel \dext{Glossario v3.0.0}.
	\end{itemize}
