\usepackage[italian]{babel}
\usepackage[margin=20mm, footskip = 20pt]{geometry}
\usepackage{array}
\usepackage{tabularx}
\usepackage{graphicx}
\usepackage{subfiles}
\usepackage{hyperref}
\usepackage{nameref}
\usepackage{titlesec}
\usepackage{longtable}
\usepackage[table]{xcolor}
\usepackage{titling}
\usepackage{lastpage}
\usepackage{ifthen}
\usepackage{calc}
\usepackage{soulutf8}
\usepackage{contour}
\usepackage{float}
\usepackage{fancyhdr}
\usepackage{multirow}
\usepackage{pgfgantt}
\usepackage{pdflscape}
\usepackage{eurosym}
\usepackage{amsmath}
\usepackage{ifthen}


\newcommand{\hr}{\par\vspace{-.1\ht\strutbox}\noindent\hrulefill\par}

\graphicspath{ {./}
	{./commons/res}
}

%--------------------------------------------------
% Comandi per inserire contenuto del documento
%--------------------------------------------------
\makeatletter

\newcommand\appendToGraphicsPath[1]{%
	\g@addto@macro\Ginput@path{{#1}}%
}

\newcommand{\setTitle}[1]{%
	\newcommand{\@phTitle}{#1}%
}
\newcommand{\phTitle}{\@phTitle}

\newcommand{\setDate}[1]{%
	\newcommand{\@phDate}{#1}%
}
\newcommand{\phDate}{\@phDate}

\newcommand{\setUso}[1]{%
	\newcommand{\@uso}{#1}%
}
\newcommand{\uso}{\@uso}

\newcommand{\setVersione}[1]{%
	\newcommand{\@versione}{#1}%
}
\newcommand{\versione}{\@versione}

\newcommand{\disabilitaVersione}{%
	\renewcommand{\setVersione}[1]{}%
	\renewcommand{\versione}{DISABILITATA}
}

\newcommand{\setResponsabile}[1]{%
	\newcommand{\@responsabile}{#1}%
}
\newcommand{\responsabile}{\@responsabile}

\newcommand{\setRedattori}[1]{%
	\newcommand{\@redattori}{#1}%
}
\newcommand{\redattori}{\@redattori}

\newcommand{\setVerificatori}[1]{%
	\newcommand{\@verificatori}{#1}%
}
\newcommand{\verificatori}{\@verificatori}

\newcommand{\setModifiche}[1]{%
	\newcommand{\@modifiche}{#1}%
}
\newcommand{\modifiche}{\@modifiche}

\makeatother

%--------------------------------------------------
% Comandi per i documenti esterni e il glossario
%--------------------------------------------------

\newcommand{\dext}[1]{\textsc{#1\textsubscript{\textit{D}}}}

\newcommand{\glock}[1]{\textsc{#1\textsubscript{\textit{G}}}}

%--------------------------------------------------
% Comandi per impostare sottotitoli di quarto e quinto livello
%--------------------------------------------------

\setcounter{secnumdepth}{4}
\setcounter{tocdepth}{4}

\titleformat{\paragraph}
{\normalfont\normalsize\bfseries}{\theparagraph}{1em}{}
\titlespacing*{\paragraph}{0pt}{2.25ex plus 1ex minus .2ex}{1.5ex plus .2ex}

\titleformat{\subparagraph}
{\normalfont\normalsize\bfseries}{\thesubparagraph}{1em}{}
\titlespacing*{\subparagraph}{0pt}{1.75ex plus 1ex minus .2ex}{.75ex plus .1ex}

%--------------------------------------------------
% Comando per togliere i box rossi dai link del table of contents
%--------------------------------------------------

\hypersetup{pdfborder = {0 0 0}}