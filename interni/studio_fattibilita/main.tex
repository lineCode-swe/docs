\documentclass[]{article}

\usepackage[italian]{babel}
\usepackage[margin=20mm, footskip = 20pt]{geometry}
\usepackage{array}
\usepackage{tabularx}
\usepackage{graphicx}
\usepackage{subfiles}
\usepackage{hyperref}
\usepackage{nameref}
\usepackage{titlesec}
\usepackage{longtable}
\usepackage[table]{xcolor}
\usepackage{titling}
\usepackage{lastpage}
\usepackage{ifthen}
\usepackage{calc}
\usepackage{soulutf8}
\usepackage{contour}
\usepackage{float}
\usepackage{fancyhdr}
\usepackage{multirow}
\usepackage{pgfgantt}
\usepackage{lscape}

\newcommand{\hr}{\par\vspace{-.1\ht\strutbox}\noindent\hrulefill\par}

\graphicspath{ {./}
	{./commons/res}
}

%--------------------------------------------------
% Comandi per inserire contenuto del documento
%--------------------------------------------------
\makeatletter

\newcommand\appendToGraphicsPath[1]{%
	\g@addto@macro\Ginput@path{{#1}}%
}

\newcommand{\setTitle}[1]{%
	\newcommand{\@phTitle}{#1}%
}
\newcommand{\phTitle}{\@phTitle}

\newcommand{\setDate}[1]{%
	\newcommand{\@phDate}{#1}%
}
\newcommand{\phDate}{\@phDate}

\newcommand{\setUso}[1]{%
	\newcommand{\@uso}{#1}%
}
\newcommand{\uso}{\@uso}

\newcommand{\setVersione}[1]{%
	\newcommand{\@versione}{#1}%
}
\newcommand{\versione}{\@versione}

\newcommand{\disabilitaVersione}{%
	\renewcommand{\setVersione}[1]{}%
	\renewcommand{\versione}{DISABILITATA}
}

\newcommand{\setResponsabile}[1]{%
	\newcommand{\@responsabile}{#1}%
}
\newcommand{\responsabile}{\@responsabile}

\newcommand{\setRedattori}[1]{%
	\newcommand{\@redattori}{#1}%
}
\newcommand{\redattori}{\@redattori}

\newcommand{\setVerificatori}[1]{%
	\newcommand{\@verificatori}{#1}%
}
\newcommand{\verificatori}{\@verificatori}

\newcommand{\setModifiche}[1]{%
	\newcommand{\@modifiche}{#1}%
}
\newcommand{\modifiche}{\@modifiche}

\makeatother 

%--------------------------------------------------
% Comandi per i documenti esterni e il glossario
%--------------------------------------------------

\newcommand{\dext}[1]{\textsc{#1\textsubscript{\textit{D}}}}

\newcommand{\glock}[1]{\textsc{#1\textsubscript{\textit{G}}}}

%--------------------------------------------------
% Comandi per impostare sottotitoli di quarto e quinto livello
%--------------------------------------------------

\setcounter{secnumdepth}{4}
\setcounter{tocdepth}{4}

\titleformat{\paragraph}
{\normalfont\normalsize\bfseries}{\theparagraph}{1em}{}
\titlespacing*{\paragraph}{0pt}{2.25ex plus 1ex minus .2ex}{1.5ex plus .2ex}

\titleformat{\subparagraph}
{\normalfont\normalsize\bfseries}{\thesubparagraph}{1em}{}
\titlespacing*{\subparagraph}{0pt}{1.75ex plus 1ex minus .2ex}{.75ex plus .1ex}

\appendToGraphicsPath{../../commons/res/}

%------------------------------
%
% COMANDI DI CONFIGURAZIONE
%
%------------------------------

\setTitle{Studio di Fattibilità}

\setVersione{0.1.0}

\setDate{12-12-2020}

\setResponsabile{Paolo Scanferlato}

\setRedattori{Alessandro Chimetto}

\setVerificatori{Giacomo Bulbarelli}

\setUso{Interno}

\setModifiche{
%%	Vers.	&	Nome				&	Ruolo		&	Data		&	Desrizione					\\%
%	M1		&	M2					&	M3			&	M3			&	M4							\\%
	0.1.1	&	Lucia Fenu	&	Redattore	&	14-12-2020	&	Stesura iniziale + studio C6 \\
	0.1.0	&	Alessandro Chimetto	&	Redattore	&	12-12-2020	&	Stesura iniziale + studio C1}
\begin{document}

	% Direttive per la creazione del titolo tramite comando maketitle
\title{\huge \textsc{\phTitle{}} \\
	\vspace{11pt} \large \textsc{\phDate{}}}

\author{} % Non toccare
\date{} % Non toccare

%--------------------
% Frontespizio
%--------------------

% Logo del gruppo
\begin{figure}[t!]
	\centering
	\includegraphics[width=20em]{lclong}
\end{figure}

% Titolo / Nome
\maketitle
\thispagestyle{empty}

% Dati specifici sul doc in forma tabulare
\begin{table}[ht]
	\begin{center}
		\label{tab:Dati sul documento}
		\begin{tabular}{r|l}
			\multicolumn{2}{c}{ \textsc{Dati sul documento} } \\
			\hline
			\textbf{Versione} & \versione{} \\
			\textbf{Uso} & \uso{}  \\
			\textbf{Redattori} & \redattori{} \\
			\textbf{Verificatori} & \verificatori{} \\
			\textbf{Responsabile} & \responsabile{} \\
			\textbf{Destinatari} & lineCode \\
								& prof.\ Vardanega Tullio \\		
								& prof.\ Cardin Riccardo \\
			\ifthenelse{\equal{\uso}{Esterno}}{
								& Sanmarco Informatica
			}{} \\
		\end{tabular}
	\end{center}
\end{table}

\newpage

\renewcommand{\arraystretch}{2} % allarga le righe con dello spazio sotto e sopra
\begin{longtable}[H]{>{\centering\bfseries}m{2cm} >{\centering}m{3.5cm} >{\centering}m{2.5cm} >{\centering}m{3cm} >{\centering\arraybackslash}m{5cm}}
	\rowcolor{lightgray}
	{\textbf{Versione}} & {\textbf{Nominativo}} & {\textbf{Ruolo}} & {\textbf{Data}} & {\textbf{Descrizione}}  \\
	\endfirsthead%
	\rowcolor{lightgray}
	{\textbf{Versione}} & {\textbf{Nominativo}}  & {\textbf{Ruolo}} & {\textbf{Data}} & {\textbf{Descrizione}}  \\
	\endhead%
	\modifiche{}%
\end{longtable}

	\newpage

	%--------------------------------
	%
	% IL CONTENUTO INIZIA DA QUI
	%
	%--------------------------------

	%! INDICE

	\section{Introduzione}
		\subsection{Scopo del documento}
		Questo documento riporta lo studio svolto su ogni \glock{capitolato} proposto,
		enunciando per ognuno: finalità, tecnologie e caratteristiche ritenute
		negative o positive dai membri del gruppo.

		\subsection{Altri Documenti e Glossario}
		Terminologie specifiche all'interno del documento saranno scritte in \textsc{maiuscoletto} per evitare ambiguità e avranno a pedice:
		\begin{itemize}
			\item \textit{D} se indicano un documento specifico;
			\item \textit{G} se incluse nel \dext{glossario}.
		\end{itemize}

		\subsection{Riferimenti}
			\subsubsection{Normativi}
			\begin{itemize}
				\item \textbf{\dext{Norme di Progetto}}
			\end{itemize}

			\subsubsection{Informativi}
			\begin{itemize}
				\item \textbf{\glock{Capitolato} C1 - BlockCOVID: supporto digitale al contrasto della pandemia} \\
				(https://www.math.unipd.it/~tullio/IS-1/2020/Progetto/C1.pdf);

				\item \textbf{\glock{Capitolato} C2 - EmporioLambda: piattaforma di e-commerce in stile Serverless} \\
				(https://www.math.unipd.it/~tullio/IS-1/2020/Progetto/C2.pdf);

				\item \textbf{\glock{Capitolato} C3 - GDP: Gathering Detection Platform} \\ (https://www.math.unipd.it/~tullio/IS-1/2020/Progetto/C3.pdf);

				\item \textbf{\glock{Capitolato} C4 - HD Viz: visualizzazione di dati multidimensionali} \\
				(https://www.math.unipd.it/~tullio/IS-1/2020/Progetto/C4.pdf);

				\item \textbf{\glock{Capitolato} C5 - PORTACS: piattaforma di controllo mobilità autonoma} \\
				(https://www.math.unipd.it/~tullio/IS-1/2020/Progetto/C5.pdf);

				\item \textbf{\glock{Capitolato} C6 - RGP: Realtime Gaming Platform} \\
				(https://drive.google.com/file/d/1MQ8j4plXMsKjBfLPrUfHCv\_6pfnjcU5T/view?usp=sharing);

				\item \textbf{\glock{Capitolato} C7 - SSD: soluzioni di sincronizzazione desktop} \\
				(https://www.math.unipd.it/~tullio/IS-1/2020/Progetto/C7.pdf)
			\end{itemize}

	\newpage

	\section{Valutazione capitolato scelto}
		\subsection{C5 - PORTACS}
		%! AGGIUNGERE CAPITOLATO C5

	\newpage

	\section{Valutazione capitolati non scelti}
		\subsection{C1 - BlockCOVID}
			\subsubsection{Informazioni generali}
			\begin{itemize}
				\item \textbf{Titolo:} BlockCOVID;
				\item \textbf{Sottotitolo:} Supporto digitale al contrasto della pandemia;
				\item \textbf{Proponente:} Imola Informatica;
				\item \textbf{Committente:} Prof. Tullio Vardanega e Prof. Riccardo Cardin.
			\end{itemize}

			\subsubsection{Descrizione}
			Vista la corrente pandemia di SARS-CoV-2 e conseguenti accordi fra sindacati e imprese, si vuole creare un'applicazione in grado di tracciare la presenza del personale di un organizzazione (es. scuola o azienda) nelle proprie postazioni di lavoro e di tracciare l'igienizzazione di quest'ultime da parte dell'utente o un'azienda specializzata.

			\subsubsection{Finalità}
			Si vuole creare un'applicazione mobile che permetta all'utente di:
			\begin{itemize}
				\item segnalare, tramite tag \glock{RFID}, l'occupazione di una postazione di lavoro ad un server dedicato;
				\item prenotare una postazione libera ed igienizzata;
				\item segnalare se la postazione è stata pulita autonomamente alla fine dell'utilizzo.
			\end{itemize}
			L'azienda esterna specializzata deve poter utilizzare l'applicazione per:
			\begin{itemize}
				\item ricevere una lista delle postazioni da igienizzare;
				\item marcare la stanza come igienizzata.
			\end{itemize}
			Il server dedicato deve tenere traccia delle informazioni in una struttura dati di tipo \glock{blockchain} e deve mettere a disposizione dell'amministratore:
			\begin{itemize}
				\item un'interfaccia che consenta di creare, modificare o eliminare postazioni e stanze;
				\item la gestione del calendario delle prenotazioni;
				\item la possibilità di eseguire ricerche e creare report su personale addetto e igienizzazioni.
			\end{itemize}

			\subsubsection{Tecnologie}
			Parte dell'interesse di ricerca dell'azienda è l'esplorazione di nuove soluzioni tecnologiche dunque quasi tutte le tecnologie da essa proposte non sono obbligatorie ma solamente consigliate sulla base dell'esperienza:
			\begin{itemize}
				\item \textsc{\glock{API Rest} o \glock{gRPC}:} tecnologie di comunicazione asincrona fra applicazione e server;
				\item \textsc{\glock{Ethereum}:} tecnologia \glock{blockchain} per immagazzinare dati con opponibilità a terzi;
				\item \textsc{\glock{Kubernetes}, \glock{Openshift} o \glock{Rancher}:} orchestratori per gestire \glock{deploy} e scalabilità delle componenti server.
			\end{itemize}

			\subsubsection{Linguaggi di Programmazione}
			\begin{itemize}
				\item \textsc{\glock{Java}, \glock{Python} o \glock{Node.js}:} linguaggi molto diffusi, consigliati (ma non obbligatori) per lo sviluppo delle componenti \glock{back-end};
				\item \textsc{\glock{Kotlin} o \glock{Swift}:} linguaggi per lo sviluppo di applicazioni mobile rispettivamente su piattaforme \glock{Android} e \glock{iOS};
				\item \textsc{\glock{Solidity} o \glock{Vyper}:} linguaggi per lo sviluppo di \glock{smart contract} per la piattaforma \glock{Ethereum}.
			\end{itemize}

			\subsubsection{Vincoli di Progetto}
			\begin{itemize}
				\item Per la comunicazione asincrona fra applicazione e server, si richiede di utilizzare \glock{API Rest} o in alternativa \glock{gRPC};
				\item scansione dei codici nel tempo sufficiente a certificare la presenza della persona in postazione;
				\item presentazione di un resoconto su scelte e test effettuati per garantire una buona durata della batteria del dispositivo mobile nonostante l'utilizzo del lettore \glock{RFID};
			\end{itemize}

			\subsubsection{Aspetti positivi}
			\begin{itemize}
				\item Si sta affrontando un tema di grande attualità e la sua realizzazione, potenzialmente, implicherebbe un contributo anche a livello sociale;
				\item il \glock{capitolato} è fortemente politematico in quanto spazia dall'uso di hardware specializzato all'uso di linguaggi di programmazione attuali e strumenti rilevanti nell'ambito \glock{DevOps} per non parlare dell'utilizzo della struttura dati \glock{blockchain}; tutto ciò rende il lavoro proposto molto stimolante.
			\end{itemize}

			\subsubsection{Criticità}
			\begin{itemize}
				\item Il fatto che questo \glock{capitolato} sia estremamente politematico implica che la mole di lavoro nel progetto da esso derivato sia imponente. Ci sono molte tecnologie nuove da imparare in poco tempo e al fine di soddisfare i soli requisiti obbligatori si richiede lo sviluppo di due applicazioni di più che modeste dimensioni (mobile e server) che implementino tutte le tecnologie;
				\item non è chiaro se l'hardware specifico per tag \glock{RFID} venga fornito dall'azienda e in che quantità; ciò potrebbe diventare un problema nel momento in cui le unità a disposizione fossero poche in quanto poche persone ne avrebbero accesso e, vista la mancanza di contatti causa pandemia, non ci sarebbe la possibilità di scambiarle con altri membri del gruppo.
			\end{itemize}

			\subsubsection{Conclusioni}
			Nonostante il questo \glock{capitolato} abbia suscitato un forte interesse in tutti i membri del gruppo per questioni sia tecnologiche che sociali, si teme che la mole di lavoro sia troppo ampia e di conseguenza che non si riesca a consegnare il prodotto in tempi accettabili per il gruppo.

		\newpage

		\subsection{C2 - EmporioLambda}
		%! AGGIUNGERE RESTANTI CAPITOLATI E CONCLUSIONE FINALE
		
		\subsection{C6 - Realtime Gaming Platform }
		\subsubsection{Informazioni generali}
		\begin{itemize}
			\item \textbf{Titolo:} Realtime Gaming Platform;
			\item \textbf{Proponente:} Zero12;
			\item \textbf{Committente:} Prof. Tullio Vardanega e Prof. Riccardo Cardin.
		\end{itemize}
		
		\subsubsection{Descrizione}
		Sviluppo e progettazione di un videogioco a scorrimento verticale fruibili su dispositivi mobile con la possibilità di giocare anche in multiplayer oltre alla campagna in singleplayer.
		
		\subsubsection{Finalità}
		Il fulcro è la progettazione di un sistema multiplayer “fantasma” in cui i giocatori non possono interagire tra loro, ma solo controllare in real-time le mosse dell’avversario senza poter intervenire.
		La modalità singleplayer sarà di tipo infinito, con difficoltà crescente, fino a che il giocatore non avrà esaurito le vite o i power-up in suo possesso.
		Al termine della progettazione, dovranno essere forniti i seguenti materiali:
		\begin{itemize}
			\item Report/Retrospettiva sulle tecnologie utilizzate per capire se la scelta iniziale è stata corretta;
			\item Configurazione dell’architettura cloud ed eventualmente il codice sorgente tramite GIT (se il servizio lo prevede);
			\item codice sorgente dell’app tramite repository GIT.
		\end{itemize}
		
		\subsubsection{Tecnologie}
		Le tecnologie consigliate dall’azienda riguardano principalmente AWS (Amazon Web Services),  che fornisce servizi di cloud computing; dando libertà di scelta sull’uso dei servizi offerti da tale tecnologia, motivandone poi la scelta.
		Su consiglio:
		\begin{itemize}
			\item \glock{AWS Gamelift}: servizio specializzato in giochi multiplayer;
			\item \glock{AWS Appsync}: tecnologia che permette l'utilizzo del linguaggio \glock{GraphQL} in modo semplice e sicuro.
			
		\end{itemize}
		
		\subsubsection{Linguaggi di Programmazione}
		\begin{itemize}
			\item \glock{Node.js}: linguaggio per lo sviluppo delle componenti \glock{back-end};
			\item \glock{Kotlin}: linguaggio per lo sviluppo di applicazioni mobile per piattaforme \glock{Android} (richiesta minima Android 8);
			\item \glock{Swift}: linguaggio per lo sviluppo di applicazioni mobile per piattaforme \glock{iOS}(richiesta minima iOS 13);
			\item \glock{GraphQL}: linguaggio di manipolazione e query di dati Open Source per \glock{API} e un runtime per la realizzazione di query con dati esistenti.
		\end{itemize}
		
		\subsubsection{Vincoli di Progetto}
		\begin{itemize}
			\item Svolgere un'analisi preliminare sulle tecnologie AWS, motivandone la scelta;
			\item L'architettura server dovrà essere scalabile, che sarà una delle caratteristiche che influenzeranno la scelta dell'AWS;
			
			
		\end{itemize}
		
		\subsubsection{Aspetti positivi}
		\begin{itemize}
			\item La libertà che l'azienda offre sulla scelta delle tecnologie da utilizzare;
			\item La disponibilità nello studio preliminare dei diversi servizi cloud;
			\item Repository GIT offerta dalla azienda per lo sviluppo del progetto.
			
		\end{itemize}
		
		\subsubsection{Criticità}
		\begin{itemize}
			\item Non avendo esperienza sulle tecnologie offerte non è chiara la mole di lavoro che si affronterà, in particolare sullo studio individuale;
			\item Troppa libertà di scelta potrebbe essere difficile da gestire se non si hanno le competenze preliminari e quindi richiederebbe lavoro aggiuntivo per la conoscenza di un numero abbastanza sufficiente per la scelta finale da utilizzare al fine di un ragionato sviluppo nell'ambiente di lavoro.
		\end{itemize}
		
		\subsubsection{Conclusioni}
		Per quanto l'argomento videogiochi sia sempre ben gradito all'interno del gruppo, il team non ha le conoscenze adatte per soddisfare i requisiti nei tempi stabiliti, in particolare per la libera scelta che il proponente offre.
		Riteniamo comunque che sia una buona proposta, ma non per le nostre competenze generali attuali difficilmente sanabili in tempo utile.		
		\newpage
\end{document}

