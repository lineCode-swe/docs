\subsection{Verifica}
		\subsubsection{Descrizione}
		Il processo di \textbf{Verifica} mira a garantire la correttezza del prodotto stabilendo dei criteri precisi con cui giudicarlo. \\ Esso verrà garantito:
		\begin{itemize}
			\item definendo criteri di accettazione;
			\item prescrivendo attività di verifica (accompagnate dalla relativa documentazione);
			\item eseguendo dei test di verifica;
			\item correggendo potenziali difetti riscontrati.
		\end{itemize}
		La verifica ha lo scopo di individuare e correggere potenziali difetti nel prodotto, e precede l'attuazione del processo di validazione.
		
		\subsubsection{Attività}
		Il processo di verifica prevede due attività principali:
		\begin{itemize}
			\item \textbf{Analisi statica};
			\item \textbf{Analisi dinamica}.
		\end{itemize}
		
		Si rende necessaria una classificazione dei test previsti dall'attività di Analisi dinamica. Di seguito verrà esposta la nomenclatura relativa ai suddetti test che verranno effettuati sul prodotto.
		
		\subsubsection{Test}
		
		Per rappresentare le specifiche dei test si è deciso di utilizzare una forma tabellare che contiene:
		\begin{itemize}
			\item codice identificativo del componente da testare;
			\item descrizione del test;
			\item stato di avanzamento, identificato nel modo seguente:
				\begin{itemize}
					\item \textbf{NI}: non implementato;
					\item \textbf{I}: implementato;						
				\end{itemize}		
			\item risultato del test, identificato nel modo seguente:
				\begin{itemize}
					\item \textbf{S}: superato;
					\item \textbf{F}: fallito.
				\end{itemize}
		\end{itemize}	
		\textbf{N.B.}: Il parametro relativo al risultato viene momentaneamente omesso in quanto superfluo.						 
					
		
		\paragraph{Test di sistema}
		I test di sistema vengono classificati nella maniera seguente:
		\begin{center}
			\textbf{TS[Priorità]-[Categoria]-[Codice]}
		\end{center}		 
		dove:\\
		\begin{itemize}
			\item \textbf{Priorità}: indica la priorità del requisito associato al test, e può assumere valori quali:
			\begin{itemize}
				\item \textbf{M}: mandatory/obbligatorio;
				\item \textbf{D}: desirable/desiderabile;
				\item \textbf{O}: optional/ opzionale, relativamente utile ed eventualmente trascurabile.
			\end{itemize}
			\item \textbf{Categoria}: indica la tipologia di requisito associato al test, e può assumere valori quali:
			\begin{itemize}
				\item \textbf{F}: functional/funzionale;
				\item \textbf{P}: performance/prestazionale;
				\item \textbf{Q}: qualitative/qualitativo;
				\item \textbf{C}: constraint/vincolo.
			\end{itemize}
			\item \textbf{Codice}: intero positivo che identifica il singolo componente da testare. Ha valore di default 1.
		\end{itemize}
		 
		\paragraph{Test di integrazione}
		I test di integrazione vengono classificati nella maniera seguente:
		\begin{center}
			\textbf{TI[Codice]}
		\end{center}	
		dove:\\
		\begin{itemize}
			\item \textbf{Codice}: intero positivo che identifica il singolo componente da testare. Ha valore di default 1.
		\end{itemize}		 
		
		\paragraph{Test di unità}
		I Test di unità vengono classificati nella maniera seguente:
		\begin{center}
			\textbf{TU[Codice]}
		\end{center}		
		dove:\\
		\begin{itemize}
			\item \textbf{Codice}: intero positivo che identifica il singolo componente da testare. Ha valore di default 1.
		\end{itemize}
		
		\paragraph{Test di regressione}
		I Test di regressione consistono nell'esecuzione di tutti i test già effettuati in precedenza sulle unità in relazione con quella che ha subito una modifica. Non risulta dunque necessaria una nomenclatura specifica per essi.
		
		\subsubsection{Strumenti}
		Si utilizzano le funzionalità di correzione automatica e di controllo ortografico degli \glock{IDE} utilizzati dal gruppo, che sono:
		\begin{itemize}
			\item \glock{TeXStudio};
			\item \glock{Texmaker}.
		\end{itemize}
