		\section{Validazione}
		\subsection{Scopo}
		Il processo di \textbf{Validazione} mira a garantire che il prodotto rispetta le specifiche del committente. Esso consiste nell'esecuzione di test che producono dei risultati da confrontare con dei valori attesi.
		
		\subsection{Descrizione}
		A seguito dell'esecuzione dei test su ciascuna delle componenti del prodotto, è possibile stabilire se il prodotto rispetta le specifiche, dipendentemente dal fatto che i risultati ottenuti corrispondano a quelli attesi o meno. 
	
		\subsection{Attività}
		\subsubsection{Test}
		Per rappresentare le specifiche dei \glock{test} si è deciso di utilizzare una forma tabellare che contiene:
				\begin{itemize}
					\item Codice identificativo del componente da testare
					\item Descrizione del test
					\item Stato di avanzamento, identificato nel modo seguente:
						\begin{itemize}
							\item \textbf{NI}: non implementato;
							\item \textbf{I}: implementato.						
						\end{itemize}		
					\item Risultato del test, identificato nel modo seguente:
						\begin{itemize}
							\item \textbf{S}: superato;
							\item \textbf{F}: fallito.
						\end{itemize}
						\textbf{N.B.}: il parametro relativo al risultato viene momentaneamente omesso in quanto superfluo.						 
				\end{itemize}		
		
		\subsubsection{Test di accettazione}
		I \glock{Test di accettazione} vengono classificati nella maniera seguente:
		\begin{center}
			\textbf{TA[Priorità]-[Categoria]-[Codice]}
		\end{center}		 
		dove:\\
		\begin{itemize}
			\item \textbf{Priorità}: indica la priorità del requisito associato al test, e può assumere valori quali:
			\begin{itemize}
				\item \textbf{M}: mandatory/obbligatorio;
				\item \textbf{D}: desirable/desiderabile;
				\item \textbf{O}: optional/opzionale, relativamente utile ed eventualmente trascurabile.
			\end{itemize}
			\item \textbf{Categoria}: indica la tipologia di requisito associato al test, e può assumere valori quali:
			\begin{itemize}
				\item \textbf{F}: functional/funzionale;
				\item \textbf{P}: performance/prestazionale;
				\item \textbf{Q}: qualitative/qualitativo;
				\item \textbf{C}: constraint/vincolo.
			\end{itemize}
			\item \textbf{Codice}: intero positivo che identifica il singolo componente da testare. Ha valore di default 1.
		\end{itemize}