\subsection{Garanzia della qualità}
					
	\subsubsection{Scopo}
	Effettuare delle scrupolose classificazioni e stabilire metriche precise nell'ambito della verifica e della validazione mantenendo un livello di qualità che rimanga uniforme e quantificabile durante l'intero ciclo di vita del software.
					
	\subsubsection{Descrizione}
	La garanzia della qualità comprende diverse tipologie di controlli che devono essere effettuati su:
	\begin{itemize}
		\item software;
		\item documentazione;
		\item processi atti a produrre il software e la documentazione.						
	\end{itemize}	
				
	\subsubsection{Obiettivi di qualità del prodotto}	
	Per garantire una buona qualità del prodotto si attuano dei processi di verifica e validazione basati su fondamenti normativi:
	\begin{itemize}
		\item \textbf{verifica}: processo di controllo che assicura la qualità dei processi di fornitura del prodotto;
		\item \textbf{validazione}: processo di controllo del prodotto predisposto a garantire le aspettative, i requisiti e le funzionalità decise.
	\end{itemize}
	Tutti questi processi devono portare ad una migliore qualità del prodotto sottoposto agli standard del \textit{way of working}.
					
	\subsubsection{Obiettivi di qualità del processo}
	La qualità del prodotto deve soddisfare attraverso il ciclo di vita del software i principi di efficacia ed efficienza mirati al prodotto:
	\begin{itemize}
		\item \textbf{efficacia}: il prodotto deve soddisfare le richieste del proponente;
		\item \textbf{efficienza}: i processi devono mantenere costi ridotti in termini di risorse a parità di qualità del prodotto.
	\end{itemize}
	Durante tutta la produzione del prodotto i processi vanno migliorati attraverso dei monitoraggi che permettono di ottenere, attraverso l'esperienza, una risposta critica relativa alla qualità stessa del prodotto.
					
	\subsubsection{Conformità del contratto}
	Deve essere garantito che l'acquirente e le altre parti ricevano il supporto e la cooperazione richiesta, in conformità con il contratto.
					
	\subsubsection{Personale assegnato}
	Si deve garantire che il personale assegnato abbia le capacità e le conoscenze necessarie per soddisfare il requisiti del progetto e ricevere la formazione necessaria.
					
	\subsubsection{Garanzia del prodotto}
	Deve essere assicurato che tutti i termini richiesti dal contratto siano documentati, conformi a suddetto contratto, reciprocamente coerenti e vengano eseguiti come richiesto. E' quindi fondamentale che il contenuto dei suddetti documenti rifletta in maniera chiara gli estremi concordati.
					
	\subsubsection{Garanzia del processo}
	Deve essere garantito che i processi durante il ciclo di vita del software (fornitura, sviluppo, funzionamento, manutenzione, processi di supporto e garanzia della qualità) impiegati per il progetto siano conformi con il contratto e aderiscano a quanto indicato all'interno di esso.
					
	\subsubsection{Classificazione delle metriche}
	Per indicare la qualità di processo e prodotto raggiunte sono state utilizzate delle metriche, seguendo la notazione \textbf{M[id][int]}, dove:
	\begin{itemize}					
		\item \textbf{M}: è l’abbreviazione di metrica;
		\item \textbf{id}: è il codice identificativo associato alla tipologia di metrica e può assumere i seguenti valori:
		\begin{itemize}
			\item \textbf{P}: indica una metrica di processo;
			\item \textbf{D}: indica una metrica di documento;
			\item \textbf{S}: indica una metrica del prodotto software;
			\item \textbf{T}: indica una metrica di test.
		\end{itemize}
		\item \textbf{int}: rappresenta un numero incrementale a due cifre a partire da 01.
	\end{itemize}