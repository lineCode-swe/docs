\subsection{Gestione Processi}

	\subsubsection{Scopo}
	L'utilizzo di una gestione dei processi ha lo scopo di:
	\begin{itemize}
		\item identificare e gestire i rischi;
		\item pianificare singole attività in base a determinate scadenze temporali;
		\item definire un modello di lavoro da seguire;
		\item calcolare un prospetto economico in base ai ruoli ed al monte ore;
		\item calcolare il bilancio finale.
	\end{itemize}
	Queste attività sono riportate in modo dettagliato nel Documento Piano di Fattibilità, elaborato dal responsabile di progetto in collaborazione con l'amministratore.
	
	\subsubsection{Aspettative}
	Gli obiettivi prefissati da tale gestione sono i seguenti:
	\begin{itemize}
		\item adattare le attività in base ai rischi identificati, permettendo di limitare i danni qualora si presentassero;
		\item gestire i membri del team in base alle singole attività, creando un lavoro in parallelo, tenendo conto dei ruoli assegnati;
		\item migliorare la comunicazione all'interno del team;
		\item tenere sotto controllo l'andamento, al fine di ottimizzare l'efficienza del lavoro.
	\end{itemize}

	\subsubsection{Descrizione}
	Le attività previste secondo il Piano di Progetto sono:
	\begin{itemize}
		\item analisi e classificazione dei rischi;
		\item pianificazione dei processi principali;
		\item assegnazione dei ruoli e dei processi individuati;
		\item revisione periodica delle attività;
		\item stima dei costi, tempi e risorse;
		\item calcolo del bilancio.	 
	\end{itemize}

	\subsubsection{Ruoli di progetto}
	Tutti i membri del gruppo dovranno assumere diversi ruoli a rotazione e, per ognuno di essi, saranno previsti compiti specializzati descritti all'interno del Piano di Progetto. Di seguito sono riportate, in linea generale, gli aspetti caratterizzanti di ogni ruolo che verrà ricoperto.
	
		\paragraph{Responsabile di progetto}
		Il responsabile di progetto ha la responsabilità sulla pianificazione, il controllo e la gestione delle attività del gruppo. Essendo la figura principale all'interno del team, si interessa anche dell'amministrazione esterna, come la comunicazione e corrispondenza con il committente e/o proponente del capitolato.\\
		Riassumendo, si occupa di:
		\begin{itemize}
			\item gestione e controllo delle attività;
			\item coordinamento dei compiti in base ai ruoli assunti dai singoli membri;
			\item approvare i documenti finali e le offerte proposte;
			\item gestire i rischi in base al Piano di Progetto;
			\item redigere il Piano di Progetto.
		\end{itemize}
		
		\paragraph{Amministratore}
		L'amministratore ha il compito di supportare e garantire il controllo dell'ambiente di lavoro. \\
		In generale, si occupa di:	
		\begin{itemize}
			\item coordinare l'ambiente di lavoro con adeguato supporto;
			\item gestire la documentazione;
			\item controllare versioni e configurazioni;
			\item risolvere i problemi legati alla gestione;
			\item redigere le Norme di Progetto.
		\end{itemize}
	
		\paragraph{Analista}
		L'analista si occupa dell'analisi dei rischi e della loro classificazione. Per questo motivo, non sarà presente per l'intera realizzazione del progetto.
		Si occupa di:
		\begin{itemize}
			\item studio del dominio;
			\item definizione dei requisiti necessari;
			\item definizione dei rischi;
			\item redigere l'Analisi dei Requisiti e Studio di Fattibilità.
		\end{itemize}
		
		\paragraph{Progettista}
		Il progettista ha il compito di gestire tutte le attività in campo tecnico e tecnologico e deve:
		\begin{itemize}
			\item compiere delle decisioni su aspetti tecnici e tecnologici;
			\item definire e sviluppare l'architettura che verrà sviluppata, in modo stabile e mantenibile;
			\item redigere il Piano di Qualifica, la Specifica Tecnica e la Definizione del Prodotto.	
		\end{itemize}

		\paragraph{Programmatore}
		Il programmatore è il responsabile della codifica di tutte le attività atte allo sviluppo del progetto. \\
		Ha il compito di:
		\begin{itemize}
			\item implementare le specifiche del progettista 
			\item gestire le componenti di sviluppo, verifica e mantenimento del prodotto.
		\end{itemize}
			
		\paragraph{Verificatore}
		Il verificatore controlla il lavoro svolto dagli altri membri, ma è esonerato dalla correzione di eventuali errori. \\
		Il suo compito dunque, è quello di:
		
		\begin{itemize}
			\item controllare i lavori in fase di revisione seguendo le Norme di Progetto;
			\item rilevare e comunicare gli errori.
		\end{itemize}
		
		 \subsubsection{Gestione delle comunicazioni}
		 
		 \paragraph{Comunicazioni interne}
		 Le comunicazioni interne al gruppo vengono effettuate attraverso l'utilizzo di \glock{Discord}, un software che permette la divisione della chat e delle chiamate in base agli argomenti; tramite l'uso di canali personalizzabili manualmente. \\ Permette l'uso di bot con cui è possibile ad esempio notificare le \glock{pull-request} o la ricezione di nuove email sul profilo aziendale lineCode. \\
		 Il  \glock{workspace} è suddiviso nel seguente modo:
		 	\begin{itemize}
		 	\item {\ttfamily generale}: per comunicazioni di bassa importanza;
		 	\item {\ttfamily files}: per inviare i documenti e/o link;
		 	\item {\ttfamily riferimenti}: contenitore di indirizzi e riferimenti importanti;
		 	\item {\ttfamily pull-request}: contenitore delle \glock{pull-request} e dei suoi verificatori.
		 	\end{itemize}
	 
		 \paragraph{Comunicazioni esterne}
		 Le comunicazioni esterne sono a cura del responsabile di progetto. \\
		 Per la corrispondenza è stato creato un indirizzo di posta elettronica \glock{Gmail}: \url{linecode@gmial.com}, reso accessibile a tutti i membri del gruppo.\\
		 Il proponente del capitolato ha  permesso l'uso dei seguenti indirizzi e-mail:
		 \begin{itemize}
		 	\item \url{alex.beggiato@sanmarcoinformatica.it}
		 	\item \url{alessandra.piva@sanmarcoinformatica.it}
		 \end{itemize}
		 
		 \subsubsection{Gestione degli incontri}
		 Gli incontri, sia interni che esterni, sono esclusivamente online causa SARS-CoV-2.
		 \paragraph{Incontri interni}
		 Si utilizza il canale vocale di \glock{Discord}, suddiviso nel seguente modo:
		 
		 \begin{itemize}
		 	\item {\sffamily Generale}: per svolgere le riunioni;
		 	\item {\sffamily Ot}: nel caso in cui durante una riunione fosse necessario dividersi per non disturbare il lavoro degli altri membri.
	 	\end{itemize}
 		Le riunioni vengono stabilite di volta in volta durante gli incontri online. 
 		Nel caso ci fossero degli imprevisti si utilizza il canale testuale per avvisare tutti i membri di un eventuale ritardo da parte di un componente, e/o modifica dell'orario previsto assicurandosi che tutti diano conferma.
		 \paragraph{Incontri Esterni}
		 Il proponente ha espresso libertà nell'uso dei software a nostra disposizione per effettuare videochiamate, previo accordo tramite corrispondenza e-mail.
		 
		 \paragraph{Verbali}
		 All'inizio di ogni riunione si decide chi ha il compito di trascrivere degli appunti sulle questioni e decisioni sollevate durante l'incontro. \\
		 La trascrizione sarà a rotazione. \\
     
		 Dopodiché si trattano gli argomenti del giorno che sono stati tracciati alla riunione precedente o durante comunicazioni interne. 
		 
		 \subsubsection{Strumenti di coordinamento}
		 Per la gestione della attività è previsto l'uso delle \glock{Issue Tracking System} fornito da \glock{GitHub}. \\
		 Le \glock{Issue} create vengono automatizzate tramite il \glock{Project}, template \glock{Automated Kanban}, che permette una visualizzazione più pratica sul tracciamento delle attività. \\
		 Ogni \glock{Issue} ha le seguenti informazioni:
		 \begin{itemize}
		 	\item {\bfseries Titolo}: nome del compito da eseguire;
		 	\item {\bfseries Descrizione}: una descrizione breve e concisa su come svolgere il compito;
		 	\item {\bfseries Scadenza}: tramite l'uso di \glock{Milestone}, si definisce la scadenza secondo la quale si dovrebbe terminare il compito.
		 \end{itemize}
		 La lavagna è divisa nel seguente modo:
		 \begin{itemize}
		 	\item {\bfseries to do}: compiti ancora svolgere;
		 	\item {\bfseries in progress}: compiti in fase di svolgimento;
		 	\item {\bfseries done}: compiti svolti.	
		 	\end{itemize}
	 	L'\glock{Issue Tracking System} in quanto completamente integrato con il \glock{Repository} del progetto, è semplice da usare e da gestire.
		 \subsubsection{Gestione dei rischi}
		 I rischi che verranno rilevati e che saranno classificati, sono descritti dettagliatamente nel Piano di Progetto che sarà sempre aggiornato su:
		 \begin{itemize}
		 	\item nuovi rischi individuati;
		 	\item ridefinizione, all'occorrenza, di nuove strategie sulla gestione dei rischi;
		 	\item monitoraggio dei rischi pervenuti. \\ 
		 \end{itemize}
	 	\noindent
	 	I rischi sono così suddivisi:
	 	\begin{itemize}
	 		\item {\bfseries RS-O}: rischio Organizzativo;
	 		\item {\bfseries RS-T}: rischio Tecnologico;
	 		\item {\bfseries RS-P}: rischio di carattere Privato/Personale. \\
	 	\end{itemize}
	 	
	 \subsubsection{Metriche di Qualità}
	 Durante la realizzazione del processo verranno utilizzate le seguenti metriche di qualità sulla gestione ed il tracciamento di processi e rischi; al fine di ottenere efficacia ed efficienza.  
	 \paragraph{Schedule Variance - SV}
	 La Schedule Variance (SV) indica quanto si è in linea, in anticipo o in ritardo rispetto alla schedulazione delle attività di progetto pianificate. \\
	 È un indicatore di efficacia soprattutto nei confronti del cliente.\\
	 Il valore è espresso in percentuale, secondo la formula:
	\begin{displaymath}
		SV = \frac{BCWP-BCWS}{BCWP}\times100
	\end{displaymath}
	Dove:

		\begin{itemize}
			\item {\bfseries BCWP}: {\itshape Budgeted Cost of Work Performed} o {\itshape
			Earned Value}, è  il valore (in giorni o EURO) delle attività realizzate alla data corrente.\\
			Il suo valore è una stima approssimata, secondo i seguenti procedimenti:
			\begin{displaymath}
				BCWP_t = \sum BCWP_s
			\end{displaymath}
			dove:
			\begin{itemize}
				\item {\bfseries $BCWP_t$}: è il BCWP totale del progetto;
				\item {\bfseries $BCWP_s$}: è il BCWP della singola attività, secondo la formula:
				\begin{displaymath}
					BCWP_s = CA\times BAC
				\end{displaymath}
			dove:
			\begin{itemize}
				\item  CA: {\itshape Completamento Attività}, espresso in percentuale;
				\item BCA: {\itshape Budget at Completion}, è il valore inizialmente previsto per la realizzazione del progetto.
			\end{itemize}
			\end{itemize}
			\item {\bfseries BCWS}: {\itshape Budgeted Cost of Work Scheduled}, è il costo pianificato (in giorni o EURO) per realizzare le attività di progetto alla data corrente. \\
		\end{itemize} 
		
		{\bfseries Sul risultato:}
		\begin{itemize}
		\item {\bfseries Positivo}: indica che si è avanti rispetto alla schedulazione;
		\item {\bfseries Negativo}: indica che si è indietro rispetto alla schedulazione.\\
		\end{itemize}
	 \paragraph{Cost Variance - CV}
	 La Cost variance, indica se il valore del costo realmente maturato è maggiore, uguale o minore rispetto al costo effettivo; ovvero quanto sia il livello di efficienza del gruppo rispetto a quanto pianificato. \\
	 Viene usato come indicatore di produttività o efficienza soprattutto nei confronti del {\itshape Management} dell'azienda. \\
	 Il suo valore è espresso in percentuale, secondo la seguente formula:
	 \begin{displaymath}
	 	CV = \frac{BCWP - ACWP}{BCWP}\times100
	 \end{displaymath}
 		dove:
 		\begin{itemize}
 			\item {\bfseries BCWP}: {\itshape Budgeted Cost of Work Performed} o {\itshape
 				Earned Value}, è il valore (in giorni o EURO) delle attività realizzate alla data corrente;
 			\item {\bfseries ACWP}: {\itshape Actual Cost of Work Performed},è il costo effettivamente sostenuto (in giorni o EURO) alla data corrente.
 		\end{itemize}
 		La differenza tra BCWP e ACWP è che se l'attuale costo (ACWP) alla data corrente è più alto rispetto a quello guadagnato (BCWP) alla data corrente; sappiamo che l'appaltatore sta attualmente superando i costi previsti nella stima al completamento, sforando il budget.
 		\\ \\
 		{\bfseries Sul risultato}:
 		\begin{itemize}
 			\item {\bfseries Positivo}: indica che il progetto viene sviluppato con un costo minore rispetto a quanto preventivato;
 			\item {\bfseries Negativo}:  indica che il progetto viene sviluppato con un costo maggiore rispetto a quanto preventivato;
 			\item {\bfseries Uguale a zero}: indica che il progetto viene sviluppato con un costo in linea rispetto a quanto preventivato;
 		\end{itemize} 
	
	 \paragraph{Unbudgeted Risks - UR}
	 L'Unbudgeted Risks, indica il numero di rischi non rilevati in fase di analisi, in modo incrementale. Per ogni rischio non preventivato che viene rilevato, si incrementa di una unità (partendo dallo zero) il numero di rischi rilevati dall'inizio del progetto fino a data corrente; secondo la formula:
	 \begin{displaymath}
	 	UR = UR + 1 
	 \end{displaymath}
 	{\bfseries Sul risultato}:
 	\begin{itemize}
 		\item {\bfseries pari a zero}: non sono stati rilevati rischi in fase di progetto;
 		\item {\bfseries maggiore di zero}: sono stati trovati rischi in fase di progetto.\\
 	\end{itemize}
	 \paragraph{Defects Removal Efficiency - DRE}
	 La Defects Removal Efficiency è una metrica utile sia a livello di processo che di progetto. Indica la misura della capacità del team di rimuovere i difetti prima del rilascio, secondo la formula:
	 \begin{displaymath}
	 DRE = \frac{DI}{DI+DE}\times100
	 \end{displaymath}
 	dove:
 	\begin{itemize}
 		\item {\bfseries DI}: numero dei difetti interni, rilevati dai test in fase di sviluppo;
 		\item {\bfseries DE}: numero di difetti esterni, rilevati dal consumatore.\\
 		
 	\end{itemize}
 
	 
		 
		 \subsubsection{Strumenti}
		 Per permettere l'attuazione delle procedure sopra citate, sono utilizzati i seguenti strumenti:
		 \begin{itemize}
		 	\item {\bfseries \glock{Discord}}: strumento per comunicazioni e incontri interni al gruppo;
		 	\item {\bfseries \glock{GitHub}}: strumento pe il versionamento remoto dei file condivisibili a tutti i membri del gruppo;
		 	\item {\bfseries \glock{GitKraken}}: strumento collegato alla \glock{Repository} \glock{GitHub}, per gestire in modo più semplice ed intuitivo i branch, commit e pull request;
		 	\item {\bfseries \glock{GMail}}: servizio di posta elettronica per le comunicazioni esterne;
		 	\item {\bfseries \glock{Issue Tracking System}}: strumento fornito da \glock{GitHub} per la gestione e il tracciamento dei compiti assegnati.
		 \end{itemize}
		