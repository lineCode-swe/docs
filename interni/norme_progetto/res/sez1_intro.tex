\subsection{Scopo del documento}
Il documento ha lo scopo di definire le guidelines del way of working adottato dal team lineCode. Le attività presenti in questo documento sono redatte da processi contenuti nello standard ISO/IEC 12207:1995. Risulta quindi necessario che tutti i membri del gruppo prendano visione di questo documento ai fini di coesione e uniformità all'interno del progetto.

\subsection{Scopo del Prodotto}
Il \glock{capitolato} C5 ha come obbiettivo la realizzazione di un applicativo \glock{Real-Time} in grado di guidare delle unità dotate di mobilità autonoma in ambienti specifici, partendo dal presupposto che queste si muovano in ambienti in cui sono presenti altre unità (autonome o meno).

\subsection{Glossario e documenti esterni}
In supporto alla documentazione viene fornito un glossario per chiarire, con una definizione, eventuali termini specifici contenuti in questo documento.
Saranno adottati quindi questi due simboli a pedice:
\begin{itemize}
	\item \textit{D} se indicano un documento specifico;
	\item \textit{G} se incluse nel \dext{glossario}.
\end{itemize}

\subsection{Riferimenti}
	\subsubsection{Riferimenti Normativi}
	\begin{itemize}
		\item \textbf{{\glock{capitolato} C5 - PORTACS}}: \url{https://www.math.unipd.it/~tullio/IS-1/2020/Progetto/C5.pdf};
	\end{itemize}
	\subsubsection{Rifermenti informativi}
	\begin{itemize}
		\item \textbf{ISO/IEC 12207:1995}: \url{https://www.math.unipd.it/~tullio/IS-1/2009/Approfondimenti/ISO_12207-1995.pdf};
		\item \textbf{Documentazione git}: \url{https://git-scm.com/doc};
		\item \textbf{Gitflow}: \url{http://nvie.com/posts/a-successful-git-branching-model/};
		\item \textbf{Documentazione GitHub}: \url{https://docs.github.com};
		\item \textbf{Documentazione Zapier}: \url{https://zapier.com/help};
		\item \textbf{Documentazione act}: \url{https://github.com/nektos/act/blob/master/README.md};
		\item \textbf{Studio di Fattibilità}: ; %! da completare
		\item \textbf{Piano di Qualifica}: ;
		\item \textbf{Piano di Progetto}: .
	\end{itemize}