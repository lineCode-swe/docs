\documentclass[]{article}

\usepackage[italian]{babel}
\usepackage[margin=20mm, footskip = 20pt]{geometry}
\usepackage{array}
\usepackage{tabularx}
\usepackage{graphicx}
\usepackage{subfiles}
\usepackage{hyperref}
\usepackage{nameref}
\usepackage{titlesec}
\usepackage{longtable}
\usepackage[table]{xcolor}
\usepackage{titling}
\usepackage{lastpage}
\usepackage{ifthen}
\usepackage{calc}
\usepackage{soulutf8}
\usepackage{contour}
\usepackage{float}
\usepackage{fancyhdr}
\usepackage{multirow}
\usepackage{pgfgantt}
\usepackage{lscape}

\newcommand{\hr}{\par\vspace{-.1\ht\strutbox}\noindent\hrulefill\par}

\graphicspath{ {./}
	{./commons/res}
}

%--------------------------------------------------
% Comandi per inserire contenuto del documento
%--------------------------------------------------
\makeatletter

\newcommand\appendToGraphicsPath[1]{%
	\g@addto@macro\Ginput@path{{#1}}%
}

\newcommand{\setTitle}[1]{%
	\newcommand{\@phTitle}{#1}%
}
\newcommand{\phTitle}{\@phTitle}

\newcommand{\setDate}[1]{%
	\newcommand{\@phDate}{#1}%
}
\newcommand{\phDate}{\@phDate}

\newcommand{\setUso}[1]{%
	\newcommand{\@uso}{#1}%
}
\newcommand{\uso}{\@uso}

\newcommand{\setVersione}[1]{%
	\newcommand{\@versione}{#1}%
}
\newcommand{\versione}{\@versione}

\newcommand{\disabilitaVersione}{%
	\renewcommand{\setVersione}[1]{}%
	\renewcommand{\versione}{DISABILITATA}
}

\newcommand{\setResponsabile}[1]{%
	\newcommand{\@responsabile}{#1}%
}
\newcommand{\responsabile}{\@responsabile}

\newcommand{\setRedattori}[1]{%
	\newcommand{\@redattori}{#1}%
}
\newcommand{\redattori}{\@redattori}

\newcommand{\setVerificatori}[1]{%
	\newcommand{\@verificatori}{#1}%
}
\newcommand{\verificatori}{\@verificatori}

\newcommand{\setModifiche}[1]{%
	\newcommand{\@modifiche}{#1}%
}
\newcommand{\modifiche}{\@modifiche}

\makeatother 

%--------------------------------------------------
% Comandi per i documenti esterni e il glossario
%--------------------------------------------------

\newcommand{\dext}[1]{\textsc{#1\textsubscript{\textit{D}}}}

\newcommand{\glock}[1]{\textsc{#1\textsubscript{\textit{G}}}}

%--------------------------------------------------
% Comandi per impostare sottotitoli di quarto e quinto livello
%--------------------------------------------------

\setcounter{secnumdepth}{4}
\setcounter{tocdepth}{4}

\titleformat{\paragraph}
{\normalfont\normalsize\bfseries}{\theparagraph}{1em}{}
\titlespacing*{\paragraph}{0pt}{2.25ex plus 1ex minus .2ex}{1.5ex plus .2ex}

\titleformat{\subparagraph}
{\normalfont\normalsize\bfseries}{\thesubparagraph}{1em}{}
\titlespacing*{\subparagraph}{0pt}{1.75ex plus 1ex minus .2ex}{.75ex plus .1ex}

\appendToGraphicsPath{../../commons/res/}

%------------------------------
%
% COMANDI DI CONFIGURAZIONE
%
%------------------------------

\setTitle{Norme di Progetto}

\setVersione{0.2.0}

\setDate{20-12-2020}

\setResponsabile{Paolo Scanferlato}

\setRedattori{Valton Tahiraj}

\setVerificatori{Alessandro Chimetto}

\setUso{Interno}

\setModifiche{
	0.2.0 & Valton Tahiraj 			& Redattore 	& 20-12-2020 & Aggiunto Processo di fornitura \\ 
	0.1.0 & Alessandro Chimetto		& Verificatore	& 15-12-2020 & Verificata Introduzione\\
	0.1.0 & Valton Tahiraj 			& Redattore 	& 14-12-2020 & Aggiunto Introduzione \\
	0.0.0 & Alessandro Chimetto		& Verificatore	& 14-12-2020 & Verifica prima stesura\\
	0.0.0 & Valton Tahiraj 			& Redattore 	& 14-12-2020 & Prima stesura
}

\begin{document}
	% Direttive per la creazione del titolo tramite comando maketitle
\title{\huge \textsc{\phTitle{}} \\
	\vspace{11pt} \large \textsc{\phDate{}}}

\author{} % Non toccare
\date{} % Non toccare

%--------------------
% Frontespizio
%--------------------

% Logo del gruppo
\begin{figure}[t!]
	\centering
	\includegraphics[width=20em]{lclong}
\end{figure}

% Titolo / Nome
\maketitle
\thispagestyle{empty}

% Dati specifici sul doc in forma tabulare
\begin{table}[ht]
	\begin{center}
		\label{tab:Dati sul documento}
		\begin{tabular}{r|l}
			\multicolumn{2}{c}{ \textsc{Dati sul documento} } \\
			\hline
			\textbf{Versione} & \versione{} \\
			\textbf{Uso} & \uso{}  \\
			\textbf{Redattori} & \redattori{} \\
			\textbf{Verificatori} & \verificatori{} \\
			\textbf{Responsabile} & \responsabile{} \\
			\textbf{Destinatari} & lineCode \\
								& prof.\ Vardanega Tullio \\		
								& prof.\ Cardin Riccardo \\
			\ifthenelse{\equal{\uso}{Esterno}}{
								& Sanmarco Informatica
			}{} \\
		\end{tabular}
	\end{center}
\end{table}

\newpage

\renewcommand{\arraystretch}{2} % allarga le righe con dello spazio sotto e sopra
\begin{longtable}[H]{>{\centering\bfseries}m{2cm} >{\centering}m{3.5cm} >{\centering}m{2.5cm} >{\centering}m{3cm} >{\centering\arraybackslash}m{5cm}}
	\rowcolor{lightgray}
	{\textbf{Versione}} & {\textbf{Nominativo}} & {\textbf{Ruolo}} & {\textbf{Data}} & {\textbf{Descrizione}}  \\
	\endfirsthead%
	\rowcolor{lightgray}
	{\textbf{Versione}} & {\textbf{Nominativo}}  & {\textbf{Ruolo}} & {\textbf{Data}} & {\textbf{Descrizione}}  \\
	\endhead%
	\modifiche{}%
\end{longtable}
	\newpage
	%--------------------------------
	%
	% IL CONTENUTO INIZIA DA QUI
	%
	%--------------------------------
	\section{Introduzione}
		\subsection{Scopo del documento}
		Il documento ha lo scopo di definire le guidelines del way of working adottato dal team lineCode. Le attività presenti in questo documento sono redatte da processi contenuti nello standard ISO/IEC 12207:1995. Risulta quindi necessario che tutti i membri del gruppo prendano visione di questo documento ai fini di coesione e uniformità all'interno del progetto.

		\subsection{Scopo del Prodotto}
		Il \glock{capitolato} C5 ha come obbiettivo la realizzazione di un applicativo \glock{Real-Time} in grado di guidare delle unità dotate di mobilità autonoma in ambienti specifici, partendo dal presupposto che queste si muovano in ambienti in cui sono presenti altre unità (autonome o meno).

		\subsection{Glossario e documenti esterni}
		In supporto alla documentazione viene fornito un glossario per chiarire, con una definizione, eventuali termini specifici contenuti in questo documento.
		Saranno adottati quindi questi due simboli a pedice:
		\begin{itemize}
			\item \textit{D} se indicano un documento specifico;
			\item \textit{G} se incluse nel \dext{glossario}.
		\end{itemize}

		\subsection{Riferimenti}
			\subsubsection{Riferimenti Normativi}
			\begin{itemize}
				\item \textbf{{\glock{capitolato} C5 - PORTACS}}: \url{https://www.math.unipd.it/~tullio/IS-1/2020/Progetto/C5.pdf};
				\item \textbf{{ISO 8601:1988 (Data and Time Formats)}}: \url {https://www.w3.org/TR/NOTE-datetime}. %! formati al momento non implementati
			\end{itemize}
			\subsubsection{Rifermenti informativi}
			\begin{itemize}
				\item \textbf{ISO/IEC 12207:1995}: \url{https://www.math.unipd.it/~tullio/IS-1/2009/Approfondimenti/ISO_12207-1995.pdf}
				\item \textbf{Studio di Fattibilità}: ; %! da completare
				\item \textbf{Piano di Qualifica}: ;
				\item \textbf{Piano di Progetto}: .
			\end{itemize}
		
		
			\newpage
			
			\section{Processi primari}
			
				\subsection{Fornitura}
				
					\subsubsection{Obbiettivi}
					L'obbiettivo di questa sezione è descrivere le norme che il gruppo lineCode si impegna a rispettare per potersi proporre come fornitore nei confronti di Sanmarco Informatica SpA e dei committenti Prof. Tullio Vardanega e Prof. Riccardo Cardin per la progettazione,sviluppo e consegna del progetto \glock{portacs}
					
					\subsubsection{Attività}
						\paragraph{Studio di fattibilità}
					    	Documento che riporta lo studio svolto dagli analisti sui \glock{capitolati} proposti. Per ciascun \glock{capitolato} viene riportato:
						 \begin{itemize}
						 	\item \textbf{Descrizione:} sintesi del prodotto richiesto da sviluppare presentato nel \glock{capitolato}; 
						 	\item \textbf{Finalità:} ambito di utilizzo del prodotto da sviluppare;
						 	\item \textbf{Tecnologie interessate:} lista di tutte le tecnologie interessate nello sviluppo del prodotto;
						 	\item \textbf{Analisi motivazione, criticità e rischi:} racchiude le motivazioni, le criticità e i rischi risultati dall'analisi del \glock{capitolato}; 
						 	\item \textbf{Valutazione finale:} indica le motivazioni per le quali il \glock{capitolato} è stato respinto o accettato.
						 \end{itemize}
						  
							
						\paragraph{Documentazione esterna}
						 Il gruppo lineCode si impegna a fornire al proponente Sanmarco Informatica SpA e ai committenti Prof. Tullio Vardanega e Prof. Riccardo Cardin i seguenti documenti:
						 \begin{itemize}
						 	\item \textbf{Piano di progetto:} documento che descrive le metodologie di pianificazione, consegna e completamento del progetto;
						 	\item \textbf{Piano di qualifica:} documento che contiene le attività di verifica,validazione e garantisce la qualità dei processi e di prodotto;
						 	\item \textbf{Analisi dei requisiti:} documento contenente l'analisi dei requisiti e dei casi d'uso del gruppo.
						 	contenuto...
						 \end{itemize}
					
					\subsubsection{Procedure}
						Il gruppo intende instaurare un dialogo costante ed un profondo rapporto di collaborazione con il proponente Sanmarco Informatica SpA, al fine di:
						\begin{itemize}
							\item determinare i bisogni del proponente;
							\item stabilire vincoli e  requisiti dei processi;
							\item stabilire vincoli e requisiti del prodotto;
							\item stimare tempistiche e costi del lavoro;
							\item accordarsi sulla qualifica di prodotto;
							\item chiarire eventuali dubbi emersi.
							
						\end{itemize}
						
						
					
					\subsubsection{Strumenti di Supporto}
						%! Da chiarire
			
		



\end{document}

