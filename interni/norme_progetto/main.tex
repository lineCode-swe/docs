\documentclass[]{article}

\usepackage[italian]{babel}
\usepackage[margin=20mm, footskip = 20pt]{geometry}
\usepackage{array}
\usepackage{tabularx}
\usepackage{graphicx}
\usepackage{subfiles}
\usepackage{hyperref}
\usepackage{nameref}
\usepackage{titlesec}
\usepackage{longtable}
\usepackage[table]{xcolor}
\usepackage{titling}
\usepackage{lastpage}
\usepackage{ifthen}
\usepackage{calc}
\usepackage{soulutf8}
\usepackage{contour}
\usepackage{float}
\usepackage{fancyhdr}
\usepackage{multirow}
\usepackage{pgfgantt}
\usepackage{lscape}

\newcommand{\hr}{\par\vspace{-.1\ht\strutbox}\noindent\hrulefill\par}

\graphicspath{ {./}
	{./commons/res}
}

%--------------------------------------------------
% Comandi per inserire contenuto del documento
%--------------------------------------------------
\makeatletter

\newcommand\appendToGraphicsPath[1]{%
	\g@addto@macro\Ginput@path{{#1}}%
}

\newcommand{\setTitle}[1]{%
	\newcommand{\@phTitle}{#1}%
}
\newcommand{\phTitle}{\@phTitle}

\newcommand{\setDate}[1]{%
	\newcommand{\@phDate}{#1}%
}
\newcommand{\phDate}{\@phDate}

\newcommand{\setUso}[1]{%
	\newcommand{\@uso}{#1}%
}
\newcommand{\uso}{\@uso}

\newcommand{\setVersione}[1]{%
	\newcommand{\@versione}{#1}%
}
\newcommand{\versione}{\@versione}

\newcommand{\disabilitaVersione}{%
	\renewcommand{\setVersione}[1]{}%
	\renewcommand{\versione}{DISABILITATA}
}

\newcommand{\setResponsabile}[1]{%
	\newcommand{\@responsabile}{#1}%
}
\newcommand{\responsabile}{\@responsabile}

\newcommand{\setRedattori}[1]{%
	\newcommand{\@redattori}{#1}%
}
\newcommand{\redattori}{\@redattori}

\newcommand{\setVerificatori}[1]{%
	\newcommand{\@verificatori}{#1}%
}
\newcommand{\verificatori}{\@verificatori}

\newcommand{\setModifiche}[1]{%
	\newcommand{\@modifiche}{#1}%
}
\newcommand{\modifiche}{\@modifiche}

\makeatother 

%--------------------------------------------------
% Comandi per i documenti esterni e il glossario
%--------------------------------------------------

\newcommand{\dext}[1]{\textsc{#1\textsubscript{\textit{D}}}}

\newcommand{\glock}[1]{\textsc{#1\textsubscript{\textit{G}}}}

%--------------------------------------------------
% Comandi per impostare sottotitoli di quarto e quinto livello
%--------------------------------------------------

\setcounter{secnumdepth}{4}
\setcounter{tocdepth}{4}

\titleformat{\paragraph}
{\normalfont\normalsize\bfseries}{\theparagraph}{1em}{}
\titlespacing*{\paragraph}{0pt}{2.25ex plus 1ex minus .2ex}{1.5ex plus .2ex}

\titleformat{\subparagraph}
{\normalfont\normalsize\bfseries}{\thesubparagraph}{1em}{}
\titlespacing*{\subparagraph}{0pt}{1.75ex plus 1ex minus .2ex}{.75ex plus .1ex}

\appendToGraphicsPath{../../commons/res/}

%------------------------------
%
% COMANDI DI CONFIGURAZIONE
%
%------------------------------

\setTitle{Norme di Progetto}

\setVersione{0.2.0}

\setDate{15-12-2020}

\setResponsabile{Paolo Scanferlato}

\setRedattori{Alessandro Chimetto
	\\& Alessandro Dindinelli
	\\& Giacomo Bulbarelli
	\\& Lucia Fenu
	\\& Matteo Alba
	\\& Paolo Scanferlato
	\\& Valton Tahiraj}

\setVerificatori{Alessandro Chimetto}

\setUso{Interno}

\setModifiche{
	0.2.0 & Alessandro Dindinelli	& Redattore 	& 19-12-2020 & Aggiunto Capitolo Sviluppo \\
	0.1.0 & Alessandro Chimetto		& Verificatore	& 15-12-2020 & Verificata Introduzione \\
	0.1.0 & Valton Tahiraj 			& Redattore 	& 14-12-2020 & Aggiunto Introduzione \\
	0.0.0 & Alessandro Chimetto		& Verificatore	& 14-12-2020 & Verifica prima stesura \\
	0.0.0 & Valton Tahiraj 			& Redattore 	& 14-12-2020 & Prima stesura
}

\begin{document}
	% Direttive per la creazione del titolo tramite comando maketitle
\title{\huge \textsc{\phTitle{}} \\
	\vspace{11pt} \large \textsc{\phDate{}}}

\author{} % Non toccare
\date{} % Non toccare

%--------------------
% Frontespizio
%--------------------

% Logo del gruppo
\begin{figure}[t!]
	\centering
	\includegraphics[width=20em]{lclong}
\end{figure}

% Titolo / Nome
\maketitle
\thispagestyle{empty}

% Dati specifici sul doc in forma tabulare
\begin{table}[ht]
	\begin{center}
		\label{tab:Dati sul documento}
		\begin{tabular}{r|l}
			\multicolumn{2}{c}{ \textsc{Dati sul documento} } \\
			\hline
			\textbf{Versione} & \versione{} \\
			\textbf{Uso} & \uso{}  \\
			\textbf{Redattori} & \redattori{} \\
			\textbf{Verificatori} & \verificatori{} \\
			\textbf{Responsabile} & \responsabile{} \\
			\textbf{Destinatari} & lineCode \\
								& prof.\ Vardanega Tullio \\		
								& prof.\ Cardin Riccardo \\
			\ifthenelse{\equal{\uso}{Esterno}}{
								& Sanmarco Informatica
			}{} \\
		\end{tabular}
	\end{center}
\end{table}

\newpage

\renewcommand{\arraystretch}{2} % allarga le righe con dello spazio sotto e sopra
\begin{longtable}[H]{>{\centering\bfseries}m{2cm} >{\centering}m{3.5cm} >{\centering}m{2.5cm} >{\centering}m{3cm} >{\centering\arraybackslash}m{5cm}}
	\rowcolor{lightgray}
	{\textbf{Versione}} & {\textbf{Nominativo}} & {\textbf{Ruolo}} & {\textbf{Data}} & {\textbf{Descrizione}}  \\
	\endfirsthead%
	\rowcolor{lightgray}
	{\textbf{Versione}} & {\textbf{Nominativo}}  & {\textbf{Ruolo}} & {\textbf{Data}} & {\textbf{Descrizione}}  \\
	\endhead%
	\modifiche{}%
\end{longtable}
	\newpage
	%--------------------------------
	%
	% IL CONTENUTO INIZIA DA QUI
	%
	%--------------------------------
	\section{Introduzione}
		\subsection{Scopo del documento}
		Il documento ha lo scopo di definire le guidelines del way of working adottato dal team lineCode. Le attività presenti in questo documento sono redatte da processi contenuti nello standard ISO/IEC 12207:1995. Risulta quindi necessario che tutti i membri del gruppo prendano visione di questo documento ai fini di coesione e uniformità all'interno del progetto.

		\subsection{Scopo del Prodotto}
		Il \glock{capitolato} C5 ha come obbiettivo la realizzazione di un applicativo \glock{Real-Time} in grado di guidare delle unità dotate di mobilità autonoma in ambienti specifici, partendo dal presupposto che queste si muovano in ambienti in cui sono presenti altre unità (autonome o meno).

		\subsection{Glossario e documenti esterni}
		In supporto alla documentazione viene fornito un glossario per chiarire, con una definizione, eventuali termini specifici contenuti in questo documento.
		Saranno adottati quindi questi due simboli a pedice:
		\begin{itemize}
			\item \textit{D} se indicano un documento specifico;
			\item \textit{G} se incluse nel \dext{glossario}.
		\end{itemize}

		\subsection{Riferimenti}
			\subsubsection{Riferimenti Normativi}
			\begin{itemize}
				\item \textbf{{\glock{capitolato} C5 - PORTACS}}: \url{https://www.math.unipd.it/~tullio/IS-1/2020/Progetto/C5.pdf};
				\item \textbf{{ISO 8601:1988 (Data and Time Formats)}}: \url {https://www.w3.org/TR/NOTE-datetime}. %! formati al momento non implementati
			\end{itemize}
			\subsubsection{Rifermenti informativi}
			\begin{itemize}
				\item \textbf{ISO/IEC 12207:1995}: \url{https://www.math.unipd.it/~tullio/IS-1/2009/Approfondimenti/ISO_12207-1995.pdf}
				\item \textbf{Studio di Fattibilità}: ; %! da completare
				\item \textbf{Piano di Qualifica}: ;
				\item \textbf{Piano di Progetto}: .
			\end{itemize}

		\newpage
		\section{Processi Primari}
		
			%! inserire "Fornitura"
		
			\newpage
			\subsection{Sviluppo}
			
				\subsubsection{Scopo}
				Questa sezione raccoglie le linee guida per i compiti e le attività da svolgere al fine di ottenere il prodotto finale richiesto dal proponente. \\
				Per implementare correttamente il processo si devono stabilire i seguenti punti:
				\begin{itemize}
					\item obbiettivi di sviluppo;
					\item vincoli tecnologici e di design.
				\end{itemize}
				Il prodotto finale deve rispettare i requisiti, superare i test definiti e soddisfare le richieste del proponente.
				\subsubsection{Descrizione}
				Seguendo lo standard ISO/IEC 12207:1995, il processo di sviluppo si divide in:
				\begin{itemize}
					\item Analisi dei Requisiti;
					\item Progettazione;
					\item Codifica.
				\end{itemize}
			
				\subsubsection{Attività}
				
					\paragraph{Analisi dei requisiti} %! COME MAI NON C'E L'INDICE?
						\subparagraph{Scopo} %! COME MAI E' FORAMATTATO MALE?
						Il compito degli analisti è quello di redigere il documento di Analisi dei Requisiti, che andrà ad elencare e definire i requisiti del \glock{capitolato}. Lo scopo dei requisiti è quello di:
						\begin{itemize}
							\item descrivere il prodotto da realizzare;
							\item rendere disponibili ai progettisti riferimenti precisi;
							\item esprimere casi d'uso e requisiti concordati;
							\item rendere disponibili ai verificatori riferimenti per il controllo dei test;
							\item ragionare sul lavoro richiesto per produrre una stima dei costi.
						\end{itemize}
						\subparagraph{Classificazione dei requisiti} %! COME MAI E' FORAMATTATO MALE?
						Ogni requisito sarà associato ad un codice che rispetterà il seguente formato:
						\begin{center}
							\textbf{R[Priorità]-[Categoria]-[Codice]}
						\end{center}
						\begin{itemize}
							\item \textbf{R:} requisito;
							\item \textbf{Priorità:}
								\begin{itemize}
									\item \textbf{M:} mandatory/obbligatorio, quindi necessario a garantire le funzioni base del prodotto;
									\item \textbf{D:} desirable/desiderabile, cioè non strettamente necessario, ma che porta alla completezza del prodotto;
									\item \textbf{O:} optional/opzionale, quindi non pregiudica la funzionalità del prodotto finale.
								\end{itemize}
							\item \textbf{Categoria:}
								\begin{itemize}
									\item \textbf{F:} functional/funzionale;
									\item \textbf{P:} performance/prestazionale;
									\item \textbf{Q:} qualitative/qualitativo;
									\item \textbf{C:} constraint/vincolo.
								\end{itemize}
							\item \textbf{Codice:} numero progressivo per riconoscere univocamente il requisito.
						\end{itemize}
						\subparagraph{Classificazione dei casi d'uso} %! COME MAI E' FORAMATTATO MALE?
						Come per i requisiti, si prevede di identificare univocamente i casi d'uso con il seguente formato:
						\begin{center}
							\textbf{UC[Codice]}
						\end{center}
						\begin{itemize}
							\item \textbf{UC:} use case/caso d'uso;
							\item \textbf{Codice:} serie di cifre divise tra loro da ’.’ in modo da poter dividere in modo gerarchico i casi e sotto casi.
						\end{itemize}
						Oltre all'identificativo, ogni caso d'uso verrà descritto dai seguenti campi:
						\begin{itemize}
							\item \textbf{Descrizione:} breve spiegazione della situazione modellata;
							\item \textbf{Grafici UML:} diagrammi realizzati usando la versione 2.0 del linguaggio, ed il software \glock{StarUML};
							\item \textbf{Attori:} gli attori primari e secondari coinvolti;
							\item \textbf{Scenario Principale:} elenco numerato degli eventi descritti dal caso d'uso;
							\item \textbf{Precondizione:} condizioni che si assumono vere prima che si verifichino gli eventi descritti dal caso d'uso;
							\item \textbf{Postcondizione:} condizioni che si assumono vere dopo che si sono verificati gli eventi descritti dal caso d'uso;
							\item eventuali estensioni ed inclusioni coinvolte.
						\end{itemize}
					
					\paragraph{Progettazione} %! COME MAI NON C'E L'INDICE?
						\subparagraph{Scopo} %! COME MAI E' FORAMATTATO MALE?
						Mentre l'Analisi dei Requisiti divide il problema in parti per capirne completamente il dominio applicativo, la Progettazione rimette insieme tali parti specificandone le funzionalità in modo da realizzare l'architettura che il prodotto finale dovrà seguire. Questa dovrà seguire i seguenti principi:
						\begin{itemize}
							\item rispetto di tutti i requisiti;
							\item affidabilità nello svolgere i propri compiti;
							\item possibilità e facilità di garantire la manutenzione nel tempo;
							\item essere sicura rispetto ad intrusioni e malfunzionamenti;
							\item avere componenti coese, incapsulate e con scarse dipendenze tra loro.
						\end{itemize}
						\subparagraph{Descrizione} %! COME MAI E' FORAMATTATO MALE?
						Possiamo dividere questa fase in due componenti principali:
						\begin{itemize}
							\item \textbf{Technology Baseline:} contenente le specifiche della progettazione ad alto livello del prodotto, i diagrammi UML utilizzati per la realizzazione dell'architettura ed i test di verifica;
							\item \textbf{Product Baseline:} approfondisce ulteriormente l'attività di progettazione, integrando la Technology Baseline. Inoltre definisce i test necessari alla verifica.
						\end{itemize}
						\subparagraph{Technology Baseline} %! COME MAI E' FORAMATTATO MALE?
						Le componenti principali che individuiamo sono:
						\begin{itemize}
							\item diagrammi UML, utilizzati per rendere più chiare le soluzioni progettuali utilizzate, possono essere diagrammi di classi, package, attività o sequenza;
							\item descrizione delle tecnologie adottate, specificandone l'utilizzo nel progetto, i vantaggi e gli svantaggi;
							\item design patterns per adottare opportune soluzioni progettuali a problemi ricorrenti;
							\item test di integrazione, per verificare che ogni componente del sistema funzioni nella maniera voluta.
						\end{itemize} 
						\subparagraph{Product Baseline} %! COME MAI E' FORAMATTATO MALE?
						Gli aspetti principali su cui soffermarsi sono:
						\begin{itemize}
							\item definizione delle classi e loro descrizione per scopo e funzionalità;
							\item tracciamento delle classi, in modo che ogni per requisito esista una classe che lo soddisfa;
							\item test di unità, definiti in modo da verificare che le parti funzionino individualmente nel modo stabilito.
						\end{itemize} 
					
					\paragraph{Codifica} %! COME MAI NON C'E L'INDICE?
						\subparagraph{Scopo} %! COME MAI E' FORAMATTATO MALE?
						Questa fase ha lo scopo di realizzare il prodotto software richiesto. I programmatori dovranno attenersi a queste norme durante la fase di programmazione ed implementazione.
						\subparagraph{Descrizione} %! COME MAI E' FORAMATTATO MALE?
						La scrittura del codice dovrà rispettare delle linee guida in modo da ottenere codice leggibile ed uniforme per i programmatori, ed agevolare poi le fasi di manutenzione, verifica e validazione. Le convenzioni principali comprendono:
						\begin{itemize}
							\item regole per la dichiarazione e la denominazione di variabili e funzioni;
							\item definizione degli standard di indentazione, spazi e commenti;
							\item prassi e principi di programmazione.
						\end{itemize}
						


\end{document}

