\subsection{Gestione processi}

	\subsubsection{Scopo}
	L'utilizzo di una gestione dei processi ha lo scopo di:
	\begin{itemize}
		\item identificare e gestire i rischi;
		\item pianificare singole attività in base a determinate scadenze temporali;
		\item definire un modello di lavoro da seguire;
		\item calcolare un prospetto economico in base ai ruoli ed al monte ore;
		\item calcolare il bilancio finale.
	\end{itemize}
	Queste attività sono riportate in modo dettagliato nel \dext{Piano di Progetto v3.0.0}, elaborato dal responsabile di progetto in collaborazione con l'amministratore.
	Gli obiettivi prefissati da tale gestione sono i seguenti:
	\begin{itemize}
		\item adattare le attività in base ai rischi identificati, permettendo di limitare i danni qualora si presentassero;
		\item gestire i membri del team in base alle singole attività, creando un lavoro in parallelo, tenendo conto dei ruoli assegnati;
		\item migliorare la comunicazione all'interno del team;
		\item tenere sotto controllo l'andamento, al fine di ottimizzare l'efficienza del lavoro.
	\end{itemize}

	\subsubsection{Attività}
		\paragraph{Gestione dei rischi}
		I rischi che verranno rilevati e conseguentemente classificati, sono descritti dettagliatamente nel \dext{Piano di Progetto v3.0.0} che sarà sempre aggiornato su:
		\begin{itemize}
			\item nuovi rischi individuati;
			\item ridefinizione, all'occorrenza, di nuove strategie sulla gestione dei rischi;
			\item monitoraggio dei rischi pervenuti.
		\end{itemize}
		\noindent
		I rischi sono così classificati:
		\begin{center}
			{\bfseries RS[Categoria][Codice]}
		\end{center}
		dove:
		\begin{itemize}
			\item \textbf{Categoria}: indica la tipologia a cui appartiene il rischio con i seguenti valori:
			\begin{itemize}
				\item {\bfseries O}: rischio Organizzativo;
				\item {\bfseries T}: rischio Tecnologico;
				\item {\bfseries P}: rischio di carattere Privato/Personale. \\
			\end{itemize}
			\item \textbf{Codice}: intero positivo che identifica il singolo rischio. Il valore parte da 01.
		\end{itemize}

    	\paragraph{Assegnazione ruoli di progetto}
    	Tutti i membri del gruppo dovranno assumere diversi ruoli a rotazione e, per ognuno di essi, saranno previsti compiti specializzati. Di seguito sono riportate, in linea generale, gli aspetti caratterizzanti di ogni ruolo che verrà ricoperto.

    		\subparagraph{Responsabile di progetto}
    		Il responsabile di progetto ha la responsabilità sulla pianificazione, il controllo e la gestione delle attività del gruppo. Essendo la figura principale all'interno del team, si interessa anche dell'amministrazione esterna, come la comunicazione e corrispondenza con il committente e/o proponente del capitolato.\\
    		Riassumendo, si occupa di:
    		\begin{itemize}
    			\item gestione e controllo delle attività;
    			\item coordinamento dei compiti in base ai ruoli assunti dai singoli membri;
    			\item approvare i documenti finali e le offerte proposte;
    			\item gestire i rischi in base al \dext{Piano di Progetto v3.0.0};
    			\item redigere il \dext{Piano di Progetto v3.0.0}.
    		\end{itemize}

    		\subparagraph{Amministratore}
    		L'amministratore ha il compito di supportare e garantire il controllo dell'ambiente di lavoro. \\
    		In generale, si occupa di:
    		\begin{itemize}
    			\item coordinare l'ambiente di lavoro con adeguato supporto;
    			\item gestire la documentazione;
    			\item controllare versioni e configurazioni;
    			\item risolvere i problemi legati alla gestione;
    			\item redigere le \dext{Norme di Progetto v3.0.0} e parte del \dext{Piano di Qualifica v3.0.0}.
    		\end{itemize}

    		\subparagraph{Analista}
    		L'analista si occupa dell'analisi dei requisiti e della loro classificazione. Per questo motivo, non sarà presente per l'intera realizzazione del progetto.
    		Si occupa di:
    		\begin{itemize}
    			\item studio del dominio;
    			\item definizione dei requisiti necessari;
    			\item definizione dei rischi;
    			\item redigere l'\dext{Analisi dei Requisiti v3.0.0} e lo \dext{Studio di Fattibilità v3.0.0}.
    		\end{itemize}

    		\subparagraph{Progettista}
    		Il progettista ha il compito di gestire tutte le attività in campo tecnico e tecnologico e deve:
    		\begin{itemize}
    			\item compiere delle decisioni su aspetti tecnici e tecnologici;
    			\item definire e sviluppare l'architettura che verrà sviluppata, in modo stabile e mantenibile;
    			\item redigere parte del \dext{Piano di Qualifica v3.0.0}, la Specifica Tecnica e la Definizione del Prodotto.
    		\end{itemize}

    		\subparagraph{Programmatore}
    		Il programmatore è il responsabile della codifica di tutte le attività atte allo sviluppo del progetto. \\
    		Ha il compito di:
    		\begin{itemize}
    			\item implementare le specifiche del progettista;
    			\item gestire le componenti di sviluppo, verifica e mantenimento del prodotto.
    		\end{itemize}

    		\subparagraph{Verificatore}
    		Il verificatore controlla il lavoro svolto dagli altri membri, ma è esonerato dalla correzione di eventuali errori. \\
    		Il suo compito dunque, è quello di:

    		\begin{itemize}
    			\item controllare i lavori in fase di revisione seguendo le \dext{Norme di Progetto v3.0.0};
    			\item rilevare e comunicare gli errori.
    		\end{itemize}

		\paragraph{Gestione delle comunicazioni}
    		\subparagraph{Comunicazioni interne}
    		Le comunicazioni interne al gruppo vengono effettuate attraverso l'utilizzo di \glock{Discord}, un software che permette la divisione della chat e delle chiamate in base agli argomenti tramite l'uso di canali personalizzabili manualmente. \\ Permette l'uso di bot con cui è possibile notificare le \glock{pull-request}.\\
    		Il Workspace è suddiviso nel seguente modo:
    		\begin{itemize}
    		 	\item {\ttfamily generale}: per comunicazioni di bassa importanza;
    		 	\item {\ttfamily files}: per inviare i documenti e/o link;
    		 	\item {\ttfamily riferimenti}: contenitore di indirizzi e riferimenti importanti;
    		    \item {\ttfamily pull-request}: contenitore delle \glock{pull-request} e dei suoi verificatori.
    		\end{itemize}

    		\subparagraph{Comunicazioni esterne}
    		Le comunicazioni esterne sono a cura del responsabile di progetto. \\
    		Per la corrispondenza è stato creato un indirizzo di posta elettronica \glock{Gmail}: \url{linecode.swe@gmail.com}, reso accessibile a tutti i membri del gruppo.\\
    		Il proponente del capitolato ha  permesso l'uso dei seguenti indirizzi e-mail:
    		\begin{itemize}
    		    \item \url{alex.beggiato@sanmarcoinformatica.it}
    		 	\item \url{alessandra.piva@sanmarcoinformatica.it}
    		\end{itemize}

        \paragraph{Gestione degli incontri}
        Gli incontri, sia interni che esterni, sono esclusivamente online causa SARS-CoV-2.

            \subparagraph{Incontri interni}
            Si utilizza il canale vocale di \glock{Discord}, suddiviso nelle seguenti stanze:
            \begin{itemize}
                \item {\sffamily Generale}: per svolgere le riunioni;
                \item {\sffamily Room1} o {\sffamily Room2}: nel caso in cui durante una riunione fosse necessario dividersi per non disturbare il lavoro degli altri membri.
            \end{itemize}
            Le riunioni vengono stabilite di volta in volta durante gli incontri online.
            Nel caso ci fossero degli imprevisti si utilizza il canale testuale per avvisare tutti i membri di un eventuale ritardo da parte di un componente, e/o modifica dell'orario previsto assicurandosi che tutti diano conferma.

            \subparagraph{Incontri esterni}
            Per gli incontri con il proponente viene utilizzata la piattaforma \glock{Google Meet} per incontri sincroni, mentre per comunicazioni asincrone si utilizza la piattaforma \glock{Google Chat}.

		\paragraph{Verbali}
        All'inizio di ogni riunione si decide chi ha il compito di trascrivere degli appunti sulle questioni e decisioni sollevate durante l'incontro. \\
        La trascrizione sarà a rotazione.\\
        Dopodiché si trattano gli argomenti del giorno che sono stati tracciati alla riunione precedente o durante comunicazioni interne.

	\subsubsection{Strumenti}
	Per la gestione delle attività viene utilizzato il sistema di \glock{board} integrato con il servizio \glock{GitKraken}.\\
	Le \glock{board} sono configurate nella maniera seguente:
	\begin{itemize}
		\item ad ogni \glock{task} identificato dal gruppo viene associata una \glock{card} nella \glock{board};
		\item ad ogni \glock{task} vengono associati dei \glock{label} che aggiungono contenuto informativo sul \glock{task} stesso e che si presentano con la seguente struttura:
		\begin{center}
 			{\bfseries Lettera - Specifica}
 		\end{center}
dove Lettera può assumere i seguenti valori:
		\begin{itemize}
			\item \textbf{A}: identifica il target del task con campo Specifica che può assumere valori:
			\begin{itemize}
				\item Documentazione;
				\item Codice;
				\item Amministrazione.
			\end{itemize}
			\item \textbf{B}: identifica il tipo di operazione che il task richiede con campo Specifica che può assumere valori:
			\begin{itemize}
				\item Incremento;
				\item Correzione;
				\item Approvazione.
			\end{itemize}
			\item \textbf{C}: identifica la priorità del task con campo Specifica che può assumere valori:
			\begin{itemize}
				\item Urgente;
				\item Neutra;
				\item Bassa.
			\end{itemize}
		\end{itemize}
		\item ad ogni \glock{card} che corrisponde a un \glock{task} preso a carico da uno o più membri del gruppo vengono associati gli assegnatari che si occuperanno di svolgere il suddetto task e tale assegnazione sarà visibile all'interno della card stessa per facilitare il tracciamento dei membri che hanno a carico un determinato incarico;
		\item le \glock{card} possono trovarsi in 1 di 4 colonne che ne identificano lo stato di completamento, divise in:
		\begin{itemize}
			\item \textbf{To Do}: \glock{task} da svolgere e non ancora cominciati;
			\item \textbf{In progress} \glock{task} in fase di svolgimento presi a carico da uno o più membri del gruppo;
			\item \textbf{Pending review}: \glock{task} che sono stati completati e in attesa di verifica;
			\item \textbf{Done}: \glock{task} che sono stati completati e verificati.
		\end{itemize}
	\end{itemize}

	\subsubsection{Metriche}
	Durante la realizzazione del processo verranno utilizzate le seguenti metriche di qualità sulla gestione ed il tracciamento di processi e rischi; al fine di ottenere efficacia ed efficienza.

	 	\paragraph{MP01 - Schedule Variance - SV}
	 	\begin{itemize}
	 		\item \textbf{Descrizione}:
	 			La Schedule Variance (SV) indica quanto si è in linea, in anticipo o in ritardo rispetto alla schedulazione delle attività di progetto pianificate. \\
	 			È un indicatore di efficacia soprattutto nei confronti del cliente.\\
	 			Il valore è espresso in percentuale, secondo la formula:
	 			\begin{displaymath}
	 				SV = \frac{BCWP-BCWS}{BCWP}\times100
	 			\end{displaymath}
	 			Dove:
	 			\begin{itemize}
	 				\item {\bfseries Budgeted Cost of Work Performed - BCWP}
	 				\begin{itemize}
	 					\item \textbf{Descrizione}:
	 					è  il valore (in giorni o EURO) delle attività realizzate alla data corrente.\\
	 					Il suo valore è una stima approssimata, secondo i seguenti procedimenti:
	 					\begin{displaymath}
	 						BCWP_t = \sum BCWP_s
	 					\end{displaymath}
	 					dove:
	 					\begin{itemize}
	 						\item {\bfseries $BCWP_t$}: è il BCWP totale del progetto;
	 						\item {\bfseries $BCWP_s$}: è il BCWP della singola attività, secondo la formula:
	 						\begin{displaymath}
	 							BCWP_s = CA\times BAC
	 						\end{displaymath}
	 						dove:
	 						\item  CA: {\itshape Completamento Attività}, espresso in percentuale;
	 						\item BCA: {\itshape Budget at Completion}, è il valore inizialmente previsto per la realizzazione del progetto.
	 					\end{itemize}
	 					\item \textbf{Unità di misura}: numero;
	 				\end{itemize}
	 				\item {\bfseries Budgeted Cost of Work Scheduled - BCWS}
	 				\begin{itemize}
	 					\item \textbf{Descrizione}: è il costo pianificato (in giorni o EURO) per realizzare le attività di progetto alla data corrente;
	 					\item \textbf{Unità di misura}: numero;
	 				\end{itemize}
	 			\end{itemize}
	 		\item \textbf{Unità di misura}: numero;
	 		\item \textbf{Risultato}:
	 			\begin{itemize}
	 				\item positivo: indica che si è avanti rispetto alla schedulazione;
	 				\item negativo: indica che si è in ritardo rispetto alla schedulazione;
	 				\item uguale a zero: indica che il progetto è in linea con i tempi rispetto a quanto preventivato. \\
	 			\end{itemize}
	 	\end{itemize}

 		\paragraph{MP02 - Cost Variance - CV}
 		\begin{itemize}
 			\item \textbf{Descrizione}:
 			La Cost variance, indica se il valore del costo realmente maturato è maggiore, uguale o minore rispetto al costo effettivo; ovvero quanto sia il livello di efficienza del gruppo rispetto a quanto pianificato. \\
 			Viene usato come indicatore di produttività o efficienza soprattutto nei confronti del {\itshape Management} dell'azienda. \\
 			Il suo valore è espresso in percentuale, secondo la seguente formula:
 			\begin{displaymath}
 				CV = \frac{BCWP - ACWP}{BCWP}\times100
 			\end{displaymath}
 			dove:
 			\begin{itemize}
 				\item {\bfseries Budgeted Cost of Work Performed - BCWP}: è il valore (in giorni o EURO) delle attività realizzate alla data corrente;
 				\item {\bfseries Actual Cost of Work Performed - ACWP}: è il costo effettivamente sostenuto (in giorni o EURO) alla data corrente.
 			\end{itemize}
 			La differenza tra BCWP e ACWP è che se l'attuale costo (ACWP) alla data corrente è più alto rispetto al guadagno (BCWP) alla data corrente, allora sappiamo che l'appaltatore sta attualmente superando i costi previsti nella stima al completamento, sforando il budget.
 			\item \textbf{Unità di misura}: numero;
 			\item \textbf{Risultato}:
 			\begin{itemize}
 				\item positivo: indica che il progetto viene sviluppato con un costo minore rispetto a quanto preventivato;
 				\item negativo:  indica che il progetto viene sviluppato con un costo maggiore rispetto a quanto preventivato;
 				\item uguale a zero: indica che il progetto viene sviluppato con un costo in linea rispetto a quanto preventivato. \\
 			\end{itemize}
 		\end{itemize}

 		\paragraph{MP03 - Budget Variance - BV}
 		\begin{itemize}
 			\item \textbf{Descrizione}:
 			La Budget Variance è una metrica che indica se si è speso di più o di meno rispetto a quanto previsto, alla data corrente. Segue la formula:
 			\begin{displaymath}
 				BV = {BCWS – ACWP}
 			\end{displaymath}
 			dove:
 			\begin{itemize}
 				\item {\bfseries Budgeted Cost of Work Scheduled - BCWS}: è il costo pianificato (in giorni o EURO) per realizzare le attività di progetto alla data corrente;
 				\item {\bfseries Actual Cost of Work Performed - ACWP}: è il costo effettivamente sostenuto (in giorni o EURO) alla data corrente.
 			\end{itemize}
 			\item \textbf{Unità di misura}: numero;
 			\item \textbf{Risultato}:
            \begin{itemize}
     			\item positivo: indica che il progetto sta spendendo il proprio budget con minor velocità di quanto pianificato;
     			\item negativo: il progetto sta spendendo il proprio budget con maggior velocità di quanto pianificato;
     			\item uguale a zero: indica che il progetto sta spendendo il proprio budget con una velocità in linea rispetto a quanto preventivato.\\
            \end{itemize}
 		\end{itemize}

 		\paragraph{MP04 - Unbudgeted Risks - UR}
 		\begin{itemize}
 			\item \textbf{Descrizione}:
 			L'Unbudgeted Risks, indica il numero di rischi non rilevati in fase di analisi, in modo incrementale. Per ogni rischio non preventivato che viene rilevato, si incrementa di una unità (partendo dallo zero) il numero di rischi rilevati dall'inizio del progetto fino a data corrente; secondo la formula:
 			\begin{displaymath}
 				UR = UR + 1
 			\end{displaymath}
 			\item \textbf{Unità di misura}: numero intero maggiore o uguale a 0;
 			\item \textbf{Risultato}:
 			\begin{itemize}
 				\item maggiore di zero: sono stati trovati rischi in fase di progetto;
 				\item uguale a zero: non sono stati rilevati rischi in fase di progetto. \\
 			\end{itemize}
 		\end{itemize}

 	

 	\subsubsection{Strumenti}
 	Per permettere l'attuazione delle procedure sopra citate, sono utilizzati i seguenti strumenti:
 	\begin{itemize}
 		\item \textbf{\glock{Discord}}: strumento per comunicazioni e incontri interni al gruppo;
 		\item \textbf{\glock{GitHub}}: strumento per il versionamento remoto dei file condivisibili a tutti i membri del gruppo;
 		\item \textbf{\glock{GitKraken}}: strumento collegato alla \glock{Repository} \glock{GitHub}, per gestire in modo più semplice ed intuitivo i branch, commit e pull request. Viene inoltre utilizzato per l'implementazione e la gestione del sistema di \glock{board} descritto alla §4.1.3;
 		\item \textbf{\glock{GMail}}: servizio di posta elettronica per le comunicazioni esterne con i professori;
        \item \textbf{\glock{Google Meet}}: utilizzato per la comunicazione sincrona con il proponente;
        \item \textbf{\glock{Google Chat}}: utilizzato per la comunicazione asincrona con il proponente;
 		\item {\bfseries \glock{Issue Tracking System}}: strumento fornito da \glock{GitHub} per la gestione e il tracciamento dei compiti assegnati.
 	\end{itemize}
