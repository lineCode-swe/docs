\subsection{Sviluppo}

	\subsubsection{Descrizione}
	Questa sezione raccoglie le linee guida per i compiti e le attività da svolgere al fine di ottenere il prodotto finale richiesto dal proponente.\\
	Istanziando lo standard ISO/IEC 12207:1995 per renderlo conforme alle nostre esigenze, il processo di sviluppo si divide in:
	\begin{itemize}
		\item Analisi dei Requisiti;
		\item Progettazione;
		\item Codifica.
	\end{itemize}

	\subsubsection{Attività}
        \paragraph{Analisi dei requisiti}
			Il compito degli analisti è quello di redigere il documento \dext{Analisi dei Requisiti v3.0.0}, che andrà ad elencare e definire i requisiti del \glock{capitolato}. Lo scopo dei requisiti è quello di:
			\begin{itemize}
				\item descrivere il prodotto da realizzare;
				\item rendere disponibili ai progettisti riferimenti precisi;
				\item esprimere casi d'uso e requisiti concordati;
				\item rendere disponibili ai verificatori riferimenti per il controllo dei test;
				\item ragionare sul lavoro richiesto per produrre una stima dei costi.
			\end{itemize}

		\paragraph{Progettazione}
   			Mentre l'\dext{Analisi dei Requisiti v3.0.0} divide il problema in parti per capirne completamente il dominio applicativo, la Progettazione rimette insieme tali parti specificandone le funzionalità in modo da realizzare l'architettura che il prodotto finale dovrà seguire. Questa dovrà seguire i seguenti principi:
   			\begin{itemize}
   				\item rispetto di tutti i requisiti;
   				\item affidabilità nello svolgere i propri compiti;
   				\item possibilità e facilità di garantire la manutenzione nel tempo;
   				\item essere sicura rispetto ad intrusioni e malfunzionamenti;
   				\item avere componenti coese, incapsulate e con scarse dipendenze tra loro.
   			\end{itemize}
   			Parte della Progettazione implica la creazione di due sottoprodotti fondamentali per l'andamento del progetto:
   			\begin{itemize}
   				\item \textbf{Technology baseline}: software eseguibile utilizzato come dimostrazione della bontà delle scelte effettuate sulle tecnologie per la realizzazione del prodotto;
   				\item \textbf{Product baseline}: parte del prodotto finale che implementi il design definitivo e funga da baseline sulla quale si baseranno i successivi incrementi.
            \end{itemize}

		\paragraph{Codifica}
   			\subparagraph{Scopo}
   			In questa attività i programmatori, sulla base dell'architettura proposta dai progettisti, realizzano il prodotto software richiesto.

            \subparagraph{Regole}
   			La scrittura del codice dovrà rispettare le seguenti regole in modo da ottenere codice leggibile, uniforme ed agevolare verifica e manutenzione:
   			\begin{enumerate}
                \item tutta la nomenclatura (es. variabili, classi, ecc.) deve essere in lingua inglese;
                \item ogni file sorgente deve cominciare con un commento multi-riga dove sono indicati i diritti d'autore e un riferimento al file di licenza;
                \item il codice \glock{Java} dovrà seguire tutte le convenzioni stilistiche indicate nelle \dext{Java Code Conventions};
                \item il codice \glock{TypeScript} dovrà seguire le convenzioni stilistiche indicate nella \dext{Angular coding style guide}.
            \end{enumerate}

    \subsubsection{Strumenti}
    \begin{itemize}
        \item \textbf{\glock{Java}}: linguaggio utilizzato per la realizzazione del motore di calcolo;
        \item \textbf{\glock{TypeScrypt}}: linguaggio utilizzato per la realizzazione di applicativi al di fuori del motore di calcolo;
        \item \textbf{\glock{Maven}}: strumento utilizzato per la \glock{build automation};
        \item \textbf{\glock{JaCoCo}}: libreria utilizzata per il \glock{code coverage};
        \item \textbf{\glock{Tyrus}}: \glock{framework} utilizzato per l'implementazione di \glock{WebSocket} in \glock{Java};
        \item \textbf{\glock{Grizzly}}: server \glock{http} e servlet container;
        \item \textbf{\glock{Checkstyle}}: strumento utilizzato per analisi statica del codice \glock{Java};
        \item \textbf{\glock{sonarcloud}}: utilizzato per rilevare le metriche di qualità;
        \item \textbf{\glock{Angular}}: libreria grafica utilizzata per lo sviluppo dell'interfaccia utente;
        \item \textbf{\glock{Node.js}}: programma applicativo utilizzato per la creazioni di applicazioni \glock{TypeScrypt} che comunicano con \glock{http};
        \item \textbf{\glock{Intellij IDEA}}: \glock{IDE} utilizzato per la scrittura del codice;
        \item \textbf{\glock{ESLint}}: strumento utilizzato per analisi statica del codice \glock{TypeScript};
        \item \textbf{\glock{Docker}}: utilizzato sotto richiesta del proponente per la containerizzazione delle applicazioni;
        \item \textbf{\glock{Docker Compose}}: utilizzato per la gestione dei \glock{container};
        \item \textbf{\glock{Github Action}}: strumento utilizzato per la \glock{continuous integration};
        \item \textbf{\glock{NPM (Node Package Manager)}}: \glock{gestore di pachetti} per l'ambiente di runtime Node.js.
    \end{itemize}

	\subsubsection{Metriche}
		\paragraph{MS01 - Numero Bug Rilevati}
		\begin{itemize}
			\item \textbf{Descrizione}: metrica di affidabilità; indica il numero di bug che sono stati scoperti in un componente software;
			\item \textbf{Unità di misura}: numero intero maggiore o uguale a 0.
		\end{itemize}

        \paragraph{MS02 - Vulnerabilità}
        \begin{itemize}
            \item \textbf{Descrizione}: metrica di sicurezza; indica il numero di vulnerabilità nel prodotto software facendo intendere quanto il software sia sicuro nel suo complesso;
            \item \textbf{Unità di misura}: numero intero maggiore o uguale a 0.
        \end{itemize}

        \paragraph{MS03 - Code Smells}
        \begin{itemize}
            \item \textbf{Descrizione}: metrica di manutenibilità; indica il numero di parti di codice in cui questo risulta, per sua stesura, confusionario e difficile da mantenere;
            \item \textbf{Unità di misura}: numero intero maggiore o uguale a 0.
        \end{itemize}

        \paragraph{MS04 - Debito Tecnico}
        \begin{itemize}
            \item \textbf{Descrizione}: metrica di manutenibilità; indica una stima del tempo necessario ad un programmatore per risolvere tutti i Code Smells sulla base della loro difficoltà;
            \item \textbf{Unità di misura}: ore.
        \end{itemize}

        \paragraph{MS05 - Complessità Ciclomatica}
        \begin{itemize}
            \item \textbf{Descrizione}: metrica di complicatezza del software; misura il numero di cammini linearmente indipendenti attraverso il grafo di controllo di flusso del programma: minore è il suo valore, minore è il rischio di bug in caso di modifiche e maggiore è la leggibilità. In termini più pragmatici, indica il numero minimo di test per ottenere Code Coverage completa;
            \item \textbf{Unità di misura}: numero intero maggiore o uguale a 1.
        \end{itemize}

		\paragraph{MS06 - Requisiti Obbligatori Soddisfatti}
		\begin{itemize}
			\item \textbf{Descrizione}: la metrica indica il quantitativo di requisiti obbligatori soddisfatti (progettati, sviluppati e verificati) fino alla data corrente; ciò permette, sia al gruppo che al committente, di comprendere la percentuale di completezza base del prodotto. La formula usata per il calcolo della metrica è la seguente:
            \[
            MS07 = \frac{requisiti\ obbligatori\ soddisfatti}{requisiti\ obbligatori\ totali} \times 100
            \]
			\item \textbf{Unità di misura}: percentuale.
		\end{itemize}

		\paragraph{MS07 - Requisiti Desiderabili Soddisfatti}
		\begin{itemize}
			\item \textbf{Descrizione}: la metrica indica il quantitativo di requisiti desiderabili soddisfatti (progettati, sviluppati e verificati) fino alla data corrente; ciò permette, sia al gruppo che al committente, di comprendere la percentuale di valore aggiunto del prodotto. La formula usata per il calcolo della metrica è la seguente:
            \[
            MS08 = \frac{requisiti\ desiderabili\ soddisfatti}{requisiti\ desiderabili\ totali} \times 100
            \]
			\item \textbf{Unità di misura}: percentuale.
		\end{itemize}

		\paragraph{MS08 - Requisiti Opzionali Soddisfatti}
		\begin{itemize}
			\item \textbf{Descrizione}: la metrica indica il quantitativo di requisiti opzionali soddisfatti (progettati, sviluppati e verificati) fino alla data corrente. La formula usata per il calcolo della metrica è la seguente:
            \[
            MS09 = \frac{requisiti\ opzionali\ soddisfatti}{requisiti\ opzionali\ totali} \times 100
            \]
			\item \textbf{Unità di misura}: percentuale.
		\end{itemize}