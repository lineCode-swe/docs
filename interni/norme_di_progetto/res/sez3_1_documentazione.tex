\subsection{Documentazione}

	\subsubsection{Descrizione}
	Questo capitolo descrive le linee guida decise dal gruppo per redigere, verificare e approvare tutti i documenti ufficiali, interni ed esterni, rilasciati durante tutto il ciclo di vita del prodotto software da sviluppare.\\
	I documenti ufficiali prodotti saranno suddivisi in due classi principali relative all'uso che ne viene fatto:
	\begin{enumerate}
		\item \textbf{interni}: possono essere visualizzati solo all'interno del gruppo;
		\item \textbf{esterni}: documenti ufficiali redatti a favore anche di chi non è parte del gruppo, come la Proponente ed i Committenti.
	\end{enumerate}

	\subsubsection{Attività}
		\paragraph{Stesura della documentazione ufficiale}
        Di seguito, sono elencati i documenti da produrre, il loro scopo e regole specifiche per la loro stesura.
            \subparagraph{Analisi dei requisiti}
            File: analisi\_dei\_requisiti\_vX.Y.Z.pdf, uso esterno. Analizza ed espone i requisiti del prodotto richiesto dalla Proponente, classificandoli in \textbf{obbligatori}, \textbf{desiderabili} ed \textbf{opzionali}.
            \\\\
            Ogni requisito sarà associato ad un identificativo che rispetterà il seguente formato:
            \begin{center}
                \textbf{R[Priorità]-[Categoria]-[Codice]}
            \end{center}
            \begin{itemize}
                \item \textbf{R:} requisito;
                \item \textbf{Priorità:}
                \begin{itemize}
                    \item \textbf{M}: mandatory/obbligatorio, quindi necessario a garantire le funzioni base del prodotto;
                    \item \textbf{D}: desirable/desiderabile, cioè non strettamente necessario, ma che porta alla completezza del prodotto;
                    \item \textbf{O}: optional/opzionale, quindi che non pregiudica la funzionalità del prodotto finale.
                \end{itemize}
                \item \textbf{Categoria:}
                \begin{itemize}
                    \item \textbf{F}: functional/funzionale;
                    \item \textbf{P}: performance/prestazionale;
                    \item \textbf{Q}: qualitative/qualitativo;
                    \item \textbf{C}: constraint/vincolo.
                \end{itemize}
                \item \textbf{Codice:} numero progressivo per riconoscere univocamente il requisito.
            \end{itemize}
            Come per i requisiti, si prevede di identificare univocamente i casi d'uso con il seguente formato:
            \begin{center}
                \textbf{UC[Codice]}
            \end{center}
            \begin{itemize}
                \item \textbf{UC}: use case/caso d'uso;
                \item \textbf{Codice}: serie di cifre separate da ’.’ così da poter dividere in modo gerarchico i casi e sotto casi.
            \end{itemize}
            Oltre all'identificativo, ogni caso d'uso verrà approfondito dai seguenti campi:
            \begin{itemize}
                \item \textbf{Descrizione}: breve spiegazione della situazione modellata;
                \item \textbf{Grafici \glock{UML}}: diagrammi realizzati usando la versione 2.0 del linguaggio;
                \item \textbf{Attori}: gli attori primari e secondari coinvolti;
                \item \textbf{Scenario Principale}: elenco numerato degli eventi descritti dal caso d'uso;
                \item \textbf{Precondizione}: condizioni che si assumono vere prima che si verifichino gli eventi descritti dal caso d'uso;
                \item \textbf{Postcondizione}: condizioni che si assumono vere dopo che si sono verificati gli eventi descritti dal caso d'uso;
                \item eventuali estensioni ed inclusioni coinvolte.
            \end{itemize}

            \subparagraph{Glossario}
            File: glossario\_vX.Y.Z.pdf, uso esterno. Un'unica raccolta dei termini presenti in tutti i documenti che potrebbero essere di difficile comprensione. È un documento esterno in quanto semplifica la lettura della documentazione relativa al progetto.

            \subparagraph{Norme di progetto}
            File: norme\_di\_progetto\_vX.Y.Z.pdf, uso interno. Specifica tutte le norme tenute dal gruppo di lavoro per l'intero sviluppo delle attività del progetto. È istanziazione di IEEE/IEC 12207:1995 e da esso prende ispirazione per strutturare le sue sezioni secondo la gerarchia seguente: categoria di processi, processo, descrizione/attività. Definisce le metriche di prodotto e di progetto che verranno poi istanziate nel \dext{Piano di qualifica}.

            \subparagraph{Piano di progetto}
            File: piano\_di\_progetto\_vX.Y.Z.pdf, uso esterno. Descrive le modalità con cui vengono utilizzate le risorse umane e temporali nello svolgimento del progetto. Indica inoltre le il modello di sviluppo scelto e fa un'analisi dei rischi previsti e riscontrati.\\
            I rischi sono così classificati:
            \begin{center}
                {\bfseries RS[Categoria][Codice]}
            \end{center}
            dove:
            \begin{itemize}
                \item \textbf{Categoria}: indica la tipologia a cui appartiene il rischio con i seguenti valori:
                \begin{itemize}
                    \item {\bfseries O}: rischio Organizzativo;
                    \item {\bfseries T}: rischio Tecnologico;
                    \item {\bfseries P}: rischio di carattere Privato/Personale. \\
                \end{itemize}
                \item \textbf{Codice}: intero positivo che identifica il singolo rischio. Il valore parte da 01.
            \end{itemize}

            \subparagraph{Piano di qualifica}
            File: piano\_di\_qualifica\_vX.Y.Z.pdf, uso esterno. Istanzia le metriche definite nelle \dext{Norme di progetto} con valori di soglia o intervallo e mette a disposizione un cruscotto per mostrare l'andamento attuale del progetto in funzione di tali metriche.

            \subparagraph{Studio di fattibilità}
            File: studio\_di\_fattibilita\_vX.Y.Z.pdf, uso interno. Documento che riporta lo studio svolto dagli analisti sui capitolati proposti. Per ciascun \glock{capitolato} viene riportato:
            \begin{itemize}
                \item \textbf{descrizione}: sintesi del prodotto richiesto da sviluppare presentato nel \glock{capitolato};
                \item \textbf{finalità}: ambito di utilizzo del prodotto da sviluppare;
                \item \textbf{tecnologie interessate:} lista di tutte le tecnologie interessate nello sviluppo del prodotto;
                \item \textbf{analisi motivazione, criticità e rischi}: racchiude le motivazioni, le criticità e i rischi risultati dall'analisi del \glock{capitolato};
                \item \textbf{valutazione finale}: indica le motivazioni per le quali il \glock{capitolato} è stato respinto o accettato.
            \end{itemize}

            \subparagraph{Verbali delle riunioni}
            Il contenuto viene suddiviso nei seguenti tre punti:
            \begin{enumerate}
                \item \textbf{introduzione}: comprende i dettagli tecnici della riunione:
                \begin{itemize}
                    \item luogo dell'incontro o in caso di modalità telematica specifica la tecnologia utilizzata;
                    \item data;
                    \item ora di inizio e di fine;
                    \item presenti e assenti del gruppo (in ordine alfabetico per cognome);
                    \item eventuali ospiti come la Proponente o i Committenti suddivisi per ruolo e/o azienda;
                    \item ordine del giorno concordato nella riunione precedente ed eventuali integrazioni.
                \end{itemize}
                \item \textbf{svolgimento}: specifica suddividendo in punti numerati gli argomenti discussi;
                \item \textbf{decisioni prese}: una tabella che riassume le decisioni prese dal gruppo assegnando ad ognuna di esse un identificativo univoco aggiungendo al nome del verbale un "." e un numero progressivo a partire da 1 (per ogni verbale).
            \end{enumerate}
            Riguardo alla loro nomenclatura, saranno nominati con una V maiuscola seguita dalle lettere I oppure E maiuscole per i verbali interni ed esterni rispettivamente, seguite da un underscore "\_", la data in cui è avvenuta la riunione nel formato "AAAA-MM-GG", un underscore "\_" ed il numero progressivo del verbale. Quindi ad esempio il terzo verbale interno, riguardo la riunione del 01-04-2021 sarà nominato: \textit{VI\_2021-04-01\_3}.\\
            \\Questi particolari documenti saranno classificati come interni e una volta verificati ed approvati non potranno più essere aggiornati a nuove versioni trattando, nello specifico, eventi accaduti in un determinato momento e che non possono quindi cambiare.
            \\In caso di riunioni in cui saranno presenti ospiti prenderanno il nome di \dext{Verbale Esterno} e saranno nominati con VE, mentre le altre regole (numero progressivo a partire da 1 e classificazione decisioni prese) rimarranno invariate.

            \subparagraph{Specifica dei test}
            Per tenere traccia e dare scontro dei test effettuati sul prodotto verrà fornito un documento apposito in aggiunta a quelli già dichiarati in precedenza. Tale documento verrà redatto in \glock{AsciiDoc}, avrà una struttura meno rigida di quelli fino ad ora discussi e conterrà unicamente la dichiarazione dei test che verranno effettuati sul prodotto.
            Tali test saranno riportati sotto forma di pseudo-codice, normato di seguito al fine di garantire coerenza e uniformità durante la stesura.
            \\\\
            I test di sistema vengono classificati nella maniera seguente:
            \begin{center}
                \textbf{TS[Priorità]-[Categoria]-[Codice]}
            \end{center}
            dove:\\
            \begin{itemize}
                \item \textbf{Priorità}: indica la priorità del requisito associato al test, e può assumere valori quali:
                \begin{itemize}
                    \item \textbf{M}: mandatory/obbligatorio;
                    \item \textbf{D}: desirable/desiderabile;
                    \item \textbf{O}: optional/ opzionale, relativamente utile ed eventualmente trascurabile.
                \end{itemize}
                \item \textbf{Categoria}: indica la tipologia di requisito associato al test, e può assumere valori quali:
                \begin{itemize}
                    \item \textbf{F}: functional/funzionale;
                    \item \textbf{P}: performance/prestazionale;
                    \item \textbf{Q}: qualitative/qualitativo;
                    \item \textbf{C}: constraint/vincolo.
                \end{itemize}
                \item \textbf{Codice}: intero positivo che identifica il singolo componente da testare. Ha valore di default 1.
            \end{itemize}

            I test di integrazione vengono classificati nella maniera seguente:
            \begin{center}
                \textbf{TI[Codice]}
            \end{center}
            dove:\\
            \begin{itemize}
                \item \textbf{Codice}: intero positivo che identifica il singolo componente da testare. Ha valore di default 1.
            \end{itemize}

            I Test di unità vengono classificati nella maniera seguente:
            \begin{center}
                \textbf{TU[Codice]}
            \end{center}
            dove:\\
            \begin{itemize}
                \item \textbf{Codice}: intero positivo che identifica il singolo componente da testare. Ha valore di default 1.
            \end{itemize}

            I Test di regressione consistono nell'esecuzione di tutti i test già effettuati in precedenza sulle unità in relazione con quella che ha subito una modifica. Non risulta dunque necessaria una nomenclatura specifica per essi.
            \\\\
            Le norme di pseudo-codice per la scrittura dei test sono pesantemente ispirate allo standard reperibile a questo indirizzo \url{https://users.csc.calpoly.edu/~jdalbey/SWE/pdl_std.html} cui sono stati aggiunti alcuni costrutti specifici per rendere lo standard più espressivo e meglio adattabile alle nostre specifiche esigenze. \\
            I costrutti fondamentali sono: \\
            \begin{table}[h]
                \begin{tabular}{|p{4,0cm} | p{10,0cm}|}
                    \hline
                    Costrutto & Definizione \\
                    \hline
                    SEQUENCE & Indica una sequenza di azioni svolte in successione in maniera lineare.\\
                    \hline
                    SET-TO & Indica l'assegnazione di un valore ad una variabile.\\
                    \hline
                    IF-THEN-ELSE & Indica una scelta tra due diversi flussi di esecuzione, che vengono scelti in base a un espressione condizionale.\\
                    \hline
                    FOR & Indica un ciclo con contatore, che quindi eseguirà un numero specifico e noto di volte.\\
                    \hline
                    FOREACH & Indica delle istruzioni da eseguire per ognuno degli elementi contenuti all'interno di una struttura dati.\\
                    \hline
                    CALL & Indica una chiamata a metodo.\\
                    \hline
                    WHEN & Specifica il comportamento di un metodo una volta richiamato all'interno dello stesso test.\\
                    \hline
                    EXPECT & Indica un valore o una condizione attesa alla fine di un test affinché esso sia valido.\\
                    \hline
                \end{tabular}
            \end{table}
            \\\\
            Viene di seguito riportata la sintassi per i costrutti di pseudo-codice esposti in precedenza.

            \begin{itemize}
                \item \textbf{SEQUENCE}: viene indicato scrivendo le azioni una in successione all'altra, ciascuna su una propria riga e verticalmente allineate. Le azioni vengono svolte dal basso verso l'alto nel modo in cui sono scritte.
                \begin{verbatim}
                    SET a TO 1
                    CALL dep.method(a)
                \end{verbatim}
                È possibile innestare costrutti tra loro, esprimendo tale casistica attraverso l'utilizzo dell'indentazione qualora un costrutto sia innestato all'interno di un altro.

                \item \textbf{SET-TO}: assegnazione di un valore ad una variabile.
                \begin{verbatim}
                    SET a TO 2
                    SET b TO a
                    EXPECT a TO EQUAL b
                \end{verbatim}

                \item \textbf{IF-THEN-ELSE}: viene indicato attraverso l'utilizzo di 4 parole chiave: IF, THEN, ELSE e ENDIF.
                \begin{verbatim}
                    IF condition THEN
                        SEQUENCE 1
                    ELSE
                        SEQUENCE 2
                    ENDIF
                \end{verbatim}

                \item \textbf{FOR}: viene indicato utilizzando 2 parole chiave, FOR e ENDFOR, e il numero di volte che il ciclo deve essere eseguito viene specificato a fianco della parola chiave FOR. É possibile specificare il numero di esecuzioni anche utilizzando del codice di dominio.
                \begin{verbatim}
                    FOR iteration bounds
                        SEQUENCE
                    ENDFOR
                \end{verbatim}

                \item \textbf{FOREACH}: viene indicato con le parole FOREACH, IN, DO e ENDFOREACH e il ciclo viene ripetuto per ogni elemento contenuto nella struttura dati.
                \begin{verbatim}
                    FOREACH number IN [1, 2, 3] DO
                        EXPECT number TO BE CONTAINED IN [1, 2, 3]
                    ENDFOREACH
                \end{verbatim}

                \item \textbf{WHEN}: successivamente a questa istruzione, il metodo indicato ritornerà il valore desiderato.
                \begin{verbatim}
                    WHEN foo(value1) = value2
                    SET bar TO value1
                    EXPECT foo(bar) TO EQUAL value2
                \end{verbatim}

                \item \textbf{CALL}: nel caso si desideri chiamare una funzione specifica di dominio, tale chiamata verrà espressa attraverso l'utilizzo della parola chiave CALL, con inclusi i paramentri che vengono passati alla suddetta funzione riportati tra parentesi tonde.
                \begin{verbatim}
                    CALL my_function()
                    CALL my_other_function(parameter1)
                    CALL one_more_function(parameter1, parameter2)
                \end{verbatim}

                \item \textbf{EXPECT}: per poter superare il test, l'elemento indicato dopo EXPECT deve rispettare la condizione riportata subito dopo.
                \begin{verbatim}
                    EXPECT foo TO EQUAL bar
                    EXPECT a NOT TO EQUAL b
                    EXPECT my_func() TO HAVE BEEN CALLED 2 TIMES WITH a, b
                    EXPECT array TO CONTAIN c, d, 1, 2
                \end{verbatim}
            \end{itemize}

		\paragraph{Sviluppo}
		Tutti i documenti ufficiali dovranno attraversare le seguenti 4 fasi:
		\begin{enumerate}
			\item \textbf{creazione}: il documento viene creato utilizzando il template \textbf{template.tex} presente nella cartella dedicata della \glock{repository} remota;
			\item \textbf{sviluppo}: i Redattori del documento, utilizzando il software \glock{TeXstudio}, aggiungono i contenuti assegnati in maniera incrementale, ed aggiornano la tabella delle modifiche senza modificare l'indice di versione del documento;
			\item \textbf{verifica}: i Verificatori controllano la correttezza dei contenuti e, se necessario, richiedono modifiche e/o integrazioni ai Redattori. L'approvazione delle modifiche da parte del verificatore, porta ad un apposito incremento dell'indice di versione del documento. Una volta terminata la fase di verifica, il documento, viene sottoposto al Responsabile di Progetto;
			\item \textbf{approvazione}: il Responsabile di Progetto ratifica il completamento del documento aggiornando la versione alla \glock{major release} successiva rendendolo pronto al rilascio.
		\end{enumerate}

		\subparagraph{Norme tipografiche}
		Oltre alla struttura dei documenti, andranno rispettate anche le seguenti norme tipografiche:
		\begin{itemize}
			\item \textbf{titoli}: solo l'iniziale in maiuscolo, tranne se sono presenti sigle particolari come nei titoli dei casi d'uso (esempio: UC2 - Login);
			\item \textbf{date}: scritte nel formato italiano, 2 cifre per il giorno, 2 cifre per il mese e 4 cifre per l'anno separati tra di loro dal carattere "-" (GG-MM-AAAA, esempio: 19-12-2020);
			\item \textbf{elenchi puntati e numerati}: ogni elemento termina con il ";", tranne l'ultimo, che conclude l'elenco con il ".";
			\item \textbf{termini nel Glossario}: tutti i termini presenti nel \dext{Glossario v4.0.0} sono scritti in \textsc{maiuscoletto} e hanno a pedice una G maiuscola \glock{};
			\item \textbf{riferimento a documenti}: tutti i riferimenti a documenti sono scritti in \textsc{maiuscoletto} e hanno a pedice una D maiuscola \dext{};
			\item \textbf{link a pagine internet}: sono direttamente cliccabili nel documento in formato elettronico.
		\end{itemize}

		\subparagraph{Norme strutturali}
		I Redattori utilizzeranno il template \textbf{template.tex} disponibile nella repository, che preimposta il documento da produrre come segue:
		\begin{itemize}
			\item \textbf{frontespizio}: la prima pagina di ogni documento contenente:
			\begin{itemize}
				\item logo del gruppo;
				\item titolo del documento;
				\item data di ultima modifica;
				\item versione attuale;
				\item classificazione, tipo d'uso (interno o esterno);
				\item Redattori (in ordine alfabetico per cognome);
				\item Verificatori (in ordine alfabetico per cognome);
				\item Responsabile;
				\item destinatari;
			\end{itemize}
			\item \textbf{registro delle modifiche}: inizia nella seconda pagina di ogni documento; in formato tabellare specifica tutti gli aggiornamenti (con data e autore) del documento in ordine cronologico inverso;
			\item \textbf{indice}: posto nella pagina successiva al diario delle modifiche, indica le varie sezioni del documento, numerando i capitoli e specificandone il numero di pagina in cui ognuno di essi inizia;
			\item \textbf{contenuto}: gli argomenti trattati nel documento;
			\item \textbf{numero di pagina}: è presente al centro di ogni piè di pagina, tranne che nella prima.
		\end{itemize}

    	\subparagraph{Nomenclatura}
    	Ogni documento, durante le fasi di sviluppo e verifica, sarà all'interno di una cartella specifica e suddiviso in diversi file in formato \glock{\LaTeX}:
    	\begin{itemize}
    		\item un file principale main.tex contente il frontespizio, il diario delle modifiche, l'indice e la struttura già suddivisa nelle varie sezioni previste del documento;
    		\item un file per ogni sezione che sarà automaticamente importato nella posizione prevista del file principale.
    	\end{itemize}
    	Dopo l'approvazione del Responsabile di Progetto verrà convertito in formato \glock{PDF} e rinominato secondo le seguenti regole:
    	\begin{itemize}
    		\item \textbf{nome\_del\_documento\_}: ogni parola del nome, compresa l'ultima, verrà seguita dal carattere \glock{underscore} e verranno utilizzate solo lettere minuscole senza eventuali accenti;
    		\item \textbf{vX.Y.Z}: il carattere "v" sarà sempre presente e precederà il numero di versione del documento.
    	\end{itemize}
    	Per esempio, il nome di questo file potrebbe essere: norme\_di\_progetto\_v1.0.0

	\subsubsection{Strumenti}
	La documentazione prodotta sarà in formato \LaTeX\ v2e (\url{https://www.latex-project.org/}) e per la stesura saranno utilizzati i seguenti software:
	\begin{itemize}
		\item \textbf{\glock{TeXstudio} v3.0.1} (\url{https://texstudio.org/}): per l'editing dei documenti \LaTeX\ v2e, compatibile con tutti i sistemi operativi utilizzati dal gruppo (\glock{Mac OS}, \glock{Windows}, \glock{Linux}). Sarà, inoltre, obbligatorio attivare il correttore ortografico in italiano del software per ridurre al minimo gli eventuali errori di digitazione;
		\item \textbf{\glock{Texmaker} v5.0.4}: altro editor \LaTeX\ come \glock{TeXstudio};
		\item \textbf{\glock{GanttProject} v2.8.11} (\url{https://www.ganttproject.biz/}): per la creazione dei diagrammi di \glock{Gantt};
		\item \textbf{\glock{Diagrams}} (\url{https://www.diagrams.net/}): per la creazione dei diagrammi \glock{UML};
		\item \textbf{\glock{Libreoffice Calc} v6.4.6.2} (\url{https://it.libreoffice.org/}): per i calcoli sulle varie tabelle di ore  e prezzi e la creazione dei relativi grafici.
	\end{itemize}

	\subsubsection{Metriche}
	Ogni documento verrà analizzato automaticamente da una \glock{GitHub Actions} per la verifica delle seguenti due metriche che ne testano la qualità di comprensione:

		\paragraph{MD01 - Indice di \glock{Gulpease}}
		\begin{itemize}
			\item \textbf{Descrizione}: è un valore quantificabile con una formula tarata direttamente sulla lingua italiana. Per ottenerlo, dal testo di interesse, si estrapolano: il numero delle frasi, il numero delle parole ed il numero delle lettere.
			Una volta misurati i dati, si applica la formula:
            \[
				MD01 = 89 + {{(300 \cdot numeroFrasi)} - {(10 \cdot numeroLettere)} \over {numeroParole}}
            \]
			\item \textbf{Unità di misura}: numero intero maggiore o uguale a 0;
			\item \textbf{Risultato}: L'indice ottenuto andrà interpretato in questo modo:
			\begin{itemize}
				\item \textbf{minore di 40}: difficilmente comprensibile per un lettore con diploma superiore;
				\item \textbf{tra 40 e 60}: difficilmente comprensibile per un lettore con licenza media;
				\item \textbf{tra 60 e 80}: difficilmente comprensibile per un lettore con licenza elementare;
				\item \textbf{maggiore di 80}: comprensibile per tutti.
			\end{itemize}
		\end{itemize}

		\paragraph{MD02 - Correttezza ortografica}
		\begin{itemize}
			\item \textbf{Descrizione}: oltre alla verifica automatizzata è comunque necessario fare attenzione durante la stesura dei testi, tenendo sempre i correttori ortografici attivati in modo da evidenziare subito eventuali errori. In fase di revisione, ai verificatori, è richiesta la lettura dello stesso documento più volte;
			\item \textbf{Unità di misura}: numero intero maggiore o uguale a 0.
		\end{itemize}
