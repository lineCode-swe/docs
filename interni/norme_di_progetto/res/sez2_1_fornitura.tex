\subsection{Fornitura}

	\subsubsection{Descrizione}
	L'obiettivo di questa sezione è descrivere le norme che il gruppo \textit{lineCode} si impegna a rispettare per potersi proporre come fornitore nei confronti di Sanmarco Informatica SpA e dei committenti Prof. Tullio Vardanega e Prof. Riccardo Cardin per la progettazione, sviluppo e consegna del progetto.

	\subsubsection{Attività}
		\paragraph{Studio di fattibilità}
            \subparagraph{Scopo}
    		Documento che riporta lo studio svolto dagli analisti sui capitolati proposti. Per ciascun \glock{capitolato} viene riportato:
    		\begin{itemize}
    			\item \textbf{descrizione}: sintesi del prodotto richiesto da sviluppare presentato nel \glock{capitolato};
    		 	\item \textbf{finalità}: ambito di utilizzo del prodotto da sviluppare;
    		 	\item \textbf{tecnologie interessate:} lista di tutte le tecnologie interessate nello sviluppo del prodotto;
    		 	\item \textbf{analisi motivazione, criticità e rischi}: racchiude le motivazioni, le criticità e i rischi risultati dall'analisi del \glock{capitolato};
    		 	\item \textbf{valutazione finale}: indica le motivazioni per le quali il \glock{capitolato} è stato respinto o accettato.
    		\end{itemize}

		\paragraph{Documentazione esterna}
            \subparagraph{Scopo}
    		Il gruppo \textit{lineCode} si impegna a fornire al proponente Sanmarco Informatica SpA e ai committenti Prof. Tullio Vardanega e Prof. Riccardo Cardin i seguenti documenti:
    		\begin{itemize}
    		 	\item \dext{Piano di progetto v2.0.0}: documento che descrive le metodologie di pianificazione, consegna e completamento del progetto;
    		 	\item \dext{Piano di qualifica v2.0.0}: documento che contiene le attività di verifica, validazione e garantisce la qualità dei processi e di prodotto;
    		 	\item \dext{Analisi dei requisiti v2.0.0}: documento contenente l'analisi dei requisiti e dei casi d'uso del gruppo;
    		 	\item \dext{Glossario v2.0.0}: documento con tutti i termini che potrebbero essere di difficile comprensione presenti nella documentazione.
    		\end{itemize}

    	\paragraph{Rapporto con il proponente}
            \subparagraph{Scopo}
        	Il gruppo intende instaurare un dialogo costante ed un profondo rapporto di collaborazione con il proponente Sanmarco Informatica SpA, al fine di:
        	\begin{itemize}
        		\item determinare i bisogni del proponente;
        		\item stabilire vincoli e requisiti dei processi;
        		\item stabilire vincoli e requisiti del prodotto;
        		\item stimare tempistiche e costi del lavoro;
        		\item accordarsi sulla qualifica di prodotto;
        		\item chiarire eventuali dubbi emersi.
        	\end{itemize}