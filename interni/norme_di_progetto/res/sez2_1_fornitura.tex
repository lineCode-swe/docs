\subsection{Fornitura}

	\subsubsection{Descrizione}
	L'obiettivo di questa sezione è descrivere le norme che il gruppo \textit{lineCode} si impegna a rispettare per potersi proporre come fornitore nei confronti di Sanmarco Informatica SpA e dei committenti Prof. Tullio Vardanega e Prof. Riccardo Cardin per analisi, progettazione, sviluppo e consegna del progetto.

	\subsubsection{Attività}
		\paragraph{Studio di fattibilità}
            Il gruppo \textit{lineCode} si impegna a valutare tutti i capitolati d'appalto sulla base delle proprie conoscenze e propensioni, redigendo lo \dext{Studio di Fattibilità 1.0.0} al fine di tener traccia di quanto emerso da tali valutazioni. Si arriverà in fondo a questa attività con un unico capitolato scelto.

        \paragraph{Contratto}
            Il fornitore analizza il capitolato scelto e propone, a committente e proponente, una serie di requisiti che intende portare a compimento a fine del progetto. Ciò avviene tramite l'\dext{Analisi dei requisiti 3.0.0} che funge da contratto verso il committente.
            \\\\
            I requisiti sono classificati per importanza in modo da variare dinamicamente il livello di completezza del prodotto nel caso in cui i consuntivi di periodo e le metriche predicessero un superamento del budget previsto dal preventivo iniziale.
            \\\\
            Data l'inevitabile inesperienza del fornitore, quest'ultimo si riserva il diritto di modificare i requisiti in corso d'opera per mantenerli coerenti alla richiesta del proponente e alle tecnologie utilizzate per il loro soddisfacimento (la cui esplorazione avverrà in un momento successivo alla proposta del contratto).

		\paragraph{Pianificazione}
            Il fornitore si impegna a pianificare l'attività di progetto scegliendo un opportuno modello di sviluppo software, dando un preventivo iniziale e, per ogni fase individuata, dei consuntivi di periodo comprensivi di preventivi a finire con un consuntivo finale a fine progetto.
            \\\\
            Queste informazioni sono reperibili nel \dext{Piano di progetto 4.0.0} e comunicano al committente le strategie messe in atto per arrivare al soddisfacimento dei requisiti oltre che a mostrare l'andamento del progetto in termini di ore impiegate e budget.
