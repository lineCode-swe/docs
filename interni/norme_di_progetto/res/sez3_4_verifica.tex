\subsection{Verifica}
	\subsubsection{Descrizione}
	Il processo, mira a garantire la correttezza del prodotto stabilendo dei criteri precisi con cui giudicarlo.\\
	La verifica ha lo scopo di individuare e correggere potenziali difetti nel prodotto e precede l'attuazione del processo di validazione.

    \subsubsection{Attività}
        \paragraph{Implementazione del processo}
        Su richiesta del committente, è necessario che vi sia un'adeguato sforzo nella verifica per garantire una qualità adeguata.\\
        Il Progettista si impegna progettare test che siano ritenuti adeguati a tale sforzo e riportare il risultato di questa progettazione nel documento di \dext{Specifica dei test 2.0.0} per poi lasciare al Programmatore l'effettiva implementazione.

	\subsubsection{Strumenti}
	Si utilizzano le funzionalità di correzione automatica e di controllo ortografico degli \glock{IDE} utilizzati dal gruppo, che sono:
	\begin{itemize}
		\item \glock{IntelliJ IDEA};
		\item \glock{TeXStudio};
		\item \glock{Texmaker}.
	\end{itemize}
    Le suite di test automatizzato sono:
    \begin{itemize}
        \item \glock{JUnit}: per la componente Server in linguaggio \glock{Java};
        \item \glock{Jasmine}: per la componente Ui in linguaggio \glock{TypeScript};
        \item \glock{Mocha}: per la componente Unità in linguaggio \glock{TypeScript}.
    \end{itemize}

    \subsection{Metriche}
        \paragraph{MT01 - Code Coverage}
        \begin{itemize}
            \item \textbf{Descrizione}: indica quante linee di codice del prodotto software sono coperte da test;
            \item \textbf{Unità di misura}: percentuale.
        \end{itemize}

        \paragraph{MT02 - Test di sistema implementati e superati}
        \begin{itemize}
            \item \textbf{Descrizione}: indica il numero di test di sistema superati sul numero di test di sistema implementati;
            \item \textbf{Unità di misura}: rapporto.
        \end{itemize}

        \paragraph{MT03 - Test d'integrazione implementati e superati}
        \begin{itemize}
            \item \textbf{Descrizione}: indica il numero di test d'integrazione superati sul numero di test d'integrazione implementati;
            \item \textbf{Unità di misura}: rapporto.
        \end{itemize}

        \paragraph{MT04 - Test di unità implementati e superati}
        \begin{itemize}
            \item \textbf{Descrizione}: indica il numero di test di unità superati sul numero di test di unità implementati;
            \item \textbf{Unità di misura}: rapporto.
        \end{itemize}
